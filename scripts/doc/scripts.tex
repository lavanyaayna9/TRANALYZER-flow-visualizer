\IfFileExists{t2doc.cls}{
    \documentclass[documentation]{subfiles}
}{
    \errmessage{Error: could not find 't2doc.cls'}
}

\begin{document}

\trantitle
    {scripts}
    {Various Scripts and Utilities}
    {Tranalyzer Development Team} % author(s)

\section{scripts}\label{s:scripts}
This section describes various scripts and utilities for Tranalyzer.
For a complete list of options, use the scripts {\tt --h} option.

\subsection{b64ex}
Extracts all HTTP, EMAIL, FTP, TFTP, etc base 64 encoded content extracted from T2.
To produce a list of files containing base64 use {\tt grep} as indicated below:
\begin{itemize}
    \item {\tt grep "base64" /tmp/SMTPFILE/*}
    \item {\tt ./b64ex /tmp/SMTPFILES/file@wurst.ch\_0\_1223}
\end{itemize}

\subsection{fpsGplt}
Transforms the output of the \tranrefpl{nFrstPkts} plugin signal output to gnuplot or \nameref{t2plot} format for encrypted traffic mining purposes.
It generates an output file: {\tt flowfile\_nps.txt} containing the processed PL signal according to \tranrefpl{nFrstPkts} plugin configuration.

\begin{center}
\begin{verbatim}
$ fpsGplt -h
Usage:
    fpsGplt [OPTION...] <FILE_flows.txt>

Optional arguments:
    -f findex        Flow index to extract [default: all flows]
    -d A|B           Flow direction: A or B only [default: A and B]
    -t               No time, but counts on x axis [default: time on x axis]
    -i               Invert B flow PL
    -s               Time sorted ascending
    -p s             Sample sorted signal with smplIAT in [s]; f = 1/smplIAT
    -e s             Time for each PL pulse edge in [s]
    -j               Calculate the jumps in IAT and report appropriate values
                     for MINIAT(S/U)
    -h, --help       Show this help, then exit
\end{verbatim}
\end{center}

If {\tt --f} is omitted all flows will be included.
If {\tt --d } is omitted both flow directions will be processed.
{\tt --t} removes the timestamp and replaces it with an integer count.
{\tt --i} inverts the {\tt B} flow signal to produce a symmetrical signal.
{\tt --p} samples the sorted signal with the IAT in seconds resp.\ frequency you deem necessary and {\tt --e} defines the pulse flank in seconds.
{\tt --j} calculates the jumps in IAT to allows the user to choose an appropriate {\tt MINIAT(S/U)} in \tranrefpl{nFrstPkts} plugin.

%\subsection{gpcc}
%3D plot for connectionCounter.
%\begin{center}
%    {\tt cat FILE\_connection | ./gpcc | gnuplot -p}
%\end{center}
%The script can be configured through the command line. For a full list of options, run {\tt ./gpcc --help}

\subsection{gpq3x}
Use this script to create 3D waterfall plot.
Was originally designed for the \tranrefpl{centrality} plugin:
\begin{center}
    {\tt cat FILE\_centrality | ./gpq3x}
\end{center}
The script can be configured through the command line.
For a full list of options, run {\tt ./gpq3x --help}

\subsection{osStat}
Counts the number of hosts of each operating system (OS) in a PCAP file.
In addition, a file with suffix {\tt \_IP\_OS.txt} mapping every IP to its OS is created.
This script uses {\tt p0f} which requires a fingerprints file ({\tt p0f.fp}), the location of which can be specified using the {\tt --f} option.
Version 2 looks first in the current directory, then in {\tt /etc/p0f}.
Version 3 looks only in the current directory.
\begin{itemize}
    \item list all the options: {\tt osStat --{}--help}
    \item top 10 OS: {\tt osStat file.pcap --n 10}
    \item bottom 5 OS: {\tt osStat file.pcap --n --5}
\end{itemize}

%\subsection{plot\_monitoring}

\subsection{protStat}\label{protStat}
The {\tt protStat} script can be used to sort the {\tt PREFIX\_protocols.txt} file (generated by the \tranrefpl{protoStats} plugin) or the {\tt PREFIX\_nDPI.txt} file (generated by the \tranrefpl{nDPI} plugin) for the most or least occurring protocols (in terms of number of packets or bytes).
In addition, it can also sort the {\tt PREFIX\_icmpStats.txt} and {\tt PREFIX\_igmpStats.txt} files (generated by the \tranrefpl{icmpDecode} and \tranrefpl{igmpDecode} plugins respectively).
It can output the top or bottom $N$ protocols or only those with at least a given percentage:
\begin{itemize}
    \item list all the options: {\tt protStat --{}--help}
    \item for better readability, use {\tt protStat} with {\tt tcol}: {\tt protStat ... | tcol}
    \item sorted list of protocols (by packets): {\tt protStat PREFIX\_protocols.txt}
    \item sorted list of protocols (by bytes): {\tt protStat PREFIX\_protocols.txt --b}
    \item top 10 protocols (by packets): {\tt protStat PREFIX\_protocols.txt --n 10}
    \item bottom 5 protocols (by bytes): {\tt protStat PREFIX\_protocols.txt --n --5 --b}
    \item protocols with packets percentage greater than 20\%: {\tt protStat PREFIX\_protocols.txt --p 20}
    \item protocols with bytes percentage smaller than 5\%: {\tt protStat PREFIX\_protocols.txt --b --p --5}
    \item TCP and UDP statistics only: {\tt protStat PREFIX\_protocols.txt --udp --tcp}
\end{itemize}

% Not part of the release
%\subsection{segvtrack}
%If the processing of a pcap file causes a segmentation fault, this script can be used to locate the packets which caused the error. It works by repetitively splitting the file in half until neither half causes a segmentation fault.
%Its usage is as follows:
%\begin{verbatim}
%segvtrack file.pcap
%\end{verbatim}
%Note that you might need to change the path to the Tranalyzer binary by editing the {\tt T2} variable at line 5 of the script.

\subsection{statGplt}
Transforms 2/3D statistics output from \tranrefpl{pktSIATHisto} plugin to gnuplot or \nameref{t2plot} format for encrypted traffic mining purposes.

\subsection{t2\_aliases}\label{t2aliases}
Set of aliases for Tranalyzer.

\subsubsection{Description}
{\tt t2\_aliases} defines the following aliases, functions and variables:

\paragraph{T2HOME}~\\
Variable pointing to the root folder of Tranalyzer, e.g., {\tt cd \$T2HOME}.
\paragraph{T2PLHOME}~\\
Variable pointing to the root folder of Tranalyzer plugins, e.g., {\tt cd \$T2PLHOME}.
In addition, every plugin can be accessed by typing its name instead of its full path.
For example to access \tranrefpl{tcpFlags} home folder, {\tt tcpFlags} can be used instead of {\tt cd \$T2PLHOME/tcpFlags} or {\tt cd \$T2HOME/plugins/tcpFlags}.
\paragraph{tran}~\\
Shortcut to access {\tt\$T2HOME}, e.g., {\tt tran}
\paragraph{tranpl}~\\
Shortcut to access {\tt\$T2PLHOME}, e.g., {\tt tranpl}
\paragraph{.tran}~\\
Shortcut to access {\tt\$HOME/.tranalyzer/plugins}, e.g., {\tt .tran}
\paragraph{awkf}~\\
Configures {\tt awk} to use tabs, i.e., `{\tt\textbackslash{}t}' as input and output separator (prevents issue with repetitive values), e.g.,\\
{\tt awkf `\{ print \$4 \}' file\_flows.txt}
\paragraph{tawk}~\\
Shortcut to run \tranrefpl{tawk} from anywhere, e.g., {\tt tawk}
\paragraph{tcol}~\\
Displays columns with minimum width, e.g., {\tt tcol file\_flows.txt}.
\paragraph{lsx}~\\
Displays columns with fixed width (default: 40), e.g., {\tt lsx file\_flows.txt} or {\tt lsx 45 file\_flows.txt}.\\
Note that ZSH already defines a {\tt lsx} alias, therefore if using ZSH this command will {\bf NOT} be installed.
To have it installed, add the following line to your {\tt \textasciitilde{}/.zshrc} file: {\tt unalias lsx}
\paragraph{sortu}~\\
Sort rows and count the number of times a given row appears, then sort by the most occurring rows.
(Alias for {\tt sort | uniq -c | sort -rn}).
Useful, e.g., to analyze the most occurring user-agents: {\tt tawk `\{ print \$httpUsrAg \}' FILE\_flows.txt | sortu}
\paragraph{sortup}~\\
Same as {\tt sortu}, but display the relative percentage instead of the absolute count.
e.g., to analyze the most occurring user-agents: {\tt tawk `\{ print \$httpUsrAg \}' FILE\_flows.txt | sortup}
\paragraph{t2}~\\
Shortcut to run Tranalyzer from anywhere, e.g., {\tt t2 -r file.pcap -w out}
\paragraph{gt2}~\\
Shortcut to run Tranalyzer in {\tt gdb} (Linux) or {\tt lldb} (macOS) from anywhere, e.g., {\tt gt2 -r file.pcap -w out}
\paragraph{st2}~\\
Shortcut to run Tranalyzer with sudo, e.g., {\tt st2 -i eth0 -w out}
\paragraph{tranalyzer}~\\
Shortcut to run Tranalyzer from anywhere, e.g., {\tt tranalyzer -r file.pcap -w out}
\paragraph{fextractor}~\\
Shortcut to run {\tt\tranref{fextractor}} from anywhere, e.g., {\tt fextractor -r file\_flows.xer 1234}
\paragraph{fpsGplt}~\\
Shortcut to run {\tt\tranref{fpsGplt}} from anywhere, e.g., {\tt fpsGplt file\_flows.txt}
\paragraph{protStat}~\\
Shortcut to run {\tt\tranref{protStat}} from anywhere, e.g., {\tt protStat file\_protocols.txt}
\paragraph{statGplt}~\\
Shortcut to run {\tt\tranref{statGplt}} from anywhere, e.g., {\tt statGplt file\_flows.txt}
\paragraph{t2b2t}~\\
Shortcut to run \nameref{t2b2t} from anywhere, e.g., {\tt t2b2t -r file\_flows.bin -w file\_flows.txt}.
\paragraph{t2build}~\\
Function to build Tranalyzer and the plugins from anywhere, e.g., {\tt t2build tcpFlags}.
Use {\tt <tab>} to list the available plugins and complete names.
Use {\tt t2build -h} for a full list of options.
\paragraph{t2caplist}~\\
Shortcut to run \nameref{t2caplist} from anywhere, e.g., {\tt t2caplist}
\paragraph{t2conf}~\\
Shortcut to run \nameref{t2conf} from anywhere, e.g., {\tt t2conf --{}--gui}
\paragraph{t2dmon}~\\
Shortcut to run \nameref{t2dmon} from anywhere, e.g., {\tt t2dmon dumps/}
\paragraph{t2doc}~\\
Shortcut to run \nameref{t2doc} from anywhere, e.g., {\tt t2doc tranalyzer2}
\paragraph{t2docker}~\\
Shortcut to run \nameref{t2docker} from anywhere, e.g., {\tt t2docker --r file.pcap}
\paragraph{t2dpdk}~\\
Shortcut to run \nameref{t2dpdk} from anywhere, e.g., {\tt t2dpdk -N 4 -i 0000:04:00.0}.
\paragraph{t2flowstat}~\\
Shortcut to run \nameref{t2flowstat} from anywhere, e.g., {\tt t2flowstat file\_flows.txt --c pktsSnt -s 1 -m 9000 -0}
\paragraph{t2fm}~\\
Shortcut to run \nameref{t2fm} from anywhere, e.g., {\tt t2fm --r file.pcap}
\paragraph{t2fuzz}~\\
Shortcut to run \nameref{t2fuzz} from anywhere, e.g., {\tt t2fuzz file.pcap}
\paragraph{t2locate}~\\
Shortcut to run \nameref{t2locate} from anywhere, e.g., {\tt t2locate}
\paragraph{t2mmdb}~\\
Shortcut to run {\tt t2mmdb} (see \tranrefpl{geoip} plugin documentation for more information) from anywhere, e.g., {\tt t2mmdb}
\paragraph{t2netID}~\\
Shortcut to run \nameref{t2netID} from anywhere, e.g., {\tt t2netID 0x138020a5}
\paragraph{t2plot}~\\
Shortcut to run \nameref{t2plot} from anywhere, e.g., {\tt t2plot file.txt}
\paragraph{t2plugin}~\\
Shortcut to run \nameref{t2plugin} from anywhere, e.g., {\tt t2plugin -c pluginName}.
\paragraph{t2rrd}~\\
Shortcut to run \nameref{s:t2rrd} from anywhere, e.g., {\tt t2 -i eth0 | t2rrd -m} or {\tt t2rrd V4Pkts V6Pkts}
\paragraph{t2stat}~\\
Shortcut to run \nameref{t2stat} from anywhere, e.g., {\tt t2stat -USR2}
\paragraph{t2test}~\\
Shortcut to run {\tt tests/T2Tester.py} from anywhere, e.g., {\tt t2test tranalyzer2}
\paragraph{t2timeline}~\\
Shortcut to run \nameref{t2timeline} from anywhere, e.g., {\tt t2timeline file.txt}
\paragraph{t2topcap}~\\
Shortcut to run \nameref{t2topcap} from anywhere, e.g., {\tt t2topcap file.pcapng}
\paragraph{t2update}~\\
Shortcut to run {./setup.sh --C} from anywhere, e.g., {\tt t2update}.
This alias checks for the availability of a new version of Tranalyzer and proceed with the installation if requested.
\paragraph{t2viz}~\\
Shortcut to run \nameref{t2viz} from anywhere, e.g., {\tt t2viz file.txt}
\paragraph{t2whois}~\\
Shortcut to run \nameref{t2whois} from anywhere, e.g., {\tt t2whois 1.2.3.4}

\subsubsection{Usage}
Those aliases can be activated using either one of the following methods:

\begin{enumerate}
    \item Append the content of this file to {\tt\textasciitilde{}/.bash\_aliases} or {\tt\textasciitilde{}/.bashrc}
    \item Append the following line to {\tt\textasciitilde{}/.bashrc} (make sure to replace {\tt\$T2HOME}
          with the actual path, e.g.,\\{\tt\$HOME/tranalyzer2-0.9.3}):
\begin{lstlisting}
if [ -f "$T2HOME/scripts/t2_aliases" ]; then
    . $T2HOME/scripts/t2_aliases             # Note the leading `.'
fi
\end{lstlisting}
\end{enumerate}

\subsubsection{Known Bugs and Limitations}
ZSH already defines a {\tt lsx} alias, therefore if using ZSH this command will {\bf NOT} be installed.
To have it installed, add the following line to your {\tt \textasciitilde{}/.zshrc} file: {\tt unalias lsx}

\subsection{t2alive}\label{t2alive}
In order to monitor the status of T2, the t2alive script sends syslog messages to server defined by the user whenever the status of T2 changes.
It acquires the PID of the T2 process and transmits every {\tt REP} seconds a {\tt kill -SYS \$pid}.
If T2 answers with a corresponding kill command defined in {\em tranalyzer.h}, s.b., then status is set to alive, otherwise to dead.
Only if a status change is detected a syslog message is transmitted.
The following constants residing in {\em tranalyzer.h} govern the functionality of the script:

\begin{table}[!ht]
\centering
\begin{tabular}{>{\tt}lll}
    \toprule
    {\bf Name} & {\bf Default}                  & {\bf Description}\\
    \midrule
    SERVER     & {\tt\small "127.0.0.1"}        & syslog server IP\\
    PORT       & 514                            & syslog server port\\
    FAC        & {\tt\small "<25>"}             & facility code\\
    STATFILE   & {\tt\small "/tmp/t2alive.txt"} & alive status file\\
    REP        & 10                             & T2 test interval [s]\\
    \bottomrule
\end{tabular}
\caption{t2alive script configuration}
\end{table}

T2 on the other hand has also to be configured.
To preserve simplicity the unused SYS interrupt was abused to respond to the {\tt t2alive} request, hence the monitoring mode depending on USR1 and USR2 can be still functional.
Configuration is carried out in {\em tranalyzer.h} according to the table below:

\begin{table}[!ht]
\centering
\small
\begin{tabular}{>{\tt}lll}
    \toprule
    {\bf Name} & {\bf Default} & {\bf Description}\\
    \midrule
    REPSUP   & 0                     & 1: activate alive mode\\
    ALVPROG  & {\tt\small "t2alive"} & name of control program\\

    REPCMDAW & {\tt\small "a=`pgrep ALVPROG`; if [ \$a ]; then kill -USR1 \$a; fi"}
                                     & alive and stall (no packets, looping?)\\
    REPCMDAS & {\tt\small "a=`pgrep ALVPROG`; if [ \$a ]; then kill -USR2 \$a; fi"}
                                     & alive and well (working)\\
    \bottomrule
\end{tabular}
\caption{T2 configuration for t2alive mode}
\end{table}

{\tt REPSUP=1} activates the alive mode.
If more functionality is requested the {\tt REPCMDAx} constant facilitates the necessary changes.
On some Linux distributions the pcap read callback function is not thread safe, thus signals of any kind might lead to crashes especially when capturing live traffic.
Therefore {\bf MONINTTHRD=1} in {\em main.h} is set by default. \\
Note that t2alive should be executed in a shell as a standalone script.
If executed as a cron job, the while loop and the sleep command has to be removed, as described in the script itself.

\subsection{t2b2t}\label{t2b2t}
The program {\tt t2b2t} can be used to transform binary Tranalyzer files generated by the \tranrefpl{binSink} or \tranrefpl{socketSink} plugin into text or json files.
The converted files use the same format as the ones generated by the \tranrefpl{txtSink} or \tranrefpl{jsonSink} plugin.\\

The program can be found in {\tt\$T2HOME/utils/t2b2t} and can be compiled by typing {\tt make}.\\

The use of the program is straightforward:
\begin{itemize}
    \item bin$\rightarrow$txt: {\tt t2b2t -r FILE\_flows.bin -w FILE\_flows.txt}
    \item bin$\rightarrow$json: {\tt t2b2t -r FILE\_flows.bin -j -w FILE\_flows.json}
    \item bin$\rightarrow$compressed txt: {\tt t2b2t -r FILE\_flows.bin -c -w FILE\_flows.txt.gz}
    \item bin$\rightarrow$compressed json: {\tt t2b2t -r FILE\_flows.bin -c -j -w FILE\_flows.json.gz}
\end{itemize}

If the {\tt --w} option is omitted, the destination is inferred from the input file, e.g., the examples above would produce the same output files with or without the {\tt --w} option. Note that {\tt --w --} can be used to output to stdout.\\
Additionally, the {\tt --n} option can be used {\bf not} to print the name of the columns as the first row.\\
Try {\tt t2b2t -h} for more information.

\subsection{t2caplist}\label{t2caplist}
Generates a list of PCAP files with absolute path to use with Tranalyzer {\tt --R} option.
If no argument is provided, then lists all the PCAP files in the current directory.
If a folder name is given, lists all capture files in the folder.
If a list of files is given, list those files.
Try {\tt t2caplist --help} for more information.
\begin{itemize}
    \item {\tt t2caplist > pcap\_list.txt}
    \item {\tt t2caplist \textasciitilde/dumps/ > pcap\_list.txt}
    \item {\tt t2caplist \textasciitilde/dumps/testnet*.pcap > pcap\_list.txt}
\end{itemize}

\subsection{t2conf}\label{t2conf}
Use {\tt t2conf} to configure, activate and deactivate Tranalyzer plugins:
\begin{itemize}
    \item To change the value of a configuration flag run:
          \begin{center}
              {\tt t2conf pluginName --D FLAG\_NAME=new\_value}
          \end{center}
    \item To check the value of a configuration flag run:
          \begin{center}
              {\tt t2conf pluginName --G FLAG\_NAME}
          \end{center}
    \item To list the configuration flags available run:
          \begin{center}
              {\tt t2conf pluginName --I}
          \end{center}
    \item To reset a plugin configuration to its default values run:
          \begin{center}
              {\tt t2conf pluginName --{}--reset}
          \end{center}
    \item To save a plugin configuration run:
          \begin{center}
              {\tt t2conf pluginName --g}
          \end{center}
          \begin{itemize}
              \item Note that the {\tt --g} option accepts a filename:
                    \begin{center}
                        {\tt t2conf pluginName --g /where/to/save/file.config}
                    \end{center}
          \end{itemize}
    \item To load plugin configuration run:
          \begin{center}
              {\tt t2conf pluginName --C /path/to/file.config}
          \end{center}
          \begin{itemize}
              \item Note that if the default filename is used, {\tt --C auto} can be used instead:
                    \begin{center}
                        {\tt t2conf pluginName --C auto}
                    \end{center}
          \end{itemize}
    \item For more details about {\tt t2conf}, run:
          \begin{center}
              {\tt t2conf --{}--help}
          \end{center}
\end{itemize}

Alternatively, use {\tt t2conf --{}--gui} option or the {\tt t2plconf} script provided with all the plugins to configure individual plugins as follows:

\begin{itemize}
    \item {\tt cd \$T2PLHOME/pluginName}
    \item {\tt ./t2plconf}
        \begin{itemize}
            \item Navigate through the different options with the up and down arrows
            \item Use the left and right arrows to select an action:
                \begin{itemize}
                    \item {\tt ok}: apply the changes
                    \item {\tt configure}: edit the selected entry (use the space bar to select a different value)
                    \item {\tt cancel}: discard the changes
                    \item {\tt edit}: open the file containing the selected option in {\tt EDITOR} (default: {\tt vim})
                \end{itemize}
            \item Use the space bar to select a different value
        \end{itemize}
\end{itemize}

A more detailed description of the script can be found in \href{../../tranalyzer2/doc/tranalyzer2.pdf}{Tranalyzer2 documentation}.

\subsubsection{Dependencies}
The {\tt t2conf} (if the {\tt --gui} option is used) and {\tt t2plconf} scripts require {\em dialog} (version 1.1-20120703 minimum) and the {\em vim} editor.
The easiest way to install them is to use the {\tt install.sh} script provided (\refs{t2confinstall}).
Note that the editor can be changed by exporting the environment variable {\tt EDITOR} as follows: {\tt export EDITOR=/path/to/editor}, e.g., {\tt export EDITOR=/usr/bin/nano} or
by setting the {\tt EDITOR} variable at line 9 of the {\tt t2conf} script and at line 90 of the {\tt t2plconf} script.

\subsubsection{Installation}\label{t2confinstall}
The easiest way to install {\tt t2conf} and its dependencies is to use tranalyzer2 {\tt setup.sh} script.
Alternatively, the provided {\tt install.sh} script can be used to only install {\tt t2conf}.
Run {\tt ./install.sh --{}--help} to see what can be installed.

Alternatively, use \nameref{t2aliases} or add the following alias to {\tt \textasciitilde{}/.bash\_aliases}:
\begin{center}
    {\tt alias t2conf="\$T2HOME/scripts/t2conf/t2conf"}
\end{center}
Where {\tt \$T2HOME} is the root folder containing the source code of Tranalyzer2 and its plugins, i.e., where {\tt README.md} is located.

\subsubsection{Usage}
For a complete list of options use the {\tt --h} option, i.e., {\tt t2conf --h}, or the man page ({\tt man t2conf}).

\subsubsection{Patch}\label{t2confpatch}
{\tt t2conf} can be used to patch Tranalyzer and the plugins (useful to save settings such as hash table size, IPv6, \ldots).\\\\
The format of the patch file is as follows:
\begin{itemize}
    \item Empty lines and lines starting with {\tt `\%'} or {\tt `\#'} are ignored
    \item Filenames are relative to {\tt \$T2HOME}
    \item A line is composed of three or four tabs (not spaces) separated columns:
        \begin{itemize}
            \item {\tt NAME <tab> newvalue <tab> oldvalue <tab> file}
            \item {\tt NAME <tab> newvalue <tab> file}
        \end{itemize}
    \item {\tt --{}--patch} uses newvalue
    \item {\tt --{}--rpatch} uses oldvalue\footnote{This option is not valid if the patch has only three columns.}
\end{itemize}

As an example, let us take the value {\tt T2PSKEL\_IP} defined in {\em t2PSkel/src/t2PSkel.h}:
\begin{center}
    {\tt \#define T2PSKEL\_IP~~~1 // whether or not to output IP (var2)}
\end{center}
A patch to set this value to 0 would look as follows (where the spaces between the columns are tabs, i.e., {\tt `\textbackslash{}t'}):
\begin{itemize}
    \item {\tt T2PSKEL\_IP \qquad 0 \qquad 1 \qquad t2PSkel/src/t2PSkel.h}
    \item {\tt T2PSKEL\_IP \qquad 0 \qquad t2PSkel/src/t2PSkel.h}
\end{itemize}

\subsection{t2dmon}\label{t2dmon}
Monitors a folder for new files and creates symbolic links with incrementing indexes.
This can be used with the {\tt --D} option when the filenames have either multiple indexes, e.g., date and count, or when the filenames do not possess an index.

\subsubsection{Dependencies}
This script requires {\bf inotify-tools}:
\paragraph{Arch:} {\tt sudo pacman -S inotify-tools}
\paragraph{Fedora:} {\tt sudo yum install inotify-tools}
\paragraph{Gentoo:} {\tt sudo emerge inotify-tools}
\paragraph{Ubuntu:} {\tt sudo apt-get install inotify-tools}

\subsubsection{Usage}
t2dmon works as a daemon and as such, should either be run in the background (the ampersand {\tt `\&'} in step 1 below) or on a different terminal.
\begin{enumerate}
    \item {\tt t2dmon dumps/ -o nudel.pcap \&}
    \item {\tt tranalyzer -D dumps/nudel.pcap0 -w out}
    \item Finally, copy/move the pcap files into the {\tt dumps/} folder.
\end{enumerate}

\subsection{t2doc}\label{t2doc}
Access Tranalyzer documentation from anywhere, e.g., {\tt t2doc tcpFlags}.
Use {\tt <tab>} to list the available plugins and complete names.

\subsection{t2docker}\label{t2docker}
Create and manage Tranalyzer Docker containers.
\begin{itemize}
    \item Create a Docker container: {\tt t2docker -B t2-latest.tar.gz}
    \item Download the latest version of T2 and create a Docker container: {\tt t2docker -B latest}
    \item List existing Tranalyzer Docker containers: {\tt t2docker -ls}
    \item Save a Docker image: {\tt t2docker -S image-name-or-id}
    \item Load a Docker image: {\tt t2docker -L t2docker-image.tar[.gz]}
    \item Get a Shell in a Docker image: {\tt t2docker -X image-name-or-id}
    \item Run Tranalyzer inside a Docker container: {\tt t2docker -r file.pcap}
    \item Run another T2 command inside a Docker container: {\tt t2docker tawk -V flowStat}
\end{itemize}

Note that when running T2 inside a Docker container, the pcap file is copied to a temporary folder (automatically removed or shredded ({\tt --{}--shred} option) after the script has ended). This extra step make sure only the required files are accessible by the container.

\subsection{t2dpdk}\label{t2dpdk}
Run $N$ instances of T2 in DPDK multi-process mode.
T2 must be compiled with {\tt DPDK\_MP=1}:

\begin{flushleft}
    {\tt \$ t2conf tranalyzer2 -D DPDK\_MP=1}\\
    {\tt \$ t2build -R -r -f}\\
    {\tt \$ t2dpdk -N 4 -i 0000:04:00.0}
\end{flushleft}

\subsection{t2flowstat}\label{t2flowstat}
Calculates statistical distributions of selected columns/flows from a flow file.

\subsection{t2fm}\label{t2fm}
Generates a PDF report out of:
\begin{itemize}
    \item a flow file ({\tt --F} option): {\tt t2fm -F file\_flows.txt}
    \item a live interface ({\tt --i} option): {\tt t2fm -i eth0}
    \item a PCAP file ({\tt --r} option): {\tt t2fm -r file.pcap}
    \item a list of PCAP files ({\tt --R} option): {\tt t2fm -R pcap\_list.txt}
\end{itemize}

%Refer to the \tranref{t2fm_tutorial} in the {\tt doc/} folder for more information.

\subsubsection{Required Plugins}
    \begin{multicols}{4}
        \begin{itemize}
            \item basicFlow
            \item basicStats
            \item txtSink
        \end{itemize}
    \end{multicols}

\subsubsection{Optional Plugins}
    \begin{multicols}{4}
        \begin{itemize}
            \item arpDecode
            \item dnsDecode
            \item geoip
            \item httpSniffer
            \item nDPI
            \item portClassifier
            \item pwX
            \item sshDecode
        \end{itemize}
    \end{multicols}

\subsection{t2fuzz}\label{t2fuzz}
Randomly corrupt a PCAP file and run T2 against it: {\tt t2fuzz file.pcap}.\\
\noindent Try {\tt t2fuzz --{}--help} for more information.

\subsection{t2locate}\label{t2locate}
Query location database to acquire information about a city near float point terrestrial coordinates.
Build the DB first with the script: {\tt \$T2HOME/scripts/t2locate/update\_db}, duration ca 30 min.

\subsection{t2netID}\label{t2netID}
Decode hexadecimal network IDs.
An ID contain information about the country, the organization and whether or not the address is a Tor address.
An ID can be decoded as follows: {\tt t2netID 0x138020a5}.\\
\noindent Try {\tt t2netID --{}--help} for more information.

\subsection{t2plot}\label{t2plot}
2D/3D plot for Tranalyzer using gnuplot.
First row of the input file must be the column names (may start with a `{\tt \%}').
The input file must contain one, two or more columns separated by tabs ({\tt \textbackslash{}t}).
Columns to plot can be selected with {\tt --o} option.
Try {\tt t2plot --{}--help} for more information.

\paragraph{Dependencies:}
The t2plot script requires {\bf gnuplot}.
\begin{description}
    \item[Arch:] {\tt sudo pacman --S gnuplot}
    \item[Ubuntu:] {\tt sudo apt-get install gnuplot-qt}
    %\item[openSUSE:] {\tt sudo zypper install gnuplot}
    %\item[Red Hat/Fedora:] {\tt sudo yum install gnuplot}
    \item[macOS:] {\tt brew install gnuplot --{}--with-qt}
\end{description}

\paragraph{Examples:}
\begin{itemize}
    \item {\tt tawk `\{ print ip2num(shost()), ip2num(dhost()) \}' f\_flows.txt | t2plot -pt}
    \item {\tt tawk `\{ print ip2num(\$srcIP), \$timeFirst, \$connSip \}' f\_flows.txt | t2plot}
    \item {\tt tawk `\{ print \$pktsSnt \}' f\_flows.txt | t2plot -H 10 -sx "0:40"}
    \item {\tt t2plot -o pktsSnt f\_flows.txt}
    \item {\tt t2plot -D -o srcIPCC:l7BytesSnt f\_flows.txt}
    \item {\tt tawk `\{ print proto(), \$l7BytesSnt \}' f\_flows.txt | t2plot -D}
    \item {\tt t2plot file\_with\_one\_two\_or\_three\_columns.txt}
    \item {\tt t2plot -o "26:28" file\_with\_many\_columns.txt}
    \item {\tt t2plot -o "l7BytesSnt:l7BytesRcvd" file\_with\_many\_columns.txt}
    \item Try {\tt t2plot --e} for more examples.
\end{itemize}

\subsection{t2plugin}\label{t2plugin}
Use this script to create a new plugin or list existing plugins
For a more comprehensive description of how to write a plugin, refer to Appendix A (Creating a custom plugin) of \href{../../doc/documentation.pdf}{\tt \$T2HOME/doc/documentation.pdf}.

\subsection{t2rrd}\label{s:t2rrd}
Store Tranalyzer monitoring output into a RRD database ({\tt --m} option) or use the RRD database generated with {\tt --m} to monitor and plot various values, e.g., number of flows.

\subsection{t2stat}\label{t2stat}
Sends {\tt USR1} signal to Tranalyzer to produce intermediary report.
The signal sent can be changed with the {\tt --SIGNAME} option, e.g., {\tt t2stat --USR2} or {\tt t2stat --INT}.
If Tranalyzer was started as root, the {\tt --s} option can be used to run the command with {\tt sudo}.
The {\tt --p} option can be used to print the PID of running Tranalyzer instances and the {\tt --l} option provides additional information about the running instances (command and running time).
The {\tt --i} option can be used to cycle through all the running instances and will prompt for confirmation before sending the signal to a specific process.
If a numeric argument $N$ is provided, sends the signal every $N$ seconds, e.g., {\tt t2stat 10} to report every 10s.
Use {\tt t2stat --{}--help} for more information.

\subsection{t2timeline}\label{t2timeline}
Timeline plot of flows: {\tt t2timeline FILE\_flows.txt}

\begin{itemize}
    \item To use relative time, i.e., starting at 0, use the {\tt --r} option.
    \item The vertical space between A and B flows can be adapted with the {\tt --v} option, e.g., {\tt -v 50}.
    \item When hovering over a flow, the following information is displayed:\\
          {\tt flowInd\_flowStat\_srcIP:srcPort\_dstIP:dstPort\_l4Proto\_vlanID}.
    \item Additional information can be displayed with the {\tt --e} option, e.g, {\tt -e macS,macD,duration}
    \item Use {\tt t2timeline --{}--help} for more information.
\end{itemize}

\noindent An example graph is depicted in \reff{fig:t2timeline}.

\begin{figure}[!ht]
    \centering
    \tranimg[width=.6\textwidth]{timeline}
    \caption{T2 timeline flow plot}
    \label{fig:t2timeline}
\end{figure}

%\subsection{t2update}\label{t2update}
%Updates Tranalyzer and all the plugins (svn or git): {\tt ./t2update}\\
%The {\tt --d} option can be used to display the local changes and the {\tt --s} option can be used to print the status of the files (use {\tt --m} to ignore unversioned files. Can also be used to update, diff or stat selected plugins as follows: {\tt t2update pluginName(s)}

\subsection{t2topcap}\label{t2topcap}
Convert pcapng files to pcap: {\tt t2topcap file.pcapng}
\noindent Try {\tt t2topcap --{}--help} for more information.

\subsection{t2utils.sh}\label{t2utils.sh}
Collection of bash functions and variables ({\tt readonly}).

\subsubsection{Usage}
To access the functions and variables provided by this file, source it in your script as follows:
\begin{center}{\tt source "\$(dirname "\$\{0\}")/t2utils.sh"}\end{center}
Note that if your script is not in the {\tt scripts/} folder, you will need to adapt the path above to {\tt t2utils.sh} accordingly.

\paragraph{[ZSH]} If writing a script for ZSH, add the following line {\bf BEFORE} sourcing the script:
\begin{center}{\tt unsetopt function\_argzero}\end{center}

\clearpage

\subsubsection{Colors}
Alternatives to {\tt printbold}, {\tt printerr}, {\tt printinf}, {\tt printok} and {\tt printwrn}:
\begin{longtable}{>{\tt}ll>{\tt}l}
    \toprule
    {\bf Variable}              & {\bf Description}                     & {\bf Example}\\
    \midrule\endhead%
    \$\{BLUE\}                  & Set the color to blue                 & printf "\$\{BLUE\}\%s\$\{NOCOLOR\}\textbackslash{}n" "msg"\\
    \$\{GREEN\}                 & Set the color to green                & printf "\$\{GREEN\}\%s\$\{NOCOLOR\}\textbackslash{}n" "msg"\\
    \$\{ORANGE\}                & Set the color to orange               & printf "\$\{ORANGE\}\%s\$\{NOCOLOR\}\textbackslash{}n" "msg"\\
    \$\{RED\}                   & Set the color to red                  & printf "\$\{RED\}\%s\$\{NOCOLOR\}\textbackslash{}n" "msg"\\
                                &                                       & \\
    \$\{BLUE\_BOLD\}            & Set the color to blue bold            & printf "\$\{BLUE\_BOLD\}msg\$\{NOCOLOR\}\textbackslash{}n"\\
    \$\{GREEN\_BOLD\}           & Set the color to green bold           & printf "\$\{GREEN\_BOLD\}msg\$\{NOCOLOR\}\textbackslash{}n"\\
    \$\{ORANGE\_BOLD\}          & Set the color to orange bold          & printf "\$\{ORANGE\_BOLD\}msg\$\{NOCOLOR\}\textbackslash{}n"\\
    \$\{RED\_BOLD\}             & Set the color to red bold             & printf "\$\{RED\_BOLD\}msg\$\{NOCOLOR\}\textbackslash{}n"\\
                                &                                       & \\
    \$\{BLUE\_ITALIC\}          & Set the color to blue italic          & printf "\$\{BLUE\_BOLD\}\$\{msg\}\$\{NOCOLOR\}\textbackslash{}n"\\
    \$\{GREEN\_ITALIC\}         & Set the color to green italic         & printf "\$\{GREEN\_BOLD\}\$\{msg\}\$\{NOCOLOR\}\textbackslash{}n"\\
    \$\{ORANGE\_ITALIC\}        & Set the color to orange italic        & printf "\$\{ORANGE\_BOLD\}\$\{msg\}\$\{NOCOLOR\}\textbackslash{}n"\\
    \$\{RED\_ITALIC\}           & Set the color to red italic           & printf "\$\{RED\_BOLD\}\$\{msg\}\$\{NOCOLOR\}\textbackslash{}n"\\
                                &                                       & \\
    \$\{BLUE\_UNDERLINE\}       & Set the color to blue underline       & echo -e "\$\{BLUE\_ITALIC\}msg\$\{NOCOLOR\}"\\
    \$\{GREEN\_UNDERLINE\}      & Set the color to green underline      & echo -e "\$\{GREEN\_ITALIC\}msg\$\{NOCOLOR\}"\\
    \$\{ORANGE\_UNDERLINE\}     & Set the color to orange underline     & echo -e "\$\{ORANGE\_ITALIC\}msg\$\{NOCOLOR\}"\\
    \$\{RED\_UNDERLINE\}        & Set the color to red underline        & echo -e "\$\{RED\_ITALIC\}msg\$\{NOCOLOR\}"\\
                                &                                       & \\
    \$\{BOLD\}                  & Set the font to bold                  & echo -e "\$\{BOLD\}\$\{msg\}\$\{NOCOLOR\}"\\
    \$\{ITALIC\}                & Set the font to italic                & echo -e "\$\{ITALIC\}\$\{msg\}\$\{NOCOLOR\}"\\
    \$\{UNDERLINE\}             & Set the font to underline             & echo -e "\$\{UNDERLINE\}\$\{msg\}\$\{NOCOLOR\}"\\
    \$\{STRIKETHROUGH\}         & Set the font to strikethrough         & echo -e "\$\{STRIKETHROUGH\}\$\{msg\}\$\{NOCOLOR\}"\\
                                &                                       & \\
    \$\{NOCOLOR\}               & Reset the color                       & echo -e "\$\{BOLD\}\$\{msg\}\$\{NOCOLOR\}"\\
    \bottomrule
\end{longtable}

\subsubsection{Folders}
\begin{longtable}{>{\tt}lll}
    \toprule
    {\bf Variable} & {\bf Description}                               & {\bf Example}\\
    \midrule\endhead%
    \$\{SHOME\}    & Points to the folder where the script resides   & For {\tt macRecorder/utils/mconv}, {\tt \$\{SHOME\}} is\\
                   &                                                 & \qquad {\tt \$\{T2PLHOME\}/macRecorder/utils}\\
    \$\{T2HOME\}   & Points to the root folder of Tranalyzer         & {\tt cd "\$\{T2HOME\}/tranalyzer2"}\\
    \$\{T2PLHOME\} & Points to the root folder of Tranalyzer plugins & {\tt cd "\$\{T2PLHOME\}/t2PSkel"}\\
    \bottomrule
\end{longtable}

\subsubsection{Scripts and Programs}

Functions and variables pointing to programs.

\paragraph{Functions}
\begin{longtable}{>{\tt}ll>{\tt}l}
    \toprule
    {\bf Name} & {\bf Program}                                               & {\bf Example}\\
    \midrule\endhead%
    AWK        & {\tt gawk}                                                  & AWK `\{ print \}' file\\
    AWKF       & {\tt gawk -F`\textbackslash{}t' -v OFS=`\textbackslash{}t'} & AWKF `\{ print \}' file\\
    T2         & Most recent {\tt tranalyzer} executable in                  & \\
               & \qquad {\tt \$\{T2HOME\}/tranalyzer2/}                      & T2 -r file.pcap\\
    T2B2T      & Most recent {\tt t2b2t} executable in                       & \\
               & \qquad {\tt \$\{T2HOME\}/utils/t2b2t/}                      & T2B2T -r file.bin -w file.txt\\
    T2WHOIS    & Most recent {\tt t2whois} executable in                     & \\
               & \qquad {\tt \$\{T2HOME\}/utils/t2whois/}                    & T2WHOIS 1.2.3.4\\
    \bottomrule
\end{longtable}

\clearpage

\paragraph{Variables}

\begin{longtable}{>{\tt}ll>{\tt}l}
    \toprule
    {\bf Name}      & {\bf Program}                                    & {\bf Example}\\
    \midrule\endhead%
    \$\{AWK\_EXEC\} & {\tt gawk}                                       & "\$\{AWK\_EXEC\}" `NR > 0 \{ print \$2 \}' file\\
    \$\{OPEN\}      & {\tt xdg-open} (Linux), {\tt open} (macOS)       & "\$\{OPEN\}" file.pdf\\
    \$\{PYTHON\}    & {\tt python3}, {\tt python} or {\tt python2}     & "\$\{PYTHON\}" file.py\\
    \$\{READLINK\}  & {\tt readlink} (Linux) / {\tt greadlink} (macOS) & "\$\{READLINK\}" file\\
    \$\{SED\}       & {\tt sed} (Linux) / {\tt gsed} (macOS)           & "\$\{SED\}" `s/ /\_/g' <<< "\$str"\\
    \$\{T2BUILD\}   & {\tt "\$\{T2HOME\}/autogen.sh"}                  & "\$\{T2BUILD\}" tranalyzer2\\
    \$\{T2CONF\}    & {\tt "\$\{T2HOME\}/scripts/t2conf/t2conf"}       & "\$\{T2CONF\}" basicFlow -D BFO\_MAC=1\\
    \$\{T2PLOT\}    & {\tt "\$\{T2HOME\}/scripts/t2plot"}              & "\$\{T2PLOT\}" -o srcIPCC file\\
    \$\{TAWK\}      & {\tt "\$\{T2HOME\}/scripts/tawk/tawk"}           & "\$\{TAWK\}" `\{ print tuple4() \} file'\\
    \bottomrule
\end{longtable}

\subsubsection{Functions}
\begin{longtable}{>{\tt}ll}
    \toprule
    {\bf Function}                              & {\bf Description}\\
    \midrule\endhead%
    printbold "msg"                             & print a message (bold) with a newline\\
    printerr  "msg"                             & print an error message (red) with a newline\\
    printinf  "msg"                             & print an info message (blue) with a newline\\
    printok   "msg"                             & print an ok message (green) with a newline\\
    printwrn  "msg"                             & print a warning message (orange) with a newline\\
                                                & \\
    printfbold "msg"                            & print a message (bold) without a newline\\
    printferr  "msg"                            & print an error message (red) without a newline\\
    printfinf  "msg"                            & print an info message (blue) without a newline\\
    printfok   "msg"                            & print an ok message (green) without a newline\\
    printfwrn  "msg"                            & print a warning message (orange) without a newline\\
                                                & \\
    fatal     "msg"                             & print an error message (red) with a newline and exit with status 1\\
                                                & \\
    check\_dependency        "bin" "pkg"        & check whether a dependency exists (Linux/macOS)\\
    check\_dependency\_linux "bin" "pkg"        & check whether a dependency exists (Linux)\\
    check\_dependency\_macos "bin" "pkg"        & check whether a dependency exists (macOS)\\
                                                & \\
    test\_min\_version "ver" "req"              & return true if version number {\tt ver} is greater than or equal to {\tt req}\\
                                                & \\
    has\_define "file" "name"                   & return 0 if the macro {\tt name} exists in {\tt file}, 1 otherwise\\
    get\_define "name" "file"                   & return the value of the macro {\tt name} in {\tt file}\\
    set\_define "name" "value" "file"           & set the value of the macro {\tt name} in {\tt file} to {\tt value}\\
                                                & \\
    replace\_suffix "name" "old" "new"          & replace the {\tt old} suffix in {\tt name} by {\tt new}\\
                                                & \\
    join\_by "sep" "values..."                  & join {\tt values...} with a separator {\tt sep}\\
                                                & \\
    ask\_default\_no  "msg" ["answer"]          & ask the question {\tt msg (y/N)?}, read the answer and assume {\tt no}\\
                                                & if answer is not {\tt [yY]} or {\tt [yY][eE][sS]}. Return 0 if answer is\\
                                                & {\tt yes}. {\tt answer} can be set to {\tt yes} or {\tt no} to force the answer\\
    ask\_default\_yes "msg" ["answer"]          & ask the question {\tt msg (Y/n)?}, read the answer and assume {\tt yes}\\
                                                & if answer is not {\tt [nN]} or {\tt [nN][oO]}. Return 0 if answer is\\
                                                & {\tt yes}. {\tt answer} can be set to {\tt yes} or {\tt no} to force the answer\\
                                                & \\
    find\_most\_recent\_dir "dir" "dirname"     & recursively find the most recent {\tt dirname} in a directory {\tt dir}\\
    find\_most\_recent\_file "dir" "filename"   & recursively find the most recent {\tt filename} in a directory {\tt dir}\\
                                                & \\
    t2\_build\_exec "/path/to/exec" [force]     & ask whether to build the given executable if it does not exist\\
                                                & (set {\tt force} to 1 to rebuild {\tt exec} regardless)\\
                                                & \\
    get\_t2\_exec                               & search for {\tt tranalyzer} executable in {\tt \$\{T2HOME\}/tranalyzer2}\\
                                                & (return an empty string, if it could not be found)\\
    get\_t2b2t\_exec                            & search for \nameref{t2b2t} executable in {\tt \$\{T2HOME\}/utils/t2b2t}\\
                                                & (return an empty string, if it could not be found)\\
    get\_t2whois\_exec                          & search for \nameref{t2whois} executable in {\tt \$\{T2HOME\}/utils/t2whois}\\
                                                & (return an empty string, if it could not be found)\\
                                                & \\
    abort\_if\_t2\_exec\_not\_found             & search for {\tt tranalyzer} executable in {\tt \$\{T2HOME\}/tranalyzer2}.\\
                                                & Abort with status 1 and an error and info message if it could\\
                                                & not be found.\\
    abort\_if\_t2b2t\_exec\_not\_found          & search for \nameref{t2b2t} executable in {\tt \$\{T2HOME\}/utils/t2b2t}.\\
                                                & Abort with status 1 and an error and info message if it could\\
                                                & not be found.\\
    abort\_if\_t2whois\_exec\_not\_found        & search for \nameref{t2whois} executable in {\tt \$\{T2HOME\}/utils/t2whois}.\\
                                                & Abort with status 1 and an error and info message if it could\\
                                                & not be found.\\
                                                & \\
    t2\_wget    "url" ["outfile"]               & download the data at {\tt url}\\
                                                & (set {\tt outfile} (optional) to change the output path)\\
    t2\_wget\_n "url" ["outfile"]               & same as {\tt t2\_wget}, but turn on timestamping\\
                                                & (see {\tt wget -N} option)\\
                                                & \\
    get\_nproc                                  & return the number of processing units available\\
                                                & \\
    validate\_float "float"                     & return 0 if {\tt float} is a valid floating point value, 1 otherwise\\
    validate\_int   "int"                       & return 0 if {\tt int} is a valid integer, 1 otherwise\\
    validate\_num   "num"                       & return 0 if {\tt num} is a valid positive integer, 1 otherwise\\
    validate\_ip    "string"                    & return 0 if {\tt string} is a valid IPv4 address, 1 otherwise\\
    validate\_pcap  "file"                      & return 0 if {\tt file} is a valid PCAP file, 1 otherwise\\
                                                & \\
    validate\_next\_arg "curr" "next"           & check whether {\tt next} exists and is not an option\\
                                                & ({\tt curr} is used for error reporting)\\
    validate\_next\_arg\_exists   "curr" "next" & check whether {\tt next} exists\\
    validate\_next\_dir           "curr" "next" & check whether {\tt next} exists and is a directory\\
    validate\_next\_file          "curr" "next" & check whether {\tt next} exists and is a regular file\\
    validate\_next\_file\_or\_dir "curr" "next" & check whether {\tt next} exists and is a regular file or directory\\
    validate\_next\_pcap          "curr" "next" & check whether {\tt next} exists and is a PCAP file\\
    validate\_next\_num           "curr" "next" & check whether {\tt next} exists and is a positive integer\\
    validate\_next\_int           "curr" "next" & check whether {\tt next} exists and is an integer\\
    validate\_next\_float         "curr" "next" & check whether {\tt next} exists and is a float\\
                                                & \\
    arg\_is\_option "arg"                       & check whether {\tt arg} exists and is an option (starts with {\tt --})\\
                                                & \\
    abort\_missing\_arg "option"                & print an error about a missing argument and\\
                                                & \qquad exit with status 1\\
    abort\_option\_unknown "option"             & print an error about an unknown option and\\
                                                & \qquad exit with status 1\\
    abort\_required\_file                       & print an error about a missing required file and\\
                                                & \qquad exit with status 1\\
    abort\_required\_dir                        & print an error about a missing required directory and\\
                                                & \qquad exit with status 1\\
    abort\_required\_file\_or\_dir              & print an error about a missing required file or directory and\\
                                                & \qquad exit with status 1\\
    abort\_with\_help                           & print a message explaining how to get help and\\
                                                & \qquad exit with status 1\\
    \bottomrule
\end{longtable}

\subsection{t2viz}\label{t2viz}
Generates a graphviz script which can be loaded into xdot or dotty: {\tt t2viz FILE\_flows.txt}.\\
\noindent Accepts T2 flow or packet files with header description.\\
\noindent Try {\tt t2viz --{}--help} for more information.

\subsection{t2voipconv}\label{t2voipconv}
Convert raw audio files extracted with \tranrefpl{voipDetector} to wav: {\tt t2voipconv /tmp/TranVoIP}.\\
\noindent Try {\tt t2voipconv --{}--help} for more information.

\subsection{t2whois}\label{t2whois}
Query Tranalyzer subnet databases to get geolocation information about IP addresses:

\begin{itemize}
    \item Get geo info for one IP address: {\tt t2whois 1.2.3.4}
    \item Get geo info for multiple IPs: {\tt t2whois 1.2.3.4 5.6.7.8}
    \item Get geo info for one or multiple IPs listed in a file (one IP per line): {\tt t2whois -r file.txt}
    \item Get geo info for one or multiple IPs entered to stdin via a prompt: {\tt t2whois} or {\tt t2whois -r -}
    \item Get geo info for one or multiple IPs entered to stdin via a pipe: {\tt cat file.txt | t2whois -q -l}
    \item Format the output (one line per IP, no header): {\tt t2whois -l -H -r file.txt}
    \item Format the output (one line per IP, comma separated): {\tt t2whois -l -s "," -r file.txt}
    \item Query information about the databases {\tt t2whois -V}
    \item Try {\tt t2whois -h} for more information
\end{itemize}

\subsection{topNStat}
Generates sorted lists of all the columns (names or numbers) provided.
A list of examples can be displayed using the {\tt --e} option.

\IfFileExists{../t2fm/doc/t2fm.tex}
    {\subfile{../t2fm/doc/t2fm.tex}}
    {\IfFileExists{../scripts/t2fm/doc/t2fm.tex}
        {\subfile{../scripts/t2fm/doc/t2fm.tex}}
        {\error{File 't2fm.tex' not found}}}

\IfFileExists{../tawk/doc/tawk.tex}
    {\subfile{../tawk/doc/tawk.tex}}
    {\IfFileExists{../scripts/tawk/doc/tawk.tex}
        {\subfile{../scripts/tawk/doc/tawk.tex}}
        {\error{File 'tawk.tex' not found}}}

\end{document}
