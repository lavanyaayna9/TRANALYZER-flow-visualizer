\IfFileExists{t2doc.cls}{
    \documentclass[documentation]{subfiles}
}{
    \errmessage{Error: could not find 't2doc.cls'}
}

\begin{document}

\trantitle
    {nFrstPkts}
    {Statistics Over the N First Packets}
    {Tranalyzer Development Team} % author(s)

\section{nFrstPkts}\label{s:nFrstPkts}

\subsection{Description}
The nFrstPkts plugin supplies the Packet Length (PL) and Inter-Arrival Times (IAT) of the $N$ first packets per flow as a column. The default value for $N$ is 20.
It complements the packet mode ({\tt -s} option) with flow based view for the $N$ first packets signal. The plugin supplies several configuration options of how the
resulting packet length signal should be represented. Using the {\tt fpsGplt} script files are generated readily post processable by any command line tool (AWK, Perl),
Excel or Data mining suit, such as SPSS. As outlined in the configuration below, Signals can be produced with IAT, or relative/absolute time. Also the
aggregation of bursts into a single pulse can be configured via {\tt NFRST\_MINIAT}. {\tt NFRST\_MINPLAVE} controls the meaning of the PL value in pulse aggregation
mode. If 0 it corresponds to the BPP measure currently used in research for categorizing media content.

\subsection{Configuration Flags}
The following flags can be used to control the output of the plugin:
\begin{longtable}{>{\tt}lcl>{\tt\small}l}
    \toprule
    {\bf Name} & {\bf Default} & {\bf Description} & {\bf Flags}\\
    \midrule\endhead%
    NFRST\_IAT        &  1 & 0: Time relative to flow start                        & \\
                      &    & 1: Inter-arrival time                                 & \\
                      &    & 2: Absolute time                                      & \\
    NFRST\_BCORR      &  0 & 0: A,B start at 0.0                                   & \\
                      &    & 1: B shift by flow start                              & NFRST\_MINIATS=0\\
    NFRST\_PKTCNT     & 20 & Number of packets to record                           & \\
    NFRST\_HDRINFO    &  0 & add L3,L4 header length                               & \\
    NFRST\_MINIATS    &  0 & 0: Standard IAT sequence                              & \\
                      &    & > 0: minimal packet IAT us/ns defining a pulse signal & \\
    NFRST\_MINIATU    &  0 & 0: Standard IAT sequence                              & \\
                      &    & > 0: minimal packet IAT us/ns defining a pulse signal & \\
    NFRST\_MINPLENFRC &  2 & Minimal pulse length fraction                         & \\
    NFRST\_PLAVE      &  1 & 0: Sum PL (BPP)                                       & NFRST\_MINIATS>0||\\
                      &    & 1: Average PL                                         & NFRST\_MINIATU>0\\
    NFRST\_XMIN       &  0 & Min PL boundary                                       &\\
    NFRST\_XMAX       & {\tt\small UINT16\_MAX}
                           & Max PL boundary                                       & \\
    \bottomrule
\end{longtable}

For the rest of this document, {\tt NFRST\_MINIAT} is used to represent {\tt (NFRST\_MINIATS>0||NFRST\_MINIATU>0)}.

\subsection{Flow File Output}
The nFrstPkts plugin outputs the following columns:
\begin{longtable}{>{\tt}lll>{\tt\small}l}
    \toprule
    {\bf Column} & {\bf Type} & {\bf Description} & {\bf Flags}\\
    \midrule\endhead%
    nFpCnt                   & U32               & Number of signal samples\\
    L2L3L4Pl\_Iat            & R(U16\_UT)        & L2/L3/L4 or payload length and inter-arrival       & NFRST\_HDRINFO=0\&\&\\
                             &                   & \qquad times for the N first packets               & NFRST\_MINIAT=0\\
    L2L3L4Pl\_Iat\_nP        & R(U16\_UT\_UT)    & L2/L3/L4 or payload length, inter-arrival times    & NFRST\_HDRINFO=0\&\&\\
                             &                   & \qquad and pulse length for the N first packets    & NFRST\_MINIAT>0\\
    HD3l\_HD4l\_             & R(U8\_U8\_        & L3Hdr, L4Hdr, L2/L3/L4 or payload length and       & NFRST\_HDRINFO=1\&\&\\
    \qquad L2L3L4Pl\_Iat     & \qquad \_U16\_UT) & \qquad inter-arrival times for the N first packets & NFRST\_MINIAT=0\\
    HD3l\_HD4l\_             & R(U8\_U8\_U16\_   & L3Hdr, L4Hdr, L2/L3/L4 or payload length and       & NFRST\_HDRINFO=1\&\&\\
    \qquad L2L3L4Pl\_Iat\_nP & \qquad UT\_UT)    & \qquad inter-arrival times for the N first packets & NFRST\_MINIAT>0\\
    \bottomrule
\end{longtable}

\subsection{Post-Processing}
The {\tt fpsGplt} script can be used to transform the packet signal from nFrstPkts to gnuplot or \tranref{t2plot} format. It produces several signal variants which can also be used for signal processing and AI applications. For more details, refer to the traffic mining tutorial at \url{https://tranalyzer.com/tutorial/trafficmining}.\\

\begin{verbatim}
>fpsGplt -h
Usage:
    fpsGplt [OPTION...] <FILE>

Optional arguments:

    -f               Flow index to extract, default: all flows
    -d               Flow Direction: 0, 1; default both
    -t               noTime: counts on x axis; default time on x axis
    -i               invert B Flow PL
    -s               time sorted

    -h, --help       Show this help, then exit
\end{verbatim}

\end{document}
