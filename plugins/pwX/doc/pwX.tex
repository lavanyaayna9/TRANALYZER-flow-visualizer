\IfFileExists{t2doc.cls}{
    \documentclass[documentation]{subfiles}
}{
    \errmessage{Error: could not find 't2doc.cls'}
}

\begin{document}

\trantitle
    {pwX}
    {Clear-text Passwords Extractor}
    {Tranalyzer Development Team} % author(s)

\section{pwX}\label{s:pwX}

\subsection{Description}
The pwX plugin extracts usernames and passwords from different plaintext protocols.
This plugin produces only output to the flow file.
Configuration is achieved by user defined compiler switches in {\tt src/pwX.h}.

\subsection{Configuration Flags}
The following flags can be used to control the output of the plugin:
\begin{longtable}{>{\tt}lcl}
    \toprule
    {\bf Variable} & {\bf Default} & {\bf Description} \\
    \midrule\endhead%
    PWX\_USERNAME    & 1 & Output the username\\
    PWX\_PASSWORD    & 1 & Output the password\\
    \\
    PWX\_FTP         & 1 & Extract FTP authentication\\
    PWX\_POP3        & 1 & Extract POP3 authentication\\
    PWX\_IMAP        & 1 & Extract IMAP authentication\\
    PWX\_SMTP        & 1 & Extract SMTP authentication\\
    PWX\_HTTP\_BASIC & 1 & Extract HTTP Basic Authorization\\
    PWX\_HTTP\_PROXY & 1 & Extract HTTP Proxy Authorization\\
    PWX\_HTTP\_GET   & 1 & Extract HTTP GET authentication\\
    PWX\_HTTP\_POST  & 1 & Extract HTTP POST authentication\\
    PWX\_IRC         & 1 & Extract IRC authentication\\
    PWX\_TELNET      & 1 & Extract Telnet authentication\\
    PWX\_LDAP        & 1 & Extract LDAP bind request authentication\\\\
    PWX\_PAP         & 1 & Extract PAP (Password Authentication Protocol) authentication\\
    \\
    PWX\_STATUS      & 1 & Extract authentication status (success, error, \ldots).\\
    \\
    PWX\_DEBUG       & 0 & Activate debug output.\\
    \bottomrule
\end{longtable}

%\subsection{Required files}
%none

\subsection{Flow File Output}
The pwX plugin outputs the following columns:
\begin{longtable}{>{\tt}lll>{\tt\small}l}
    \toprule
    {\bf Name} & {\bf Type} & {\bf Description} & {\bf Flags}\\
    \midrule\endhead%
    \nameref{pwxType}   & U8 & Authentication type   & \\
    pwxUser             & S  & Extracted username    & PWX\_USERNAME!=0\\
    pwxPass             & S  & Extracted password    & PWX\_PASSWORD!=0\\
    \nameref{pwxStatus} & U8 & Authentication status & PWX\_STATUS!=0\\
    \bottomrule
\end{longtable}

\subsubsection{pwxType}\label{pwxType}
The {\tt pwxType} column is to be interpreted as follows:
\begin{longtable}{rl}
    \toprule
    {\bf pwxType} & {\bf Description}\\
    \midrule\endhead%
     0 & No password or username extracted\\
     1 & FTP authentication\\
     2 & POP3 authentication\\
     3 & IMAP authentication\\
     4 & SMTP authentication\\
     5 & HTTP Basic Authorization\\
     6 & HTTP Proxy Authorization\\
     7 & HTTP GET authentication\\
     8 & HTTP POST authentication \\
     9 & IRC authentication \\
    10 & Telnet authentication \\
    11 & LDAP authentication \\
    12 & PAP authentication \\
    \bottomrule
\end{longtable}

\subsubsection{pwxStatus}\label{pwxStatus}
The {\tt pwxStatus} column is to be interpreted as follows:
\begin{longtable}{rl}
    \toprule
    {\bf pwxStatus} & {\bf Description}\\
    \midrule\endhead%
    0 & Authentication status is unknown\\
    1 & Authentication was successful\\
    2 & Authentication failed\\
    \bottomrule
\end{longtable}

%\subsection{Additional Output}
%none

\subsection{Plugin Report Output}
The number of passwords extracted is reported.

\end{document}
