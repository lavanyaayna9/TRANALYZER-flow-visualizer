\IfFileExists{t2doc.cls}{
    \documentclass[documentation]{subfiles}
}{
    \errmessage{Error: could not find 't2doc.cls'}
}

\begin{document}

\trantitle
    {descriptiveStats}
    {Descriptive Statistics}
    {Tranalyzer Development Team} % author(s)

\section{descriptiveStats}\label{s:descriptiveStats}

\subsection{Description}
The descriptiveStats plugin calculates various statistics about a flow.
Because the inter-arrival time of the first packet is per definition always zero, it is removed from the statistics.
Therefore the inter-arrival time statistics values for flows with only one packet is set to zero.

\subsection{Dependencies}

%\traninput{file} % use this command to input files
%\traninclude{file} % use this command to include files

%\tranimg{image} % use this command to include an image (must be located in a subfolder ./img/)

%\subsubsection{External Libraries}
%This plugin depends on ...

\subsubsection{Other Plugins}
This plugin requires the \tranrefpl{pktSIATHisto} plugin.

%\subsubsection{Required Files}
%The file ... is required.

\subsection{Configuration Flags}
The following flags can be used to control the output of the plugin:
\begin{longtable}{>{\tt}lcl>{\tt\small}l}
    \toprule
    {\bf Name} & {\bf Default} & {\bf Description} & {\bf Flags}\\
    \midrule\endhead%
    DS\_PS\_CALC  & 1 & Compute statistics for packet sizes        & \\
    DS\_IAT\_CALC & 1 & Compute statistics for inter-arrival times & \\
    DS\_QUARTILES & 0 & Quartiles calculation:                     & DS\_PS\_CALC=1\\
                  &   & \qquad 0: Use linear interpolation         & \\
                  &   & \qquad 1: Use the mean                     & \\
    \bottomrule
\end{longtable}

\subsection{Flow File Output}
The descriptiveStats plugin outputs the following columns:
\begin{longtable}{>{\tt}lll>{\tt\small}l}
    \toprule
    {\bf Column} & {\bf Type} & {\bf Description} & {\bf Flags}\\
    \midrule\endhead%
    %\\
    %\multicolumn{4}{l}{If {\tt ENABLE\_PS\_CALC=1}, the following columns are displayed:}\\
    %\\
    dsMinPl          & F & Minimum packet length                            & DS\_PS\_CALC=1\\
    dsMaxPl          & F & Maximum packet length                            & DS\_PS\_CALC=1\\
    dsMeanPl         & F & Mean packet length                               & DS\_PS\_CALC=1\\
    dsLowQuartilePl  & F & Lower quartile of packet lengths                 & DS\_PS\_CALC=1\\
    dsMedianPl       & F & Median of packet lengths                         & DS\_PS\_CALC=1\\
    dsUppQuartilePl  & F & Upper quartile of packet lengths                 & DS\_PS\_CALC=1\\
    dsIqdPl          & F & Inter quartile distance of packet lengths        & DS\_PS\_CALC=1\\
    dsModePl         & F & Mode of packet lengths                           & DS\_PS\_CALC=1\\
    dsRangePl        & F & Range of packet lengths                          & DS\_PS\_CALC=1\\
    dsStdPl          & F & Standard deviation of packet lengths             & DS\_PS\_CALC=1\\
    dsRobStdPl       & F & Robust standard deviation of packet lengths      & DS\_PS\_CALC=1\\
    dsSkewPl         & F & Skewness of packet lengths                       & DS\_PS\_CALC=1\\
    dsExcPl          & F & Excess of packet lengths                         & DS\_PS\_CALC=1\\
    \\
    %\multicolumn{4}{l}{If {\tt ENABLE\_IAT\_CALC=1}, the following columns are displayed:}\\
    %\\
    dsMinIat         & F & Minimum inter-arrival time                       & DS\_IAT\_CALC=1\\
    dsMaxIat         & F & Maximum inter-arrival time                       & DS\_IAT\_CALC=1\\
    dsMeanIat        & F & Mean inter-arrival time                          & DS\_IAT\_CALC=1\\
    dsLowQuartileIat & F & Lower quartile of inter-arrival times            & DS\_IAT\_CALC=1\\
    dsMedianIat      & F & Median of inter-arrival times                    & DS\_IAT\_CALC=1\\
    dsUppQuartileIat & F & Upper quartile of inter-arrival times            & DS\_IAT\_CALC=1\\
    dsIqdIat         & F & Inter quartile distance of inter-arrival times   & DS\_IAT\_CALC=1\\
    dsModeIat        & F & Mode of inter-arrival times                      & DS\_IAT\_CALC=1\\
    dsRangeIat       & F & Range of inter-arrival times                     & DS\_IAT\_CALC=1\\
    dsStdIat         & F & Standard deviation of inter-arrival times        & DS\_IAT\_CALC=1\\
    dsRobStdIat      & F & Robust standard deviation of inter-arrival times & DS\_IAT\_CALC=1\\
    dsSkewIat        & F & Skewness of inter-arrival times                  & DS\_IAT\_CALC=1\\
    dsExcIat         & F & Excess of inter-arrival times                    & DS\_IAT\_CALC=1\\
    \bottomrule
\end{longtable}

\subsection{Known Bugs and Limitations}
Because the \tranrefpl{pktSIATHisto} plugin stores the inter-arrival times in statistical bins, the original time information is lost.
Therefore, the calculation of the inter-arrival times statistics is, due to its logarithmic binning, only a rough approximation of the original timing information.
Nevertheless, this representation has shown to be useful in practical cases of anomaly and application classification.

\end{document}
