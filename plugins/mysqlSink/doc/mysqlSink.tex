\IfFileExists{t2doc.cls}{
    \documentclass[documentation]{subfiles}
}{
    \errmessage{Error: could not find 't2doc.cls'}
}

\begin{document}

\trantitle
    {mysqlSink} % Plugin name
    {MariaDB / MySQL} % Short description
    {Tranalyzer Development Team} % author(s)

\section{mysqlSink}\label{s:mysqlSink}

\subsection{Description}
The mysqlSink plugin outputs flows to a MariaDB / MySQL database.

\subsection{Dependencies}

\subsubsection{External Libraries}
This plugin depends on the {\bf MariaDB} or {\bf MySQL} library.
\begin{table}[!ht]
    \centering
    \begin{tabular}{>{\bf}r>{\tt}l>{\tt}l>{\tt}l}
        \toprule
                                     &                      & {\bf MariaDB}             & {\bf MySQL}\\
        \midrule
        Ubuntu:                      & sudo apt-get install & libmariadb-dev            & libmysqlclient-dev\\
        Arch:                        & sudo pacman -S       & mariadb-libs              & \\
        Gentoo:                      & sudo emerge          & mariadb-connector-c       & mysql-connector-c\\
        openSUSE:                    & sudo zypper install  & libmariadb-devel          & \\
        Red Hat/Fedora\tablefootnote{If the {\tt dnf} command could not be found, try with {\tt yum} instead}:
                                     & sudo dnf install     & mariadb-connector-c-devel & community-mysql-devel\\
        macOS\tablefootnote{Brew is a packet manager for macOS that can be found here: \url{https://brew.sh}}:
                                     & brew install         & mariadb-connector-c       & \\
        \bottomrule
    \end{tabular}
\end{table}

\subsubsection{Core Configuration}
This plugin requires the following core configuration:
\begin{itemize}
    \item {\em \$T2HOME/tranalyzer2/src/tranalyzer.h}:
        \begin{itemize}
            \item {\tt BLOCK\_BUF=0}
        \end{itemize}
\end{itemize}

\subsection{Database Setup}

\subsubsection{Create a User with Create and Write Permissions}
\begin{verbatim}
$ sudo mysql -u root mysql
...
MariaDB [mysql]> create user `mysql'@`localhost' identified by `mysql';
MariaDB [mysql]> grant all privileges on *.* to `mysql'@'`localhost' with grant option;
\end{verbatim}

\subsection{Configuration Flags}
The following flags can be used to control the output of the plugin:
\begin{longtable}{>{\tt}lcl>{\tt\small}l}
    \toprule
    {\bf Name} & {\bf Default} & {\bf Description}\\
    \midrule\endhead%
    MYSQL\_OVERWRITE\_DB       & 2                        & 0: abort if DB already exists\\
                               &                          & 1: overwrite DB if it already exists\\
                               &                          & 2: reuse DB if it already exists\\
    MYSQL\_OVERWRITE\_TABLE    & 2                        & 0: abort if table already exists\\
                               &                          & 1: overwrite table if it already exists\\
                               &                          & 2: append to table if it already exists\\
    MYSQL\_TRANSACTION\_NFLOWS & 40000                    & 0: one transaction\\
                               &                          & > 0: one transaction every $n$ flows\\
    MYSQL\_QRY\_LEN            & 32768                    & Max length for query\\
    MYSQL\_HOST                & {\tt\small "127.0.0.1"}  & Address of the database\\
    MYSQL\_DBPORT              & 3306                     & Port the DB is listening to\\
    MYSQL\_USER                & {\tt\small "mysql"}      & Username to connect to DB\\
    MYSQL\_PASS                & {\tt\small "mysql"}      & Password to connect to DB\\
    MYSQL\_DBNAME              & {\tt\small "tranalyzer"} & Name of the database\\
    MYSQL\_TABLE\_NAME         & {\tt\small "flow"}       & Name of the table\\
    \\
    \hyperref[mysql:select]{MYSQL\_SELECT}
                               & 0                        & Only insert specific fields into the DB\\
    \hyperref[mysql:select]{MYSQL\_SELECT\_FILE}
                               & {\small\tt "mysql-columns.txt"}
                                                          & Filename of the field selector (one column name per line)\\
    \bottomrule
\end{longtable}

\subsubsection{Environment Variable Configuration Flags}
The following configuration flags can also be configured with environment variables ({\tt ENVCNTRL>0}):
\begin{itemize}
    \item {\tt MYSQL\_HOST}
    \item {\tt MYSQL\_DBPORT}
    \item {\tt MYSQL\_USER}
    \item {\tt MYSQL\_PASS}
    \item {\tt MYSQL\_DBNAME}
    \item {\tt MYSQL\_TABLE\_NAME}
    \item {\tt MYSQL\_SELECT\_FILE} (require {\tt MYSQL\_SELECT=1})
\end{itemize}

\subsection{Insertion of Selected Fields Only}\label{mysql:select}

When {\tt MYSQL\_SELECT=1}, the columns to insert into the DB can be customized with the help of {\tt MYSQL\_SELECT\_FILE}.
The filename defaults to {\tt mysql-columns.txt} in the user plugin folder, e.g., {\em \textasciitilde{}/.tranalyzer/plugins}.
The format of the file is simply one field name per line with lines starting with a {\tt `\#'} being ignored.
For example, to only insert source and destination addresses and ports, create the following file:

\begin{verbatim}
# Lines starting with a '#' are ignored and can be used to add comments
srcIP
srcPort
dstIP
dstPort
\end{verbatim}

\subsection{Example}
{\tt\color{blue} \# Run Tranalyzer}\\
{\tt \$ t2 -r file.pcap}\\
{\tt\color{blue} \# Connect to the MySQL database}\\
{\tt \$ mysql tranalyzer}\\
{\tt\color{blue} \# Number of flows}\\
{\tt MariaDB [tranalyzer]> select count(*) from flow;}\\
{\tt\color{blue} \# 10 first srcIP/dstIP pairs}\\
{\tt MariaDB [tranalyzer]> select "srcIP", "dstIP" from flow limit 10;}\\
{\tt\color{blue} \# All flows from 1.2.3.4 to 1.2.3.5}\\
{\tt MariaDB [tranalyzer]> select * from flow where "srcIP" = '1.2.3.4' and "dstIP" = '1.2.3.5';}

\end{document}
