\IfFileExists{t2doc.cls}{
    \documentclass[documentation]{subfiles}
}{
    \errmessage{Error: could not find 't2doc.cls'}
}

\begin{document}

\trantitle
    {radiusDecode}
    {Remote Authentication Dial-In User Service (RADIUS)}
    {Tranalyzer Development Team} % author(s)

\section{radiusDecode}\label{s:radiusDecode}

\subsection{Description}
The radiusDecode plugin analyzes RADIUS traffic.

\subsection{Configuration Flags}
The following flags can be used to control the output of the plugin:
\begin{longtable}{>{\tt}lcl}
    \toprule
    {\bf Name} & {\bf Default} & {\bf Description} \\
    \midrule\endhead%
    RADIUS\_CNTS    & 1   & Output counts, necessary for {\tt FORCE\_MODE}\\
    RADIUS\_AVPTYPE & 1   & Output AVP Types\\
    RADIUS\_NAS     & 1   & Output NAS info\\
    RADIUS\_FRAMED  & 1   & Output framed info\\
    RADIUS\_TUNNEL  & 1   & Output tunnel info\\
    RADIUS\_ACCT    & 1   & Output accounting info\\
    RADIUS\_NMS     & 0   & Codes and AVP types format:\\
                    &     & \qquad 0: No code/type output\\
                    &     & \qquad 1: Values,\\
                    &     & \qquad 2: Names\\
    RAD\_CNTMX      &  20 & Maximum number of codes/AVP types\\
    RADIUS\_STRMAX  & 128 & Maximum length for strings\\
    \bottomrule
\end{longtable}

\subsection{Flow File Output}
The radiusDecode plugin outputs the following columns:
\begin{longtable}{>{\tt}lll>{\tt\small}l}
    \toprule
    {\bf Column} & {\bf Type} & {\bf Description} & {\bf Flags}\\
    \midrule\endhead%
    \nameref{radiusStat}         & H8       & Status                                 & \\
    radiusAxsReq\_Acc\_Rej\_Chal & 4xU16    & Access-Request/Accept/Reject/Challenge & RADIUS\_CNTS=1\\
    radiusAccReq\_Resp           & U16\_U16 & Accounting-Request/Response            & RADIUS\_CNTS=1\\
    radiusAccStart\_Stop         & U16\_U16 & Accounting Start/Stop                  & RADIUS\_CNTS=1\\
    radiusCodes                  & R(U8)    & Radius codes                           & RADIUS\_NMS=1\\
    radiusCodeNms                & R(S)     & Radius code names                      & RADIUS\_NMS=2\\
    radiusAVPTypes               & R(U8)    & AVP types                              & RADIUS\_AVPTYPE=1\&\&\\
                                 &          &                                        & RADIUS\_NMS=1\\
    radiusAVPTypeNms             & R(S)     & AVP type names                         & RADIUS\_AVPTYPE=1\&\&\\
                                 &          &                                        & RADIUS\_NMS=2\\
    radiusUser                   & S        & Username                               & \\
    radiusPW                     & S        & Password                               & \\
    \nameref{radiusServiceType}  & U32      & Service type                           & \\
    \nameref{radiusLoginService} & U32      & Login-Service                          & \\
    \nameref{radiusVendor}       & U32      & Vendor Id (SMI)                        & \\

    \\
    \multicolumn{4}{l}{If {\tt RADIUS\_NAS=1}, the following columns are displayed:}\\
    \\

    radiusNasId                  & S        & NAS Identifier                         & \\
    radiusNasIp                  & IP4      & NAS IP address                         & \\
    radiusNasPort                & U32      & NAS IP port                            & \\
    \nameref{radiusNasPortType}  & U32      & NAS port type                          & \\
    radiusNasPortId              & S        & NAS port Id                            & \\

    \\
    \multicolumn{4}{l}{If {\tt RADIUS\_FRAMED=1}, the following columns are displayed:}\\
    \\

    radiusFramedIp               & IP4      & Framed IP address                      & \\
    radiusFramedMask             & IP4      & Framed IP netmask                      & \\
    \nameref{radiusFramedProto}  & U32      & Framed protocol                        & \\
    \nameref{radiusFramedComp}   & U32      & Framed compression                     & \\
    radiusFramedMtu              & U32      & Framed MTU                             & \\

    \\
    \multicolumn{4}{l}{If {\tt RADIUS\_TUNNEL=1}, the following columns are displayed:}\\
    \\

    \nameref{radiusTunnelMedium} & U32\_U32 & Tunnel type and medium type            & \\
    radiusTunnelCli              & S        & Tunnel client endpoint                 & \\
    radiusTunnelSrv              & S        & Tunnel server endpoint                 & \\
    radiusTunnelCliAId           & S        & Tunnel client authentication Id        & \\
    radiusTunnelSrvAId           & S        & Tunnel server authentication Id        & \\
    radiusTunnelPref             & S        & Tunnel preference                      & \\

    \\
    \multicolumn{4}{l}{If {\tt RADIUS\_ACCT=1}, the following columns are displayed:}\\
    \\

    radiusAcctSessId             & S        & Accounting session Id                  & \\
    radiusAcctSessTime           & U32      & Accounting session time (seconds)      & \\
    \nameref{radiusAcctStatType} & U32      & Accounting status type                 & \\
    \nameref{radiusAcctTerm}     & U32      & Accounting terminate cause             & \\
    radiusAcctInOct\_OutOct      & U32\_U32 & Accounting input/output octets         & \\
    radiusAcctInPkt\_OutPkt      & U32\_U32 & Accounting input/output packets        & \\
    radiusAcctInGw\_OutGw        & U32\_U32 & Accounting input/output gigawords      & \\
    \\
    radiusConnInfo               & S        & User connection info                   & \\
    radiusFilterId               & S        & Filter Identifier                      & \\
    radiusCalledId               & S        & Called Station Identifier              & \\
    radiusCallingId              & S        & Calling Station Identifier             & \\
    radiusReplyMsg               & S        & Reply message                          & \\
    \bottomrule
\end{longtable}

\subsubsection{radiusStat}\label{radiusStat}
The {\tt radiusStat} column is to be interpreted as follows:
\begin{longtable}{>{\tt}rl}
    \toprule
    {\bf radiusStat} & {\bf Description}\\
    \midrule\endhead%
    $2^0$ (=0x01) & Flow is RADIUS\\
    $2^1$ (=0x02) & Authentication and configuration traffic\\
    $2^2$ (=0x04) & Accounting traffic\\
    $2^3$ (=0x08) & ---\\
    \\
    $2^4$ (=0x10) & Connection successful\\
    $2^5$ (=0x20) & Connection failed\\
    $2^6$ (=0x40) & ---\\
    $2^7$ (=0x80) & Malformed packet\\
    \bottomrule
\end{longtable}

\subsubsection{radiusServiceType}\label{radiusServiceType}
The {\tt radiusServiceType} column is to be interpreted as follows:
\begin{longtable}{rl}
    \toprule
    {\bf radiusServiceType} & {\bf Description}\\
    \midrule\endhead%
     1 & Login\\
     2 & Framed\\
     3 & Callback Login\\
     4 & Callback Framed\\
     5 & Outbound\\
     6 & Administrative\\
     7 & NAS Prompt\\
     8 & Authenticate Only\\
     9 & Callback NAS Prompt\\
    10 & Call Check\\
    11 & Callback Administrative\\
    12 & Voice\\
    13 & Fax\\
    14 & Modem Relay\\
    15 & IAPP-Register\\
    16 & IAPP-AP-Check\\
    17 & Authorize Only\\
    18 & Framed-Management\\
    19 & Additional-Authorization\\
    \bottomrule
\end{longtable}

\subsubsection{radiusLoginService}\label{radiusLoginService}
The {\tt radiusLoginService} column is to be interpreted as follows:
\begin{longtable}{cl}
    \toprule
    {\bf radiusLoginService} & {\bf Description}\\
    \midrule\endhead%
    0 & Telnet\\
    1 & Rlogin\\
    2 & TCP Clear\\
    3 & PortMaster (proprietary)\\
    4 & LAT\\
    5 & X25-PAD\\
    6 & X25-T3POS\\
    7 & Unassigned\\
    8 & TCP Clear Quiet (suppresses any NAS-generated connect string)\\
    \bottomrule
\end{longtable}

\subsubsection{radiusVendor}\label{radiusVendor}
The {\tt radiusVendor} column represents the SMI Network Management Private Enterprise Codes which can be found at \url{https://www.iana.org/assignments/enterprise-numbers}.
Alternatively use {\tt grep} on the file {\tt vendor.txt} as follows: {\tt grep id vendor.txt}, where {\tt id} is the actual Id reported by Tranalyzer, e.g., 4874 for Juniper.

\subsubsection{radiusNasPortType}\label{radiusNasPortType}
The {\tt radiusNasPortType} column is to be interpreted as follows:
\begin{longtable}{rl}
    \toprule
    {\bf radiusNasPortType} & {\bf Description}\\
    \midrule\endhead%
     0 & Async\\
     1 & Sync\\
     2 & ISDN Sync\\
     3 & ISDN Async V.120\\
     4 & ISDN Async V.110\\
     5 & Virtual\\
     6 & PIAFS\\
     7 & HDLC Clear Channel\\
     8 & X.25\\
     9 & X.75\\
    10 & G.3 Fax\\
    11 & SDSL - Symmetric DSL\\
    12 & ADSL-CAP - Asymmetric DSL, Carrierless Amplitude Phase Modulation\\
    13 & ADSL-DMT - Asymmetric DSL, Discrete Multi-Tone\\
    14 & IDSL - ISDN Digital Subscriber Line\\
    15 & Ethernet\\
    16 & xDSL - Digital Subscriber Line of unknown type\\
    17 & Cable\\
    18 & Wireless - Other\\
    19 & Wireless - IEEE 802.11\\
    20 & Token-Ring\\
    21 & FDDI\\
    22 & Wireless - CDMA2000\\
    23 & Wireless - UMTS\\
    24 & Wireless - 1X-EV\\
    25 & IAPP\\
    26 & FTTP - Fiber to the Premises\\
    27 & Wireless - IEEE 802.16\\
    28 & Wireless - IEEE 802.20\\
    29 & Wireless - IEEE 802.22\\
    30 & PPPoA - PPP over ATM\\
    31 & PPPoEoA - PPP over Ethernet over ATM\\
    32 & PPPoEoE - PPP over Ethernet over Ethernet\\
    33 & PPPoEoVLAN - PPP over Ethernet over VLAN\\
    34 & PPPoEoQinQ - PPP over Ethernet over IEEE 802.1QinQ\\
    35 & xPON - Passive Optical Network\\
    36 & Wireless - XGP\\
    37 & WiMAX Pre-Release 8 IWK Function\\
    38 & WIMAX-WIFI-IWK: WiMAX WIFI Interworking\\
    39 & WIMAX-SFF: Signaling Forwarding Function for LTE/3GPP2\\
    40 & WIMAX-HA-LMA: WiMAX HA and or LMA function\\
    41 & WIMAX-DHCP: WiMAX DCHP service\\
    42 & WIMAX-LBS: WiMAX location based service\\
    43 & WIMAX-WVS: WiMAX voice service\\
    \bottomrule
\end{longtable}

\subsubsection{radiusFramedProto}\label{radiusFramedProto}
The {\tt radiusFramedProto} column is to be interpreted as follows:
\begin{longtable}{cl}
    \toprule
    {\bf radiusFramedProto} & {\bf Description}\\
    \midrule\endhead%
    1 & PPP\\
    2 & SLIP\\
    3 & AppleTalk Remote Access Protocol (ARAP)\\
    4 & Gandalf proprietary SingleLink/MultiLink protocol\\
    5 & Xylogics proprietary IPX/SLIP\\
    6 & X.75 Synchronous\\
    7 & GPRS PDP Context\\
    \bottomrule
\end{longtable}

\subsubsection{radiusFramedComp}\label{radiusFramedComp}
The {\tt radiusFramedComp} column is to be interpreted as follows:
\begin{longtable}{cl}
    \toprule
    {\bf radiusFramedComp} & {\bf Description}\\
    \midrule\endhead%
    0 & None\\
    1 & VJ TCP/IP header compression\\
    2 & IPX header compression\\
    3 & Stac-LZS compression\\
    \bottomrule
\end{longtable}

\subsubsection{radiusTunnel\_Medium}\label{radiusTunnelMedium}
The {\tt radiusTunnel\_Medium} column is to be interpreted as follows:
\begin{longtable}{rl}
    \toprule
    {\bf radiusTunnel} & {\bf Description}\\
    \midrule\endhead%
     1 & Point-to-Point Tunneling Protocol (PPTP)\\
     2 & Layer Two Forwarding (L2F)\\
     3 & Layer Two Tunneling Protocol (L2TP)\\
     4 & Ascend Tunnel Management Protocol (ATMP)\\
     5 & Virtual Tunneling Protocol (VTP)\\
     6 & IP Authentication Header in the Tunnel-mode (AH)\\
     7 & IP-in-IP Encapsulation (IP-IP)\\
     8 & Minimal IP-in-IP Encapsulation (MIN-IP-IP)\\
     9 & IP Encapsulating Security Payload in the Tunnel-mode (ESP)\\
    10 & Generic Route Encapsulation (GRE)\\
    11 & Bay Dial Virtual Services (DVS)\\
    12 & IP-in-IP Tunneling\\
    13 & Virtual LANs (VLAN)\\
    \bottomrule
\end{longtable}

\begin{longtable}{rl}
    \toprule
    {\bf radiusMedium} & {\bf Description}\\
    \midrule\endhead%
     1 & IPv4 (IP version 4)\\
     2 & IPv6 (IP version 6)\\
     3 & NSAP\\
     4 & HDLC (8-bit multidrop)\\
     5 & BBN 1822\\
     6 & 802 (includes all 802 media plus Ethernet ``canonical format'')\\
     7 & E.163 (POTS)\\
     8 & E.164 (SMDS, Frame Relay, ATM)\\
     9 & F.69 (Telex)\\
    10 & X.121 (X.25, Frame Relay)\\
    11 & IPX\\
    12 & Appletalk\\
    13 & Decnet IV\\
    14 & Banyan Vines\\
    15 & E.164 with NSAP format subaddress\\
    \bottomrule
\end{longtable}

\subsubsection{radiusAcctStatType}\label{radiusAcctStatType}
The {\tt radiusAcctStatType} column is to be interpreted as follows:
\begin{longtable}{rl}
    \toprule
    {\bf radiusAcctStatType} & {\bf Description}\\
    \midrule\endhead%
     1 & Start\\
     2 & Stop\\
     3 & Interim-Update\\
     7 & Accounting-On\\
     8 & Accounting-Off\\
     9 & Tunnel-Start\\
    10 & Tunnel-Stop\\
    11 & Tunnel-Reject\\
    12 & Tunnel-Link-Start\\
    13 & Tunnel-Link-Stop\\
    14 & Tunnel-Link-Reject\\
    15 & Failed\\
    \bottomrule
\end{longtable}

\subsubsection{radiusAcctTerm}\label{radiusAcctTerm}
The {\tt radiusAcctTerm} column is to be interpreted as follows:
\begin{longtable}{rl}
    \toprule
    {\bf radiusAcctTerm} & {\bf Description}\\
    \midrule\endhead%
     1 & User Request\\
     2 & Lost Carrier\\
     3 & Lost Service\\
     4 & Idle Timeout\\
     5 & Session Timeout\\
     6 & Admin Reset\\
     7 & Admin Reboot\\
     8 & Port Error\\
     9 & NAS Error\\
    10 & NAS Request\\
    11 & NAS Reboot\\
    12 & Port Unneeded\\
    13 & Port Preempted\\
    14 & Port Suspended\\
    15 & Service Unavailable\\
    16 & Callback\\
    17 & User Error\\
    18 & Host Request\\
    19 & Supplicant Restart\\
    20 & Reauthentication Failure\\
    21 & Port Reinitialized\\
    22 & Port Administratively Disabled\\
    23 & Lost Power\\
    \bottomrule
\end{longtable}

\subsection{Packet File Output}
In packet mode ({\tt --s} option), the radiusDecode plugin outputs the following columns:
\begin{longtable}{>{\tt}lll>{\tt\small}l}
    \toprule
    {\bf Column} & {\bf Type} & {\bf Description} & {\bf Flags}\\
    \midrule\endhead%
    \nameref{radiusStat} & H8    & Status         & \\
    radiusCode           & U8    & Code           & RADIUS\_NMS=1\\
    radiusCodeNm         & S     & Code name      & RADIUS\_NMS=2\\
    radiusAVPTypes       & R(U8) & AVP types      & RADIUS\_AVPTYPE=1\&\&RADIUS\_NMS=1\\
    radiusAVPTypeNms     & R(S)  & AVP type names & RADIUS\_AVPTYPE=1\&\&RADIUS\_NMS=2\\
    \bottomrule
\end{longtable}

\subsection{Monitoring Output}
In monitoring mode, the radiusDecode plugin outputs the following columns:
\begin{longtable}{>{\tt}lll>{\tt\small}l}
    \toprule
    {\bf Column} & {\bf Type} & {\bf Description} & {\bf Flags}\\
    \midrule\endhead%
    radiusPkts       & U64 & Number of RADIUS packets     & \\
    radiusAxsPkts    & U64 & Number of Access             & \\
    radiusAxsAccPkts & U64 & Number of Access-Accept      & \\
    radiusAxsRejPkts & U64 & Number of Access-Reject      & \\
    radiusAccPkts    & U64 & Number of Accounting packets & \\
    \bottomrule
\end{longtable}

\subsection{Plugin Report Output}
The following information is reported:
\begin{itemize}
    \item Aggregated {\tt\nameref{radiusStat}}
    \item Number of RADIUS packets
    \item Number of Access, Access-Accept, Access-Reject and Accounting packets
\end{itemize}

\subsection{References}
\begin{itemize}
    \item \href{https://tools.ietf.org/html/rfc2865}{RFC2865}: Remote Authentication Dial In User Service (RADIUS)
    \item \href{https://tools.ietf.org/html/rfc2866}{RFC2866}: RADIUS Accounting
    \item \href{https://tools.ietf.org/html/rfc2867}{RFC2867}: RADIUS Accounting Modifications for Tunnel Protocol Support
    \item \href{https://tools.ietf.org/html/rfc2868}{RFC2868}: RADIUS Attributes for Tunnel Protocol Support
    \item \href{https://tools.ietf.org/html/rfc2869}{RFC2869}: RADIUS Extensions
    \item \url{https://www.iana.org/assignments/radius-types/radius-types.xhtml}
\end{itemize}

\end{document}
