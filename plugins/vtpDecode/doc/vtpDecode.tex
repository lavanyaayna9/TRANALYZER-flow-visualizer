\IfFileExists{t2doc.cls}{
    \documentclass[documentation]{subfiles}
}{
    \errmessage{Error: could not find 't2doc.cls'}
}

\begin{document}

% Declare author and title here, so the main document can reuse it
\trantitle
    {vtpDecode} % Plugin name
    {VLAN Trunking Protocol (VTP)} % Short description
    {Tranalyzer Development Team} % author(s)

\section{vtpDecode}\label{s:vtpDecode}

\subsection{Description}
The vtpDecode plugin analyzes the VLAN Trunking Protocol (VTP) protocol.

\subsection{Dependencies}

\subsubsection{Core Configuration}
This plugin requires the following core configuration:
\begin{itemize}
    \item {\em \$T2HOME/tranalyzer2/src/networkHeaders.h}:
        \begin{itemize}
            \item {\tt ETH\_ACTIVATE>0}
        \end{itemize}
\end{itemize}

\subsection{Configuration Flags}
The following flags can be used to control the output of the plugin:
\begin{longtable}{>{\tt}lcl>{\tt\small}l}
    \toprule
    {\bf Name}        & {\bf Default}           & {\bf Description}                              & {\bf Flags}\\
    \midrule\endhead%
    VTP\_AGGR         & 1                       & Aggregate updater identity                     & \\
    VTP\_SAVE         & 1                       & Extract all VLANs info in a separate file      & \\
    VTP\_DEBUG        & 0                       & Print debug messages                           & \\
    VTP\_TS\_FRMT     & 1                       & Format for timestamps: 0: string, 1: timestamp & \\
    VTP\_NUM\_UPDID   & 16                      & Max number of updater identity                 & \\
    VTP\_STR\_MAX     & 64                      & Max length for strings                         & \\
    VTP\_SUFFIX       & {\tt\small "\_vtp.txt"} & Suffix for separate file                       & VTP\_SAVE=1\\
    VTP\_VLANID\_FRMT & 1                       & Format for VLAN ID: 0: int, 1: hex             & VTP\_SAVE=1\\
    \bottomrule
\end{longtable}

\subsubsection{Environment Variable Configuration Flags}
The following configuration flags can also be configured with environment variables ({\tt ENVCNTRL>0}):
\begin{itemize}
    \item {\tt VTP\_SUFFIX}
\end{itemize}

\subsection{Flow File Output}
The vtpDecode plugin outputs the following columns:
\begin{longtable}{>{\tt}lll>{\tt\small}l}
    \toprule
    {\bf Column}                          & {\bf Type} & {\bf Description}          & {\bf Flags}\\
    \midrule\endhead%
    \nameref{vtpStat}                     & H8         & Status                     & \\
    vtpVer                                & H8         & Version                    & \\
    \hyperref[vtpCode]{vtpCodeBF}         & H8         & Aggregated codes           & \\
    \hyperref[vtpVlanType]{vtpVlanTypeBF} & H8         & Aggregated VLAN types      & \\
    vtpDomain                             & S          & Management Domain          & \\
    vtpNumUpdId                           & U32        & Number of Updater identity & VTP\_NUM\_UPDID>0\\
    vtpUpdId                              & R(IP4)     & Updater identity           & VTP\_NUM\_UPDID>0\\
    vtpFirstUpdTS                         & S/TS       & Timestamp of first update  & VTP\_TS\_FRMT=0/1\\
    vtpLastUpdTS                          & S/TS       & Timestamp of last update   & VTP\_TS\_FRMT=0/1\\
    \bottomrule
\end{longtable}

\subsubsection{vtpStat}\label{vtpStat}
The {\tt vtpStat} column is to be interpreted as follows:
\begin{longtable}{>{\tt}rl}
    \toprule
    {\bf vtpStat} & {\bf Description}\\
    \midrule\endhead%
    0x000\textcolor{magenta}{1} & Flow is VTP\\
    0x000\textcolor{magenta}{2} & Different versions used\\
    0x000\textcolor{magenta}{4} & Different Management Domains used\\
    0x000\textcolor{magenta}{8} & -\\
    \\
    0x00\textcolor{magenta}{1}0 & Invalid Management Domain Length (> 32)\\
    0x00\textcolor{magenta}{2}0 & Invalid version\\
    0x00\textcolor{magenta}{4}0 & Invalid code\\
    0x00\textcolor{magenta}{8}0 & Invalid VLAN type\\
    \\
    0x0\textcolor{magenta}{1}00 & ---\\
    0x0\textcolor{magenta}{2}00 & ---\\
    0x0\textcolor{magenta}{4}00 & ---\\
    0x0\textcolor{magenta}{8}00 & ---\\
    \\
    0x\textcolor{magenta}{1}000 & ---\\
    0x\textcolor{magenta}{2}000 & Array truncated\ldots increase {\tt VTP\_NUM\_UPDID}\\
    0x\textcolor{magenta}{4}000 & String truncated\ldots increase {\tt VTP\_STR\_MAX}\\
    0x\textcolor{magenta}{8}000 & Packet snapped, decoding failed\\
    \bottomrule
\end{longtable}

\subsubsection{vtpCode and vtpCodeBF}\label{vtpCode}
The {\tt vtpCode} and {\tt vtpCodeBF} columns are to be interpreted as follows:\\
\begin{longtable}{>{\tt}r>{\tt}rl}
    \toprule
    {\bf vtpCode} & {\bf vtpCodeBF} & {\bf Description}\\
    \midrule\endhead%
    ---  & 0x01 & ---\\
    0x01 & 0x02 & Summary Advertisement\\
    0x02 & 0x04 & Subset Advertisement\\
    0x03 & 0x08 & Advertisement Request\\
    \\
    0x04 & 0x10 & Join/Prune Message\\
    ---  & 0x20 & ---\\
    ---  & 0x40 & ---\\
    ---  & 0x80 & Unknown VTP code\\
    \bottomrule
\end{longtable}

\subsubsection{vtpVlanType and vtpVlanTypeBF}\label{vtpVlanType}
The {\tt vtpVlanType} and {\tt vtpVlanTypeBF} columns are to be interpreted as follows:\\
\begin{longtable}{>{\tt}r>{\tt}rl}
    \toprule
    {\bf vtpVlanType} & {\bf vtpVlanTypeBF} & {\bf Description}\\
    \midrule\endhead%
    ---  & 0x01 & ---\\
    0x01 & 0x02 & Ethernet\\
    0x02 & 0x04 & Fiber Distributed Data Interface (FDDI)\\
    0x03 & 0x08 & Token Ring Concentrator Relay Function (TrCRF)\\
    \\
    0x04 & 0x10 & Fiber Distributed Data Interface Network Entity Title (FFID-net)\\
    0x05 & 0x20 & Token Ring Bridge Relay Function (TrBRF)\\
    ---  & 0x40 & ---\\
    ---  & 0x80 & Unknown VTP VLAN type\\
    \bottomrule
\end{longtable}

\subsection{Packet File Output}
In packet mode ({\tt --s} option), the vtpDecode plugin outputs the following columns:
\begin{longtable}{>{\tt}lll>{\tt\small}l}
    \toprule
    {\bf Column} & {\bf Type} & {\bf Description} & {\bf Flags}\\
    \midrule\endhead%
    \nameref{vtpStat}                     & H8 & Status               & \\
    vtpVer                                & H8 & Version              & \\
    \hyperref[vtpCode]{vtpCode}           & H8 & Code                 & \\
    vtpDomain                             & SC & Management Domain    & \\
    \hyperref[vtpVlanType]{vtpVlanTypeBF} & H8 & Aggregated VLAN type & \\
    \bottomrule
\end{longtable}

\subsection{Plugin Report Output}
The following information is reported:
\begin{itemize}
    \item Aggregated {\tt\nameref{vtpStat}}
    \item Aggregated {\tt\hyperref[vtpCode]{vtpCodeBF}}
    \item Aggregated {\tt\hyperref[vtpVlanType]{vtpVlanTypeBF}}
    \item Number of VTP packets
    \item Number of VTPv1, VTPv2 and VTPv3 packets
    \item Number of VTP Summary Advertisement packets
    \item Number of VTP Subset Advertisement packets
    \item Number of VTP Advertisement Request packets
    \item Number of VTP Join/Prune Message packets
    \item Number of VTP packets with unknown type
\end{itemize}

\subsection{Additional Output}
Non-standard output:
\begin{itemize}
    \item {\tt PREFIX\_vtp.txt}: List of VLANs extracted from Subset Advertisement messages
\end{itemize}

The {\tt PREFIX\_vtp.txt} file contains the following columns:
\begin{longtable}{>{\tt}rl}
    \toprule
    {\bf Name} & {\bf Description}\\
    \midrule\endhead%
    pktNo                               & Packet number\\
    flowInd                             & Flow index\\
    srcMac                              & MAC address which issued this advertisement\\
    vtpVer                              & VTP version\\
    vtpDomain                           & VTP Management Domain\\
    vtpRevNum                           & VTP Configuration Revision Number\\
    \hyperref[vtpVlanType]{vtpVlanType} & Aggregated VLAN type\\
    vtpVlanID                           & ISL VLAN ID\\
    vtpVlanName                         & VLAN Name\\
    vtpVlanSAID                         & 802.10 Index (IEEE 802.10 security association identifier for this VLAN)\\
    vtpVlanMTU                          & MTU Size\\
    vtpVlanSuspended                    & State of the VLAN (suspended or not)\\
    \bottomrule
\end{longtable}

\subsection{References}
\begin{itemize}
    \item \href{http://www.cisco.com/en/US/tech/tk389/tk689/technologies_tech_note09186a0080094c52.shtml}{Understanding VLAN Trunk Protocol (VTP)}
\end{itemize}

\end{document}
