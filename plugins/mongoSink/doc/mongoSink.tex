\IfFileExists{t2doc.cls}{
    \documentclass[documentation]{subfiles}
}{
    \errmessage{Error: could not find 't2doc.cls'}
}

\begin{document}

\trantitle
    {mongoSink} % Plugin name
    {MongoDB} % Short description
    {Tranalyzer Development Team} % author(s)

\section{mongoSink}\label{s:mongoSink}

\subsection{Description}
The mongoSink plugin outputs flows to a MongoDB database.

\subsection{Dependencies}

\subsubsection{External Libraries}
This plugin depends on the {\bf libmongoc} library.
\begin{table}[!ht]
    \centering
    \begin{tabular}{>{\bf}r>{\tt}l>{\tt}l}
        \toprule
        %                             &                      &              \\
        %\midrule
        Ubuntu:                      & sudo apt-get install & libmongoc-dev\\
        Arch:                        & sudo pacman -S       & mongo-c-driver\\
        Gentoo:                      & sudo emerge          & mongo-c-driver\\
        %openSUSE:                    & sudo zypper install  & libXXX-devel\\
        Red Hat/Fedora\tablefootnote{If the {\tt dnf} command could not be found, try with {\tt yum} instead}:
                                     & sudo dnf install     & mongo-c-driver-devel\\
        macOS\tablefootnote{Brew is a packet manager for macOS that can be found here: \url{https://brew.sh}}:
                                     & brew install         & mongo-c-driver\\
        \bottomrule
    \end{tabular}
\end{table}

\subsubsection{Core Configuration}
This plugin requires the following core configuration:
\begin{itemize}
    \item {\em \$T2HOME/tranalyzer2/src/tranalyzer.h}:
        \begin{itemize}
            \item {\tt BLOCK\_BUF=0}
        \end{itemize}
\end{itemize}

\subsection{Configuration Flags}
The following flags can be used to control the output of the plugin:
\begin{longtable}{>{\tt}lcl}
    \toprule
    {\bf Name} & {\bf Default} & {\bf Description}\\
    \midrule\endhead%
    MONGO\_HOST        & {\small\tt "127.0.0.1"}  & Address of the database\\
    MONGO\_PORT        & {\small\tt 27017}        & Port the database is listening to\\
    MONGO\_DBNAME      & {\small\tt "tranalyzer"} & Name of the database\\
    MONGO\_TABLE\_NAME & {\small\tt "flow"}       & Name of the database flow table\\
    MONGO\_NUM\_DOCS   & 1                        & Number of documents (flows) to write in bulk\\
    \\
    MONGO\_QRY\_LEN    & 2048                     & Max length for query\\
    \hyperref[mongo:select]{MONGO\_SELECT}
                       & 0                        & Only insert specific fields into the DB\\
    \hyperref[mongo:select]{MONGO\_SELECT\_FILE}
                       & {\small\tt "mongo-columns.txt"}
                                                  & Filename of the field selector (one column name per line)\\
    \\
    BSON\_SUPPRESS\_EMPTY\_ARRAY & 1              & Output empty fields\\
    BSON\_DEBUG                  & 0              & Print debug messages\\
    \bottomrule
\end{longtable}

\subsubsection{Environment Variable Configuration Flags}
The following configuration flags can also be configured with environment variables ({\tt ENVCNTRL>0}):
\begin{itemize}
    \item {\tt MONGO\_HOST}
    \item {\tt MONGO\_PORT}
    \item {\tt MONGO\_DBNAME}
    \item {\tt MONGO\_TABLE\_NAME}
    \item {\tt MONGO\_SELECT\_FILE} (require {\tt MONGO\_SELECT=1})
\end{itemize}

\subsection{Insertion of Selected Fields Only}\label{mongo:select}
When {\tt MONGO\_SELECT=1}, the columns to insert into the DB can be customized with the help of {\tt MONGO\_SELECT\_FILE}.
The filename defaults to {\tt mongo-columns.txt} in the user plugin folder, e.g., {\em \textasciitilde{}/.tranalyzer/plugins}.
The format of the file is simply one field name per line with lines starting with a {\tt `\#'} being ignored.
For example, to only insert source and destination addresses and ports, create the following file:

\begin{verbatim}
# Lines starting with a '#' are ignored and can be used to add comments
srcIP
srcPort
dstIP
dstPort
\end{verbatim}

\subsection{Working with Timestamps (ISODate)}
MongoDB stores timestamps in UTC as {\tt ISODate}.
To convert them to localtime, you may use the following query:
\begin{verbatim}
> db.flow.aggregate([{
      $project: {
          localTime: {
              $dateToString: {
                  date: "$timeFirst",
                  format: "%Y-%m-%d %H:%M:%S",
                  timezone: "Europe/Berlin"
              }
          }
      }
  }])
\end{verbatim}

\subsection{Example}

{\tt\color{blue} \# Run Tranalyzer}\\
{\tt \$ t2 -r file.pcap}\\

\noindent
{\tt\color{blue} \# Connect to the Mongo database}\\
{\tt \$ mongosh tranalyzer}\\

\noindent
{\tt\color{blue} \# Number of flows}\\
{\tt > db.flow.countDocuments()}\\

\noindent
{\tt\color{blue} \# 10 first srcIP/dstIP pairs}\\
{\tt > db.flow.find(\{\}, \{ \_id: 0, srcIP: 1, dstIP: 1 \}).limit(10)}\\

\noindent
{\tt\color{blue} \# All flows from 1.2.3.4 to 1.2.3.5}\\
{\tt > db.flow.find(\{ srcIP: "1.2.3.4", dstIP: "1.2.3.5" \})}\\

\noindent
For examples of more complex queries, have a look in {\tt \$T2HOME/scripts/t2fm/mongo/}.

\subsubsection{Clean up an existing database}

{\tt\color{blue} \# Connect to the Mongo database}\\
{\tt \$ mongosh tranalyzer}\\

\noindent
{\tt\color{blue} \# Drop the database}\\
{\tt > db.flow.drop()}\\

\end{document}
