\IfFileExists{t2doc.cls}{
    \documentclass[documentation]{subfiles}
}{
    \errmessage{Error: could not find 't2doc.cls'}
}

\begin{document}

\trantitle
    {wechatDecode}
    {WeChat}
    {Tranalyzer Development Team}

\section{wechatDecode}\label{s:wechatDecode}

\subsection{Description}

The wechatDecode plugin detects and decodes JCE encoded data found in observed HTTP traffic and writes the results as JSON into a file.\\

To identify relevant packets to decode, the HTTP payload is searched for the string {\tt "TMASDK\_"}.
If this string is found, the HTTP payload is processed using a JCE decoder.\\

Decryption and decompression of the body are handled automatically.\\

\subsection{Dependencies}

\subsubsection{External Libraries}
This plugin depends on the {\bf zlib} library.
\begin{table}[!ht]
    \centering
    \begin{tabular}{>{\bf}r>{\tt}l>{\tt}l}
        \toprule
        %                             &                      &\\
        %\midrule
        Ubuntu:                      & sudo apt-get install & zlib1g-dev\\
        Arch:                        & sudo pacman -S       & zlib\\
        Gentoo:                      & sudo emerge          & zlib\\
        openSUSE:                    & sudo zypper install  & zlib-devel\\
        Red Hat/Fedora\tablefootnote{If the {\tt dnf} command could not be found, try with {\tt yum} instead}:
                                     & sudo dnf install     & zlib-devel\\
        macOS\tablefootnote{Brew is a packet manager for macOS that can be found here: \url{https://brew.sh}}:
                                     & brew install         & zlib\\
        \bottomrule
    \end{tabular}
\end{table}

\subsection{Configuration Flags}
The following flags can be used to control the output of the plugin:
\begin{longtable}{>{\tt}lcl}
    \toprule
    {\bf Name} & {\bf Default} & {\bf Description}\\
    \midrule\endhead%
    WECHAT\_JSON\_SUFFIX                & {\tt\small "\_wechat.json"} & Suffix appended to the base output file name\\
    WECHAT\_JSON\_ARRAY                 & 0                           & 0: Output a single JSON array\\
                                        &                             & 1: Output a line-delimited JSON\\
    WECHAT\_INITIAL\_JSON\_BUFFER\_SIZE & 2048                        & Initial size of output buffer (increased dynamically)\\
    WECHAT\_MAX\_HTTP\_HDR\_FIELD\_LEN  & 1024                        & Maximum length of a HTTP header field\\
    WECHAT\_MAX\_QUA\_MATCH\_LEN        & 255                         & Max.\ length of a matching group in the QUA string\\
    WECHAT\_VERBOSITY\_LEVEL            & 0                           & Verbosity level:\\
                                        &                             & \qquad 0: Quiet mode (only warnings and errors)\\
                                        &                             & \qquad 1: Debug mode\\
    \bottomrule
\end{longtable}

\subsection{Flow File Output}
This plugin does not output any columns in the flow file. It writes the decoded data directly to the JSON output file.

\subsection{Custom File Output}
The wechatDecode plugin produces JSON output in the file {\tt PREFIX\_wechat.json}, where {\tt PREFIX} is provided via the Tranalyzer option {\tt --w} or {\tt --W} and the {\tt \_wechat.json} suffix can be overridden via the configuration flag {\tt WECHAT\_JSON\_SUFFIX}.

\subsubsection{JSON format}

For each packet detected to contain JCE data, the HTTP payload is decoded, decrypted and, if necessary, decompressed. The extracted data is then written as one JSON object per packet.\\

Each JSON object is divided into three sections, i.e., keys: {\tt flow}, {\tt requestHeader} and {\tt body}. The flow section contains the six-tuple and the {\tt firstSeen} timestamp of the flow. The requestHeader contains any information extracted from the {\tt reqHead} part of the JCE payload, including any optional fields. The {\tt body} section contains the decrypted and decoded information found in the body part of the decoded HTTP payload.\\

Numerical values in the {\tt requestHeader} extracted from the {\tt reqHead} are printed as quoted JSON strings as there are no static guarantees on the bounds of the matched numbers. This avoids integer overflows and unexpected incorrect results.\\

The following currently understood body types are supported:

\begin{itemize}
    \item {\tt ReportLog}
    \item {\tt GetSettings}
    \item {\tt GetConfig}
    \item {\tt StatReport}
\end{itemize}

If the configuration flag {\tt WECHAT\_JSON\_ARRAY} is set to {\tt 0}, the JSON objects are output as line-delimited JSON, i.e. one JSON object per line. Note that in this mode, a JSON object may not contain a newline character as it is already used as a delimiter between individual JSON objects.\\

If {\tt WECHAT\_JSON\_ARRAY} is {\tt 1}, the JSON objects are enclosed in a JSON array and separated with a comma to yield valid JSON output. The line-delimited JSON mode is more robust as it does not rely on proper termination of the application to write the closing square bracket.

\end{document}
