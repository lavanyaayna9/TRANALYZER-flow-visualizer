\IfFileExists{t2doc.cls}{
    \documentclass[documentation]{subfiles}
}{
    \errmessage{Error: could not find 't2doc.cls'}
}

\begin{document}

% Declare author and title here, so the main document can reuse it
\trantitle
    {kafkaSink} % Plugin name
    {Apache Kafka} % Short description
    {Tranalyzer Development Team} % author(s)

\section{kafkaSink}\label{s:kafkaSink}

\subsection{Description}
The kafkaSink plugin outputs flows to an Apache Kafka event streaming platform.

\subsection{Dependencies}

\subsubsection{External Libraries}
This plugin depends on the {\bf librdkafka} library.
\begin{table}[!ht]
    \centering
    \begin{tabular}{>{\bf}r>{\tt}l>{\tt}l>{\tt}l}
        \toprule
                                     &                      & {\bf OPT2=1}\\
        \midrule
        Ubuntu:                      & sudo apt-get install & librdkafka-dev\\
        Arch:                        & sudo pacman -S       & librdkafka\\
        Gentoo:                      & sudo emerge          & librdkafka\\
        openSUSE:                    & sudo zypper install  & librdkafka-devel\\
        Red Hat/Fedora\tablefootnote{If the {\tt dnf} command could not be found, try with {\tt yum} instead}:
                                     & sudo dnf install     & librdkafka-devel\\
        macOS\tablefootnote{Brew is a packet manager for macOS that can be found here: \url{https://brew.sh}}:
                                     & brew install         & librdkafka\\
        \bottomrule
    \end{tabular}
\end{table}

\subsubsection{Core Configuration}
This plugin requires the following core configuration:
\begin{itemize}
    \item {\em \$T2HOME/tranalyzer2/src/tranalyzer.h}:
        \begin{itemize}
            \item {\tt BLOCK\_BUF=0}
        \end{itemize}
\end{itemize}

\subsection{Services Initialization}

The kafkaSink plugin requires a ZooKeeper and a Kafka broker service running on {\tt KAFKA\_BROKERS} (default address and port are {\tt 127.0.0.1:9092}).\\

\noindent
{\tt \color{blue}{\# Start the ZooKeeper server and send it to the background}}\\
{\tt \$ zookeeper-server-start.sh /etc/kafka/zookeeper.properties \&}\\
\\
{\tt \color{blue}{\# Start the Kafka server and send it to the background}}\\
{\tt \$ kafka-server-start.sh /etc/kafka/server.properties \&}\\

\clearpage
\subsection{Configuration Flags}
The following flags can be used to control the output of the plugin:
\begin{longtable}{>{\tt}lcl}
    \toprule
    {\bf Name} & {\bf Default} & {\bf Description}\\
    \midrule\endhead%
    KAFKA\_BROKERS   & {\tt\small "127.0.0.1:9092"}   & Broker address(es)                                              \\
                     &                                & (comma separated list of host[:port])                           \\
    KAFKA\_TOPIC     & {\tt\small "tranalyzer.flows"} & Topic to produce to                                             \\
    KAFKA\_PARTITION & -1                             & Target partition:                                               \\
                     &                                & \qquad $\geq 0$: fixed partition                                \\
                     &                                & \qquad -1: automatic partitioning (unassigned)                  \\
    KAFKA\_RETRIES   & 3                              & Max. number of retries when message production failed [0 - 255] \\
    KAFKA\_DEBUG     & 0                              & Print debug messages                                            \\
    \bottomrule
\end{longtable}

\subsubsection{Environment Variable Configuration Flags}
The following configuration flags can also be configured with environment variables ({\tt ENVCNTRL>0}):
\begin{itemize}
    \item {\tt KAFKA\_BROKERS}
    \item {\tt KAFKA\_TOPIC}
    \item {\tt KAFKA\_PARTITION}
\end{itemize}

\subsection{Plugin Report Output}
The following information is reported:
\begin{itemize}
    \item Number of flows discarded
\end{itemize}

\subsection{Example}

In this example, the flows will be sent to Kafka to the {\tt tranalyzer.flows} topic.
In addition, Tranalyzer information ({\tt [INF]}) and warnings ({\tt [WRN]}) will be sent to the {\tt tranalyzer.out}, while errors ({\tt [ERR]}) will use the {\tt tranalyzer.err} topic.\\

\noindent
First, we want to prevent Tranalyzer from coloring the output:\\

\noindent
{\tt \$ t2conf tranalyzer2 -D T2\_LOG\_COLOR=0}\\

\noindent
In this example, only the \tranrefpl{basicFlow} and \tranrefpl{kafkaSink} plugins will be used:\\

\noindent
{\tt \$ t2build tranalyzer2 basicFlow kafkaSink}\\

\noindent
Now, the fun part! Run Tranalyzer as per usual, but redirect {\tt stdout} and {\tt stderr} to a {\tt kcat}\footnote{{\tt kcat} was formerly known as {\tt kafkacat}} process, which will send the data to Kafka:\\

\noindent
{\tt \$ t2 -r file.pcap \textbackslash{}}\\
\indent {\tt \qquad 1> >(grep -F -e "[INF]" -e "[WRN]" | kcat -P -b 127.0.0.1:9092 -t tranalyzer.out) \textbackslash{}}\\
\indent {\tt \qquad 2> >(kcat -P -b 127.0.0.1:9092 -t tranalyzer.err)}\\

\noindent
The messages can now be consumed with the help of {\tt kafka-console-consumer}.\\

\noindent
{\tt \color{blue}{\# Consume messages for tranalyzer.flows topic}}\\
{\tt \$ kafka-console-consumer \textbackslash{}}\\
\indent {\tt \qquad -{}-bootstrap-server localhost:9092 \textbackslash{}}\\
\indent {\tt \qquad -{}-from-beginning \textbackslash{}}\\
\indent {\tt \qquad -{}-topic tranalyzer.flows}\\

\noindent
{\tt \color{blue}{\# Consume messages for tranalyzer.out topic}}\\
{\tt \$ kafka-console-consumer \textbackslash{}}\\
\indent {\tt \qquad -{}-bootstrap-server localhost:9092 \textbackslash{}}\\
\indent {\tt \qquad -{}-from-beginning \textbackslash{}}\\
\indent {\tt \qquad -{}-topic tranalyzer.out}\\

\noindent
{\tt \color{blue}{\# Consume messages for tranalyzer.err topic}}\\
{\tt \$ kafka-console-consumer \textbackslash{}}\\
\indent {\tt \qquad -{}-bootstrap-server localhost:9092 \textbackslash{}}\\
\indent {\tt \qquad -{}-from-beginning \textbackslash{}}\\
\indent {\tt \qquad -{}-topic tranalyzer.err}\\

\end{document}
