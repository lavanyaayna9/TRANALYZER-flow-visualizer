\IfFileExists{t2doc.cls}{
    \documentclass[documentation]{subfiles}
}{
    \errmessage{Error: could not find t2doc.cls}
}

\begin{document}

\trantitle
    {dnsDecode}
    {Domain Name System (DNS)}
    {Tranalyzer Development Team}

\section{dnsDecode}\label{s:dnsDecode}

\subsection{Description}
The dnsDecode plugin analyzes DNS traffic.

\subsection{Configuration Flags}\label{dnsDecode:config}
The following flags can be used to control the output of the plugin:
\begin{longtable}{>{\tt}lcl}
    \toprule
    {\bf Name} & {\bf Default} & {\bf Description}\\
    \midrule\endhead%
    DNS\_MODE      &  4 & 0: Only aggregated header count info \\
                   &    & 1: +REQ records \\
                   &    & 2: +ANS records \\
                   &    & 3: +AUX records \\
                   &    & 4: +ADD records\\
    DNS\_HEXON     &  1 & 0: Hex output flags off\\
                   &    & 1: Hex output flags on\\
    DNS\_HDRMD     &  0 & Header, OpCode, RetCode:\\
                   &    & \qquad 0: Bitfield\\
                   &    & \qquad 1: Numeric\\
                   &    & \qquad 2: String\\
    DNS\_AGGR      &  0 & 0: Full vectors\\
                   &    & 1: Aggregate records\\
    DNS\_TYPE      &  0 & Q/A type format:\\
                   &    & \qquad 0: Numeric\\
                   &    & \qquad 1: String\\
    DNS\_QRECMAX   & 15 & Max \# of query records / flow \\
    DNS\_ARECMAX   & 20 & Max \# of answer records / flow \\
    DNS\_WHO       & 0  & 1: Output country and organization of DNS reply addresses\\
    DNS\_MAL\_TEST & 0  & 0: No tests for malware\\
                   &    & 1: Mal test @ flow terminated \\
                   &    & 2: Mal test @ L4Callback, pcad ops \\
    DNS\_MAL\_TYPE & 0  & Malware type format:\\
                   &    & \qquad 0: code\\
                   &    & \qquad 1: string\\

    \\
    \multicolumn{3}{l}{The following additional flag is available in {\tt malsite.h}:}\\
    \\

    DNS\_MAL\_DOMAIN & 1 & 0: Malsite IP address labeling mode \\
                     &   & 1: Malsite domain labeling mode, not implemented yet\\
    \bottomrule
\end{longtable}

{\tt DNS\_MAL\_TEST} controls where the mal test is performed. Only in L4Callback enables a cooperation
with pcaps, so that \tranrefpl{pcapd} dumps all packets of a flow after the alarm was detected.

\clearpage

\subsection{Flow File Output}
The dnsDecode plugin outputs the following columns:
\begin{longtable}{>{\tt}lll>{\tt\small}l}
    \toprule
    {\bf Column}                          & {\bf Type}  & {\bf Description}                   & {\bf Flags}\\
    \midrule\endhead%

    \nameref{dnsStat}                     & H16         & Status, warnings and errors         & \\
    \nameref{dnsHdrOPField}               & H16         & Header field of last packet in flow & \\
    \hyperref[dnsHFlgOpCRetC]{dnsHFlg\_}  & H8\_        & Aggregated header flags,            & DNS\_HDRMD=0\\
    \quad\hyperref[dnsHFlgOpCRetC]{OpC\_} & \quad H16\_ & \quad operational code and          & \\
    \quad\hyperref[dnsHFlgOpCRetC]{RetC}  & \quad H16   & \quad return code                   & \\
    \hyperref[dnsHFlgOpCRetC]{dnsHFlg}    & H8          & Aggregated header flags             & DNS\_HDRMD>0\\
    \hyperref[dnsHFlgOpCRetC]{dnsOpC}     & H16         & Operational code                    & DNS\_HDRMD=1\\
    \hyperref[dnsHFlgOpCRetC]{dnsOpN}     & S           & Operational string                  & DNS\_HDRMD=2\\
    \hyperref[dnsHFlgOpCRetC]{dnsRetC}    & H16         & Return code                         & DNS\_HDRMD=1\\
    \hyperref[dnsHFlgOpCRetC]{dnsRetN}    & S           & Return string                       & DNS\_HDRMD=2\\
    dnsCntQu\_                            & R(U16\_     & \# of question records,             & \\
    \quad Asw\_                           & \quad U16\_ & \quad answer records,               & \\
    \quad Aux\_                           & \quad U16\_ & \quad auxiliary records and         & \\
    \quad Add                             & \quad U16)  & \quad additional records            & \\
    dnsAAAqF                              & F           & DDOS DNS AAA / query factor         & \\

    \\
    \multicolumn{4}{l}{If {\tt DNS\_MODE>0}, the following columns are displayed:}\\
    \\

    \hyperref[dnsTypeBF]{dnsTypeBF3\_BF2\_BF1\_BF0}
                                    & H8\_H16\_H16\_H64
                                               & Type bitfields                & DNS\_HEXON=1\\

    dnsQname                        & R(S)     & Query name records            & \\
    \\
    dnsMalCnt                       & U32      & Domain malware count          & DNS\_MAL\_TEST>0 \&\&\\
                                    &          &                               & DNS\_MAL\_DOMAIN=1\\
    dnsMalType                      & R(S)     & Domain malware type string    & DNS\_MAL\_TEST>0 \&\&\\
                                    &          &                               & DNS\_MAL\_DOMAIN=1\&\&\\
                                    &          &                               & DNS\_MAL\_TYPE=1\&\&\\
    dnsMalCode                      & R(U32)   & Domain malware code           & DNS\_MAL\_TEST>0 \&\&\\
                                    &          &                               & DNS\_MAL\_DOMAIN=1\&\&\\
                                    &          &                               & DNS\_MAL\_TYPE=0\\
    \\
    dnsAname                        & R(S)     & Answer name records           & \\
    dnsAPname                       & R(S)     & Name CNAME entries            & \\
    dns4Aaddress                    & R(IP4)   & Address entries IPv4          & \\
    dns4CC\_Org                     & R(SC\_S) & IPv4 country and organization & DNS\_WHO=1\\
    dns6Aaddress                    & R(IP6)   & Address entries IPv6          & \\
    dns6CC\_Org                     & R(SC\_S) & IPv6 country and organization & DNS\_WHO=1\\
    dnsIPMalCode                    & R(H32)   & IP malware code               & DNS\_MAL\_TEST>0\&\&\\
                                    &          &                               & DNS\_MAL\_DOMAIN=0\\

    \hyperref[dnsTypeBF]{dnsQType}  & R(U16)   & Query record type entries     & DNS\_TYPE=0\\
    \hyperref[dnsTypeBF]{dnsQTypeN} & R(S)     & Query record type names       & DNS\_TYPE=1\\
    dnsQClass                       & R(U16)   & Query record class entries    & \\
    \hyperref[dnsTypeBF]{dnsAType}  & R(U16)   & Answer record type entries    & DNS\_TYPE=0\\
    \hyperref[dnsTypeBF]{dnsATypeN} & R(S)     & Answer record type names      & DNS\_TYPE=1\\
    dnsAClass                       & R(U16)   & Answer record class entries   & \\
    dnsATTL                         & R(U32)   & Answer record TTL entries     & \\
    dnsMXpref                       & R(U16)   & MX record preference entries  & \\
    dnsSRVprio                      & R(U16)   & SRV record priority entries   & \\
    dnsSRVwgt                       & R(U16)   & SRV record weight entries     & \\
    dnsSRVprt                       & R(U16)   & SRV record port entries       & \\
    dnsOptStat                      & R(H32)   & Option status                 & \\
    \bottomrule
\end{longtable}

\subsubsection{dnsStat}\label{dnsStat}
The {\tt dnsStat} column is to be interpreted as follows:
\begin{longtable}{>{\tt}rl}
    \toprule
    {\bf dnsStat}      & {\bf Description}\\
    \midrule\endhead%
    $2^{0}$  (=0x0001) & DNS ports detected\\
    $2^{1}$  (=0x0002) & NetBIOS DNS\\
    $2^{2}$  (=0x0004) & DNS TCP aggregated fragmented content\\
    $2^{3}$  (=0x0008) & DNS TCP fragmented content state\\
    \\
    $2^{4}$  (=0x0010) & ---\\
    %$2^{4}$  (=0x0010) & Warning: Name truncated\\
    $2^{5}$  (=0x0020) & Warning: ANY: Zone all from a domain or cached server\\
    $2^{6}$  (=0x0040) & Warning: Incremental DNS zone transfer detected\\
    $2^{7}$  (=0x0080) & Warning: DNS zone transfer detected\\
    \\
    $2^{8}$  (=0x0100) & Warning: DNS UDP length exceeded\\
    $2^{9}$  (=0x0200) & Warning: following records ignored\\
    $2^{10}$ (=0x0400) & Warning: Max DNS query records exceeded... increase {\tt DNS\_QRECMAX}\\
    $2^{11}$ (=0x0800) & Warning: Max DNS answer records exceeded... increase {\tt DNS\_ARECMAX}\\
    \\
    $2^{12}$ (=0x1000) & Error: DNS record length error\\
    $2^{13}$ (=0x2000) & Error: Wrong DNS PTR detected\\
    $2^{14}$ (=0x4000) & Warning: DNS length undercut\\
    $2^{15}$ (=0x8000) & Error: UDP/TCP DNS header corrupt or TCP packets missing\\
    \bottomrule
\end{longtable}

\subsubsection{dnsHdrOPField}\label{dnsHdrOPField}
From the 16 bit DNS header the QR bit and bit five to nine are extracted and mapped in their correct sequence
into a byte as indicated below. It provides for a normal single packet exchange flow an accurate status of the
DNS transfer. For a multiple packet exchange only the last packet is mapped into the variable.
In that case the aggregated header state flags should be considered.

\begin{longtable}{cccccccccc}
    \toprule
    {\bf QR} & {\bf Opcode} & {\bf AA} & {\bf TC} & {\bf RD} & {\bf RA} & {\bf Z} & {\bf AD} & {\bf CD} & {\bf Rcode} \\
    \midrule\endhead%
    1        & 0000         & 1        & 0        & 1        & 1        & 1       & 0        & 0        & 0000 \\
    \bottomrule
\end{longtable}

\subsubsection{dnsHFlg\_OpC\_RetC}\label{dnsHFlgOpCRetC}
For multi-packet DNS flows e.g. via TCP the aggregated header state bit field describes
the status of all packets in a flow. Thus, flows with certain client and server states can
be easily identified and extracted during post-processing.

\begin{longtable}{>{\tt}r>{\tt}cl}
    \toprule
    {\bf dnsHFlg} & {\bf Short} & {\bf Description}\\
    \midrule\endhead%
    $2^7$ (=0x01) & CD          & Checking disabled \\
    $2^6$ (=0x02) & AD          & Authenticated data \\
    $2^5$ (=0x04) & Z           & Zone transfer \\
    $2^4$ (=0x08) & RA          & Recursive query support available \\
    \\
    $2^3$ (=0x10) & RD          & Recursion desired \\
    $2^2$ (=0x20) & TC          & Message truncated \\
    $2^1$ (=0x40) & AA          & Authoritative answer\\
    $2^0$ (=0x80) & QR          & 0: Query / 1: Response \\
    \bottomrule
\end{longtable}

The four bit opcode field of the DNS header is mapped via [$2^{\text{opcode}}$] and an OR
into a 16 bit field. Thus, the client can be monitored or anomalies easily identified.
E.g. appearance of reserved bits might be an indication for a covert channel or
malware operation.

\begin{longtable}{>{\tt}rl}
    \toprule
    {\bf dnsOpC}       & {\bf Description}\\
    \midrule\endhead%
    $2^{0}$  (=0x0001) & Standard query \\
    $2^{1}$  (=0x0002) & Inverse query \\
    $2^{2}$  (=0x0004) & Server status request \\
    $2^{3}$  (=0x0008) & --- \\
    \\
    $2^{4}$  (=0x0010) & Notify \\
    $2^{5}$  (=0x0020) & Update / Register (NetBIOS) \\
    $2^{6}$  (=0x0040) & Release (NetBIOS) \\
    $2^{7}$  (=0x0080) & Wait For Acknowledge (NetBIOS) \\
    \\
    $2^{8}$  (=0x0100) & Refresh (NetBIOS) \\
    $2^{9}$  (=0x0200) & reserved \\
    $2^{10}$ (=0x0400) & reserved \\
    $2^{11}$ (=0x0800) & reserved \\
    \\
    $2^{12}$ (=0x1000) & reserved \\
    $2^{13}$ (=0x2000) & reserved \\
    $2^{14}$ (=0x4000) & reserved \\
    $2^{15}$ (=0x8000) & reserved \\
    \bottomrule
\end{longtable}

The four bit rcode field of the DNS header is mapped via [$2^{\text{rcode}}$] and an OR
into a 16 bit field. It provides valuable information about success of DNS queries
and therefore facilitates the detection of failures, misconfigurations and malicious
operations.

\begin{longtable}{>{\tt}rcl}
    \toprule
    {\bf dnsRetC}      & {\bf Short}     & {\bf Description}\\
    \midrule\endhead%
    $2^{0}$  (=0x0001) & No error        & Request completed successfully \\
    $2^{1}$  (=0x0002) & Format error    & Name server unable to interpret query \\
    $2^{2}$  (=0x0004) & Server failure  & Name server unable to process query due to problem with name server \\
    $2^{3}$  (=0x0008) & Name error      & Authoritative name server only: Domain name in query does not exist \\
    \\
    $2^{4}$  (=0x0010) & Not implemented & Name server does not support requested kind of query \\
    $2^{4}$  (=0x0020) & Refused         & Name server refuses to perform the specified operation for policy reasons \\
    $2^{5}$  (=0x0040) & YXDomain        & Name exists when it should not \\
    $2^{6}$  (=0x0080) & YXRRSet         & Resource record set exists when it should not \\
    \\
    $2^{8}$  (=0x0100) & NXRRSet         & Resource record set that should exist does not \\
    $2^{9}$  (=0x0200) & NotAuth         & Server not authoritative for zone \\
    $2^{10}$ (=0x0400) & NotZone         & Name not contained in zone \\
    $2^{11}$ (=0x0800) & ---             & --- \\
    \\
    $2^{12}$ (=0x1000) & ---             & --- \\
    $2^{13}$ (=0x2000) & ---             & --- \\
    $2^{14}$ (=0x4000) & ---             & --- \\
    $2^{15}$ (=0x8000) & ---             & --- \\
    \bottomrule
\end{longtable}

\subsubsection{dnsTypeBF3\_BF2\_BF1\_BF0}\label{dnsTypeBF}
The 16 bit Type Code field is extracted from each DNS record and mapped via [$2^{\text{Typecode}}$] into a
64 bit fields. Gaps are avoided by additional higher bitfields defining higher codes.

\begin{longtable}{>{\tt}rcl}
    \toprule
    {\bf dnsTypeBF3} & {\bf Short} & {\bf Description}\\
    \midrule\endhead%
    $2^0$ (=0x01)    & TA          & DNSSEC Trust Authorities \\
    $2^1$ (=0x02)    & DLV         & DNSSEC Lookaside Validation \\
    $2^2$ (=0x04)    & ---         & --- \\
    $2^3$ (=0x08)    & ---         & --- \\
    \\
    $2^4$ (=0x10)    & ---         & --- \\
    $2^5$ (=0x20)    & ---         & --- \\
    $2^6$ (=0x40)    & ---         & --- \\
    $2^7$ (=0x80)    & ---         & --- \\
    \bottomrule
\end{longtable}

\begin{longtable}{>{\tt}rcl}
    \toprule
    {\bf dnsTypeBF2}   & {\bf Short} & {\bf Description} \\
    \midrule\endhead%
    $2^{0}$  (=0x0001) & TKEY        & Transaction Key \\
    $2^{1}$  (=0x0002) & TSIG        & Transaction Signature \\
    $2^{2}$  (=0x0004) & IXFR        & Incremental transfer \\
    $2^{3}$  (=0x0008) & AXFR        & Transfer of an entire zone \\
    \\
    $2^{4}$  (=0x0010) & MAILB       & Mailbox-related RRs (MB, MG or MR) \\
    $2^{5}$  (=0x0020) & MAILA       & Mail agent RRs (OBSOLETE - see MX) \\
    $2^{6}$  (=0x0040) & ZONEALL     & Request for all records the server/cache has available \\
    $2^{7}$  (=0x0080) & URI         & URI \\
    \\
    $2^{8}$  (=0x0100) & CAA         & Certification Authority Restriction \\
    $2^{9}$  (=0x0200) & ---         & --- \\
    $2^{10}$ (=0x0400) & ---         & --- \\
    $2^{11}$ (=0x0800) & ---         & --- \\
    \\
    $2^{12}$ (=0x1000) & ---         & --- \\
    $2^{13}$ (=0x2000) & ---         & --- \\
    $2^{14}$ (=0x4000) & ---         & --- \\
    $2^{15}$ (=0x8000) & ---         & --- \\
    \bottomrule
\end{longtable}

\begin{longtable}{>{\tt}rcl}
    \toprule
    {\bf dnsTypeBF1}   & {\bf Short} & {\bf Description} \\
    \midrule\endhead%
    $2^{0}$  (=0x0001) & SPF         & \\
    $2^{1}$  (=0x0002) & UINFO       & \\
    $2^{2}$  (=0x0004) & UID         & \\
    $2^{3}$  (=0x0008) & GID         & \\
    \\
    $2^{4}$  (=0x0010) & UNSPEC      & \\
    $2^{4}$  (=0x0020) & NID         & \\
    $2^{5}$  (=0x0040) & L32         & \\
    $2^{6}$  (=0x0080) & L64         & \\
    \\
    $2^{8}$  (=0x0100) & LP          & \\
    $2^{9}$  (=0x0200) & EUI48       & EUI-48 address \\
    $2^{10}$ (=0x0400) & EUI64       & EUI-48 address \\
    $2^{11}$ (=0x0800) & ---         & --- \\
    \\
    $2^{12}$ (=0x1000) & ---         & --- \\
    $2^{13}$ (=0x2000) & ---         & --- \\
    $2^{14}$ (=0x4000) & ---         & --- \\
    $2^{15}$ (=0x8000) & ---         & --- \\
    \bottomrule
\end{longtable}

\begin{longtable}{>{\tt}rcl}
    \toprule
    {\bf dnsTypeBF0}                  & {\bf Short} & {\bf Description}\\
    \midrule\endhead%
    $2^{0}$  (=0x0000.0000.0000.0001) & ---         & --- \\
    $2^{1}$  (=0x0000.0000.0000.0002) & A           & IPv4 address \\
    $2^{2}$  (=0x0000.0000.0000.0004) & NS          & Authoritative name server \\
    $2^{3}$  (=0x0000.0000.0000.0008) & MD          & Mail destination. Obsolete use MX instead \\
    \\
    $2^{4}$  (=0x0000.0000.0000.0010) & MF          & Mail forwarder. Obsolete use MX instead \\
    $2^{5}$  (=0x0000.0000.0000.0020) & CNAME       & Canonical name for an alias \\
    $2^{6}$  (=0x0000.0000.0000.0040) & SOA         & Marks the start of a zone of authority \\
    $2^{7}$  (=0x0000.0000.0000.0080) & MB          & Mailbox domain name \\
    \\
    $2^{8}$  (=0x0000.0000.0000.0100) & MG          & Mail group member \\
    $2^{9}$  (=0x0000.0000.0000.0200) & MR          & Mail rename domain name \\
    $2^{10}$ (=0x0000.0000.0000.0400) & NULL        & Null resource record \\
    $2^{11}$ (=0x0000.0000.0000.0800) & WKS         & Well known service description \\
    \\
    $2^{12}$ (=0x0000.0000.0000.1000) & PTR         & Domain name pointer \\
    $2^{13}$ (=0x0000.0000.0000.2000) & HINFO       & Host information \\
    $2^{14}$ (=0x0000.0000.0000.4000) & MINFO       & Mailbox or mail list information \\
    $2^{15}$ (=0x0000.0000.0000.8000) & MX          & Mail exchange \\
    \\
    $2^{16}$ (=0x0000.0000.0001.0000) & TXT         & Text strings \\
    $2^{17}$ (=0x0000.0000.0002.0000) & ---         & Responsible Person \\
    $2^{18}$ (=0x0000.0000.0004.0000) & AFSDB       & AFS Data Base location \\
    $2^{19}$ (=0x0000.0000.0008.0000) & X25         & X.25 PSDN address \\
    \\
    $2^{20}$ (=0x0000.0000.0010.0000) & ISDN        & ISDN address \\
    $2^{21}$ (=0x0000.0000.0020.0000) & RT          & Route Through \\
    $2^{22}$ (=0x0000.0000.0040.0000) & NSAP        & NSAP address. NSAP style A record \\
    $2^{23}$ (=0x0000.0000.0080.0000) & NSAP-PTR    & --- \\
    \\
    $2^{24}$ (=0x0000.0000.0100.0000) & SIG         & Security signature \\
    $2^{25}$ (=0x0000.0000.0200.0000) & KEY         & Security key \\
    $2^{26}$ (=0x0000.0000.0400.0000) & PX          & X.400 mail mapping information \\
    $2^{27}$ (=0x0000.0000.0800.0000) & GPOS        & Geographical Position \\
    \\
    $2^{28}$ (=0x0000.0000.1000.0000) & AAAA        & IPv6 Address \\
    $2^{29}$ (=0x0000.0000.2000.0000) & LOC         & Location Information \\
    $2^{30}$ (=0x0000.0000.4000.0000) & NXT         & Next Domain (obsolete) \\
    $2^{31}$ (=0x0000.0000.8000.0000) & EID         & Endpoint Identifier \\
    \\
    $2^{32}$ (=0x0000.0001.0000.0000) & NIMLOC/NB   & Nimrod Locator / NetBIOS general Name Service \\
    $2^{33}$ (=0x0000.0002.0000.0000) & SRV/NBSTAT  & Server Selection / NetBIOS NODE STATUS \\
    $2^{34}$ (=0x0000.0004.0000.0000) & ATMA        & ATM Address \\
    $2^{35}$ (=0x0000.0008.0000.0000) & NAPTR       & Naming Authority Pointer \\
    \\
    $2^{36}$ (=0x0000.0010.0000.0000) & KX          & Key Exchanger \\
    $2^{37}$ (=0x0000.0020.0000.0000) & CERT        & --- \\
    $2^{38}$ (=0x0000.0040.0000.0000) & A6          & A6 (OBSOLETE - use AAAA) \\
    $2^{39}$ (=0x0000.0080.0000.0000) & DNAME       & --- \\
    \\
    $2^{40}$ (=0x0000.0100.0000.0000) & SINK        & --- \\
    $2^{41}$ (=0x0000.0200.0000.0000) & OPT         & --- \\
    $2^{42}$ (=0x0000.0400.0000.0000) & APL         & --- \\
    $2^{43}$ (=0x0000.0800.0000.0000) & DS          & Delegation Signer \\
    \\
    $2^{44}$ (=0x0000.1000.0000.0000) & SSHFP       & SSH Key Fingerprint \\
    $2^{45}$ (=0x0000.2000.0000.0000) & IPSECKEY    & --- \\
    $2^{46}$ (=0x0000.4000.0000.0000) & RRSIG       & --- \\
    $2^{47}$ (=0x0000.8000.0000.0000) & NSEC        & NextSECure \\
    \\
    $2^{48}$ (=0x0001.0000.0000.0000) & DNSKEY      & --- \\
    $2^{49}$ (=0x0002.0000.0000.0000) & DHCID       & DHCP identifier \\
    $2^{50}$ (=0x0004.0000.0000.0000) & NSEC3       & --- \\
    $2^{51}$ (=0x0008.0000.0000.0000) & NSEC3PARAM  & --- \\
    \\
    $2^{52}$ (=0x0010.0000.0000.0000) & TLSA        & --- \\
    $2^{53}$ (=0x0020.0000.0000.0000) & SMIMEA      & S/MIME cert association \\
    $2^{54}$ (=0x0040.0000.0000.0000) & ---         & \\
    $2^{55}$ (=0x0080.0000.0000.0000) & HIP         & Host Identity Protocol \\
    \\
    $2^{56}$ (=0x0100.0000.0000.0000) & NINFO       & --- \\
    $2^{57}$ (=0x0200.0000.0000.0000) & RKEY        & --- \\
    $2^{58}$ (=0x0400.0000.0000.0000) & TALINK      & Trust Anchor LINK \\
    $2^{59}$ (=0x0800.0000.0000.0000) & CDS         & Child DS \\
    \\
    $2^{60}$ (=0x1000.0000.0000.0000) & CDNSKEY     & DNSKEY(s) the Child wants reflected in DS \\
    $2^{61}$ (=0x2000.0000.0000.0000) & OPENPGPKEY  & OpenPGP Key \\
    $2^{62}$ (=0x4000.0000.0000.0000) & CSYNC       & Child-To-Parent Synchronization \\
    $2^{63}$ (=0x8000.0000.0000.0000) & ---         & \\
    \bottomrule
\end{longtable}

%\subsection{NetBIOS specifics}
%
%NetBIOS names have a suffix with the following meanings:
%
%\begin{longtable}{>{\tt}ll}
%    \toprule
%    {\bf NetBIOS suffix} & {\bf Type} & {\bf Description}\\
%    \midrule\endhead%
%    \bottomrule
%\end{longtable}

\subsection{Packet File Output}
In packet mode ({\tt --s} option), the dnsDecode plugin outputs the following columns:
\begin{longtable}{>{\tt}lll>{\tt\small}l}
    \toprule
    {\bf Column}                                  & {\bf Type}   & {\bf Description}                                            & {\bf Flags}\\
    \midrule\endhead%
    dnsIPs                                        & R(IP)        & IP addresses (A/AAAA records)                                & DNS\_WHO=0\\
    dnsIPs\_cntry\_org                            & R(IP\_S\_S)  & IP addresses, countries and organizations (A/AAAA records)   & DNS\_WHO=1\\
    \nameref{dnsStat}                             & H16          & Status, warnings and errors                                  & \\
    \hyperref[dnsHdrOPField]{dnsHdr}              & H16          & Header field of packet                                       & DNS\_HDRMD=0\\
    \nameref{dnsHFlgOpCRetC}                      & H8\_H16\_H16 & Aggregated header flags, operational and return codes        & DNS\_HDRMD=1\\
    \hyperref[dnsHFlgOpCRetC]{dnsHFlg\_OpN\_RetN} & H8\_S\_S     & Aggregated header flags, operational and return strings      & DNS\_HDRMD=2\\
    %\hyperref[dnsHFlgOpCRetC]{dnsHFlg\_}          & H8\_         & Aggregated header flags,                                     & DNS\_HDRMD=1\\
    %\quad\hyperref[dnsHFlgOpCRetC]{OpC\_}         & \quad H16\_  & \quad operational code and                                   & \\
    %\quad\hyperref[dnsHFlgOpCRetC]{RetC}          & \quad H16    & \quad return code                                            & \\
    %\hyperref[dnsHFlgOpCRetC]{dnsHFlg\_}          & H8\_         & Aggregated header flags,                                     & DNS\_HDRMD=2\\
    %\quad\hyperref[dnsHFlgOpCRetC]{OpN\_}         & \quad S\_    & \quad operational string and                                 & \\
    %\quad\hyperref[dnsHFlgOpCRetC]{RetN}          & \quad S      & \quad return string                                          & \\
    dnsCntQu\_                                    & U16\_        & \# of question records,                                      & \\
    \quad Asw\_                                   & \quad U16\_  & \quad answer records,                                        & \\
    \quad Aux\_                                   & \quad U16\_  & \quad auxiliary records and                                  & \\
    \quad Add                                     & \quad U16    & \quad additional records                                     & \\
    \bottomrule
\end{longtable}

\subsection{Monitoring Output}
In monitoring mode, the dnsDecode plugin outputs the following columns:
\begin{longtable}{>{\tt}lll>{\tt\small}l}
    \toprule
    {\bf Column} & {\bf Type} & {\bf Description}       & {\bf Flags}\\
    \midrule\endhead%
    dnsPkts      & U64        & Number of DNS packets   & \\
    dnsQPkts     & U64        & Number of DNS Q packets & \\
    dnsRPkts     & U64        & Number of DNS R packets & \\
    \bottomrule
\end{longtable}

\subsection{Plugin Report Output}
The following information is reported:
\begin{itemize}
    \item Aggregated {\tt\nameref{dnsStat}}
    \item Aggregated {\tt\hyperref[dnsHFlgOpCRetC]{dnsHFlg}}, {\tt\hyperref[dnsHFlgOpCRetC]{dnsOpC}}, {\tt\hyperref[dnsHFlgOpCRetC]{dnsRetC}}
    \item Number of DNS packets
    \item Number of DNS Q packets
    \item Number of DNS R packets
    \item Number of alarms ({\tt\hyperref[dnsDecode:config]{DNS\_MAL\_TEST}}>0)
\end{itemize}

\subsection{Example Output}
The idea is that the string and integer array elements of question, answer, TTL and Type record entries
match by column index so that easy script based mapping and post processing is possible. A sample output
is shown below. Especially when large records are present the same name is printed several times which
might degrade the readability. Therefore, a next version will have a multiple Aname suppressor switch,
which should be off for script based post-processing.

\begin{small}
    \begin{longtable}{ccccc}
        \toprule
        {\bf Query name}    & {\bf Answer name}                        & {\bf Answer address} & {\bf TTL} & {\bf Type} \\
        \midrule\endhead%
        www.macromedia.com; & www.macromedia.com;www-mm.wip4.adobe.com & 0.0.0.0;8.118.124.64 & 2787;4    & 5;1 \\
        \bottomrule
    \end{longtable}
\end{small}

\subsection{TODO}
\begin{itemize}
    \item Compressed mode for DNS records
\end{itemize}

\end{document}
