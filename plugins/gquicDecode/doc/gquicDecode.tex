\IfFileExists{t2doc.cls}{
    \documentclass[documentation]{subfiles}
}{
    \errmessage{Error: could not find 't2doc.cls'}
}

\begin{document}

\trantitle
    {gquicDecode} % Plugin name
    {Google Quick UDP Internet Connections (GQUIC)} % Short description
    {Tranalyzer Development Team} % author(s)

\section{gquicDecode}\label{s:gquicDecode}

\subsection{Description}
The gquicDecode plugin analyzes GQUIC traffic.

%\subsection{Dependencies}

%\traninput{file} % use this command to input files
%\traninclude{file} % use this command to include files

%\tranimg{image} % use this command to include an image (must be located in a subfolder ./img/)

%\subsubsection{External Libraries}
%This plugin depends on the {\bf XXX} library.
%\paragraph{Ubuntu:} {\tt sudo apt-get install XXX}
%\paragraph{Arch:} {\tt sudo pacman -S XXX}
%
%\subsubsection{Other Plugins}
%This plugin requires the {\bf XXX} plugin.
%
%\subsubsection{Required Files}
%The file {\tt file.txt} is required.

\subsection{Configuration Flags}
The following flags can be used to control the output of the plugin:
\begin{longtable}{>{\tt}lcl}
    \toprule
    {\bf Name} & {\bf Default} & {\bf Description}\\
    \midrule\endhead%
    GQUIC\_SLEN  & 63  & Max length for string columns, e.g., {\tt gquicSNI}\\
    GQUIC\_DEBUG &  0  & 0: do not print any debug messages\\
                 &     & 1: print warnings about unhandled cases\\
                 &     & 2: + print regular info about decoding status\\
    \bottomrule
\end{longtable}

\subsection{Flow File Output}
The gquicDecode plugin outputs the following columns:
\begin{longtable}{>{\tt}lll>{\tt\small}l}
    \toprule
    {\bf Column} & {\bf Type} & {\bf Description} & {\bf Flags}\\
    \midrule\endhead%
    \nameref{gquicStat}       & H8  & Status                       & \\
    \nameref{gquicPubFlags}   & H8  & Public Flags                 & \\
    %\nameref{gquicPrivFlags}  & H8  & Private Flags                & \\
    \nameref{gquicFrameTypes} & H16 & Frame Types                  & \\
    gquicCID                  & U64 & Connection ID                & \\
    gquicSNI                  & S   & Server Name Indication (SNI) & \\
    gquicUAID                 & S   & Client's User Agent ID       & \\
    % VERSION
    % MAX STREAMS PER CONNECTION
    % AUTHENTICATED ENCRYPTION ALGORITHMS
    % PROOF DEMAND
    % KEY EXCHANGE ALGORITHMS
    % ESTIMATED INITIAL RTT
    \bottomrule
\end{longtable}

\subsubsection{gquicStat}\label{gquicStat}
The {\tt gquicStat} column is to be interpreted as follows:
\begin{longtable}{>{\tt}rl}
    \toprule
    {\bf gquicStat} & {\bf Description}\\
    \midrule\endhead%
    0x01 & Flow is GQUIC\\
    0x02 & Handshake (Stream number is 1)\\
    0x04 & Connection ID changed\\
    0x08 & ---\\
    \\
    0x10 & ---\\
    0x20 & ---\\
    0x40 & Packet was snapped (t2buf failed)\\
    0x80 & Packet was malformed, e.g., covert channel\\
    \bottomrule
\end{longtable}

\subsubsection{gquicPubFlags}\label{gquicPubFlags}
The {\tt gquicPubFlags} column is to be interpreted as follows:
\begin{longtable}{>{\tt}rl}
    \toprule
    {\bf gquicPubFlags} & {\bf Description}\\
    \midrule\endhead%
    0x01 & Header contains a GQUIC Version\\
    0x02 & Public Reset packet\\
    0x04 & 32 byte diversification nonce is present (version >= 33)\\
    0x08 & 8 byte Connection ID is present (version >= 33)\\
    0x0c & 8 byte Connection ID is present (version < 33)\\
    0x30 & Number of low-order bytes of the packet number\\
    0x40 & Reserved for multipath use\\
    0x80 & Reserved (MUST be 0)\\
    \bottomrule
\end{longtable}

%\subsubsection{gquicPrivFlags}\label{gquicPrivFlags}
%The {\tt gquicPrivFlags} column is to be interpreted as follows:
%\begin{longtable}{>{\tt}rl}
%    \toprule
%    {\bf gquicPrivFlags} & {\bf Description}\\
%    \midrule\endhead%
%    0x01 & Entropy\\
%    0x02 & FEC Group\\
%    0x04 & FEC\\
%    0xf8 & Reserved (MUST be 0)\\
%    \bottomrule
%\end{longtable}

\subsubsection{gquicFrameTypes}\label{gquicFrameTypes}
The {\tt gquicFrameTypes} column is to be interpreted as follows:
\begin{longtable}{>{\tt}r>{\tt}l}
    \toprule
    {\bf gquicFrameTypes} & {\bf Description}\\
    \midrule\endhead%
    $2^{0}$  (=0x0001) & PADDING\\
    $2^{1}$  (=0x0002) & RST\_STREAM\\
    $2^{2}$  (=0x0004) & CONNECTION\_CLOSE\\
    $2^{3}$  (=0x0008) & GOAWAY\\
    \\
    $2^{4}$  (=0x0010) & WINDOW\_UPDATE\\
    $2^{5}$  (=0x0020) & BLOCKED\\
    $2^{6}$  (=0x0040) & STOP\_WAITING\\
    $2^{7}$  (=0x0080) & PING\\\\

    \multicolumn{2}{l}{Special Frame Types:}\\\\

    $2^{14}$ (=0x4000) & ACK\\
    $2^{15}$ (=0x8000) & STREAM\\
    \bottomrule
\end{longtable}

\subsection{Packet File Output}
In packet mode ({\tt --s} option), the gquicDecode plugin outputs the following columns:
\begin{longtable}{>{\tt}lll>{\tt\small}l}
    \toprule
    {\bf Column} & {\bf Type} & {\bf Description} & {\bf Flags}\\
    \midrule\endhead%
    \nameref{gquicPubFlags}  & H8  & Public Flags  & \\
    gquicVersion             & SC  & Version       & \\
    gquicPktNo               & U64 & Packet Number & \\
    gquicCID                 & U64 & Connection ID & \\
    %\nameref{gquicPrivFlags} & H8  & Private Flags & \\
    %gquicSID                 & U64 & Stream ID     & \\
    \bottomrule
\end{longtable}

\subsection{Plugin Report Output}
The following information is reported:
\begin{itemize}
    \item Aggregated {\tt\nameref{gquicStat}}
    \item Number of GQUIC packets
    \item Number of GQUIC Client Hello packets
    \item Number of GQUIC Rejection packets
    \item Number of GQUIC Public Reset packets
\end{itemize}

\subsection{Known Bugs and Limitations}
\begin{itemize}
    \item The gquicDecode plugin assumes every UDP packet on port 80 or 443 is GQUIC\ldots
\end{itemize}

%\subsection{References}
%\begin{itemize}
%    \item \href{https://tools.ietf.org/html/rfcXXXX}{RFCXXXX}: Title
%    \item \url{https://www.iana.org/assignments/}
%\end{itemize}

\end{document}
