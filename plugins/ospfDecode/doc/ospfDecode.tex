\IfFileExists{t2doc.cls}{
    \documentclass[documentation]{subfiles}
}{
    \errmessage{Error: could not find t2doc.cls}
}

\begin{document}

\trantitle
    {ospfDecode}
    {Open Shortest Path First (OSPF)}
    {Tranalyzer Development Team}

\section{ospfDecode}\label{s:ospfDecode}

\subsection{Description}
This plugin analyzes OSPFv4/6 traffic and provides absolute and relative statistics to the {\tt PREFIX\_ospfStats.txt} file. In addition, the {\tt rospf} script extracts the areas, networks and netmasks, along with the routers and their interfaces (\refs{s:ospf-pp}).

\subsection{Configuration Flags}\label{s:ospf-of}
The following flags can be used to control the output of the plugin:
\begin{longtable}{>{\tt}lcl>{\tt\small}l}
    \toprule
    {\bf Name}          & {\bf Default} & {\bf Description}                                     & {\bf Flags}\\
    \midrule\endhead%
    OSPF\_OUTPUT\_HLO   &  1            & Output hello messages                                 & \\
    OSPF\_OUTPUT\_DBD   &  1            & Output database description messages (routing tables) & \\
    OSPF\_OUTPUT\_MSG   &  1            & Output all other messages                             & \\
    OSPF\_OUTPUT\_STATS &  1            & Output statistics file                                & \\
    OSPF\_MASK\_AS\_IP  &  1            & Netmasks representation: 0: hex, 1: IPv4              & \\
    OSPF\_AREA\_AS\_IP  &  0            & Areas representation: 0: int, 1: IPv4, 2: hex         & \\
    OSPF\_LSID\_AS\_IP  &  0            & Link State ID representation:: 0: int, 1: IPv4        & \\
    OSPF\_TYP\_STR      &  1            & Message type representation: 0: hex, 1: string        & \\
    OSPF\_LSTYP\_STR    &  1            & LS type representation: 0: int, 1: string             & \\
    OSPF\_NEIGMAX       & 10            & Maximum neighbors to store                            & \\
    OSPF\_NUMTYP        & 10            & Maximum number of LS types to store                   & OSPF\_TYP\_STR=1\\
    \bottomrule
\end{longtable}

In addition, the suffix for the output files can be controlled with the following flags:

\begin{longtable}{>{\tt}l>{\tt\small}cl}
    \toprule
    {\bf Name}          & {\bf Default}     & {\bf Description}\\
    \midrule\endhead%
    OSPF\_SUFFIX        & "\_ospfStats.txt" & Statistics\\
    OSPF\_HELLO\_SUFFIX & "\_ospfHello.txt" & OSPFv2/3 hello messages\\
    OSPF\_DBD\_SUFFIX   & "\_ospfDBD.txt"   & OSPFv2/3 database description (routing tables)\\
    OSPF2\_MSG\_SUFFIX  & "\_ospf2Msg.txt"  & All other messages from OSPFv2 (Link State Request/Update/Ack)\\
    OSPF3\_MSG\_SUFFIX  & "\_ospf3Msg.txt"  & All other messages from OSPFv3 (Link State Request/Update/Ack)\\
    \bottomrule
\end{longtable}

\subsubsection{Environment Variable Configuration Flags}
The following configuration flags can also be configured with environment variables ({\tt ENVCNTRL>0}):
\begin{itemize}
    \item {\tt OSPF\_SUFFIX}
    \item {\tt OSPF\_HELLO\_SUFFIX}
    \item {\tt OSPF\_DBD\_SUFFIX}
    \item {\tt OSPF2\_MSG\_SUFFIX}
    \item {\tt OSPF3\_MSG\_SUFFIX}
\end{itemize}

\subsection{Flow File Output}
The ospfDecode plugin outputs the following columns:
\begin{longtable}{>{\tt}lll>{\tt\small}l}
    \toprule
    {\bf Column}          & {\bf Type}  & {\bf Description}                                    & {\bf Flags}\\
    \midrule\endhead%
    \nameref{ospfStat}    & H8          & Status                                               & \\
    ospfVersion           & H8          & Version                                              & \\
    \nameref{ospfType}    & H8/RS       & Message type                                         & \hyperref[s:ospf-of]{OSPF\_TYP\_STR=0/1} \\
    \nameref{ospfLSType}  & H64         & Update LS type                                       & \\
    \nameref{ospfAuType}  & H16         & Authentication type                                  & \\
    ospfAuPass            & RS          & Authentication password (if {\tt ospfAuType == 0x4}) & \\
    ospfArea              & U32/IP4/H32 & Area ID                                              & \hyperref[s:ospf-of]{OSPF\_AREA\_AS\_IP=0/1/2}\\
    ospfSrcRtr            & IP4         & Hello source router                                  & \\
    ospfBkupRtr           & IP4         & Hello backup router                                  & \\
    ospfNeighbors         & R(IP4)      & Hello neighbor router                                & \\
    \bottomrule
\end{longtable}

\subsubsection{ospfStat}\label{ospfStat}
The hex based status variable ({\tt ospfStat}) is defined as follows:
\begin{longtable}{>{\tt}rl}
    \toprule
    {\bf ospfStat} & {\bf Description} \\
    \midrule\endhead%
    $2^0$ (=0x01)  & OSPF detected\\
    $2^1$ (=0x02)  & OSPFv2 message had invalid TTL ($\neq1$)\\
    $2^2$ (=0x04)  & OSPFv2 message had invalid destination\\
    $2^3$ (=0x08)  & OSPF message had invalid type\\
    \\
    $2^4$ (=0x10)  & OSPF unknown version\\
    $2^5$ (=0x20)  & ---\\ %OSPF message had invalid checksum\\
    $2^6$ (=0x40)  & ---\\
    $2^7$ (=0x80)  & OSPF message was malformed (snapped, covert channels?, ...)\\
    \bottomrule
\end{longtable}

The invalid checksum status {\tt 0x08} is currently not implemented.\\
The malformed status {\tt 0x10} is currently used to report cases such as possible covert channels, e.g., {\tt authfield} used when {\tt auType} was {\tt NULL}.\\

\subsubsection{ospfType}\label{ospfType}
The hex based message type variable {\tt ospfType} is defined as follows:
\begin{longtable}{>{\tt}rl}
    \toprule
    {\bf ospfType} & {\bf Description} \\
    \midrule\endhead%
    $2^0$ (=0x01)  & Not valid\\
    $2^1$ (=0x02)  & Hello\\
    $2^2$ (=0x04)  & Database Description\\
    $2^3$ (=0x08)  & Link State Request\\
    \\
    $2^4$ (=0x10)  & Link State Update\\
    $2^5$ (=0x20)  & Link State Acknowledgement\\
    $2^6$ (=0x40)  & ---\\
    $2^7$ (=0x80)  & ---\\
    \bottomrule
\end{longtable}

\subsubsection{ospfLSType}\label{ospfLSType}
The hex based message type variable {\tt ospfLSType} is defined as follows:
\begin{longtable}{>{\tt}rl}
    \toprule
    {\bf ospfLSType}                  & {\bf Description} \\
    \midrule\endhead%
    $2^{0}$  (=0x0000 0000 0000 0001) & Reserved\\
    $2^{1}$  (=0x0000 0000 0000 0002) & OSPFv2/3 Router-LSA\\
    $2^{2}$  (=0x0000 0000 0000 0004) & OSPFv2/3 Network-LSA\\
    $2^{3}$  (=0x0000 0000 0000 0008) & OSPFv2 Summary-LSA (IP network)\\
                                      & OSPFv3 Inter-Area-Prefix-LSA\\
    \\
    $2^{4}$  (=0x0000 0000 0000 0010) & OSPFv2 Summary-LSA (ASBR)\\
                                      & OSPFv3 Inter-Area-Router-LSA\\
    $2^{5}$  (=0x0000 0000 0000 0020) & OSPFv2/3 AS-External-LSA\\
    $2^{6}$  (=0x0000 0000 0000 0040) & OSPFv2 Multicast group LSA (not implemented by Cisco)\\
                                      & Deprecated in OSPFv3\\
    $2^{7}$  (=0x0000 0000 0000 0080) & OSPFv2 Not-so-stubby area (NSSA) External LSA\\
                                      & OSPFv3 NSSA-LSA\\
    \\
    $2^{8}$  (=0x0000 0000 0000 0100) & OSPFv2 External attribute LSA for BGP\\
                                      & OSPFv3 Link-LSA\\
    $2^{9}$  (=0x0000 0000 0000 0200) & OSPFv2 Opaque LSA: Link-local scope\\
                                      & OSPFv3 Intra-Area-Prefix-LSA\\
    $2^{10}$ (=0x0000 0000 0000 0400) & OSPFv2 Opaque LSA: Area-local scope\\
                                      & OSPFv3 Intra-Area-TE-LSA\\
    $2^{11}$ (=0x0000 0000 0000 0800) & OSPFv2 Opaque LSA: autonomous system scope\\
                                      & OSPFv3 GRACE-LSA\\
    \\
    $2^{12}$ (=0x0000 0000 0000 1000) & OSPFv3 Router Information (RI)\\
    $2^{13}$ (=0x0000 0000 0000 2000) & OSPFv3 Inter-AS-TE-v3 LSA\\
    $2^{14}$ (=0x0000 0000 0000 4000) & OSPFv3 L1VPN LS\\
    $2^{15}$ (=0x0000 0000 0000 8000) & OSPFv3 Autoconfiguration (AC) LSA\\
    \\
    $2^{16}$ (=0x0000 0000 0001 0000) & OSPFv3 Dynamic Flooding LSA\\
    \\
    \multicolumn{2}{l}{$2^{17}$--$2^{32}$ are unassigned}\\
    \\
    $2^{33}$ (=0x0000 0002 0000 0000) & OSPFv3 E-Router-LSA\\
    $2^{34}$ (=0x0000 0004 0000 0000) & OSPFv3 E-Network-LSA\\
    $2^{35}$ (=0x0000 0008 0000 0000) & OSPFv3 E-Inter-Area-Prefix-LSA\\
    \\
    $2^{36}$ (=0x0000 0010 0000 0000) & OSPFv3 E-Inter-Area-Router-LSA\\
    $2^{37}$ (=0x0000 0020 0000 0000) & OSPFv3 E-AS-External-LSA\\
    $2^{38}$ (=0x0000 0040 0000 0000) & Unused (not to be allocated)\\
    $2^{39}$ (=0x0000 0080 0000 0000) & OSPFv3 E-Type-7-LSA\\
    \\
    $2^{40}$ (=0x0000 0100 0000 0000) & OSPFv3 E-Link-LSA\\
    $2^{41}$ (=0x0000 0200 0000 0000) & OSPFv3 E-Intra-Area-Prefix-LSA\\
    \bottomrule
\end{longtable}

\subsubsection{ospfAuType}\label{ospfAuType}
The hex based authentication type variable {\tt ospfAuType} is defined as follows:
\begin{longtable}{rl}
    \toprule
    {\bf ospfAuType}      & {\bf Description} \\
    \midrule\endhead%
    $2^1$ (={\tt 0x0002}) & Null authentication\\
    $2^2$ (={\tt 0x0004}) & Simple password\\
    $2^3$ (={\tt 0x0008}) & Cryptographic authentication\\
    \bottomrule
\end{longtable}

\subsection{Packet File Output}
In packet mode ({\tt --s} option), the ospfDecode plugin outputs the following columns:
\begin{longtable}{>{\tt}lll>{\tt\small}l}
    \toprule
    {\bf Column}       & {\bf Type}  & {\bf Description} & {\bf Flags}\\
    \midrule\endhead%
    \nameref{ospfStat} & H8          & Status            & \\
    ospfVersion        & U8          & Version           & \\
    ospfArea           & U32/IP4/H32 & Area ID           & \hyperref[s:ospf-of]{OSPF\_AREA\_AS\_IP=0/1/2}\\
    \nameref{ospfType} & S           & Message Type      & \\
    ospfLSType         & H64         & Update LS Type    & \\
    \bottomrule
\end{longtable}

\subsection{Plugin Report Output}
The following information is reported:
\begin{itemize}
    \item Aggregated {\tt\nameref{ospfStat}}
    \item Aggregated {\tt\nameref{ospfType}} for OSPFv2 and OSPFv3
    \item Number of OSPFv2 packets
    \item Number of OSPFv3 packets
\end{itemize}

\subsection{Additional Output}
\begin{itemize}
    \item {\tt PREFIX\_ospfStats.txt:} global statistics about OSPF traffic
    \item {\tt PREFIX\_ospfHello.txt}  Hello messages (see \refs{s:ospf-pp})
    \item {\tt PREFIX\_ospfDBD.txt:}   Routing tables (see {\tt OSPF\_OUTPUT\_DBD} in \refs{s:ospf-of})
    \item {\tt PREFIX\_ospf2Msg.txt:}  All other messages from OSPFv2 (see {\tt OSPF\_OUTPUT\_MSG} in \refs{s:ospf-of})
    \item {\tt PREFIX\_ospf3Msg.txt:}  All other messages from OSPFv3 (see {\tt OSPF\_OUTPUT\_MSG} in \refs{s:ospf-of})
\end{itemize}

\subsection{Post-Processing}\label{s:ospf-pp}

\subsubsection{rospf}
Hello messages can be used to discover the network topology and are stored in the {\tt PREFIX\_ospfHello.txt} file.
The script {\tt rospf} extracts the areas, networks, netmasks, routers and their interfaces:
\begin{center}
    {\tt ./scripts/rospf PREFIX\_ospfHello.txt}
\end{center}

\begin{figure}[!ht]
\centering
%\begin{tabular}{c}
\begin{lstlisting}
Name    Area    Network          Netmask
N1      0       192.168.21.0     0xffffff00
N2      1       192.168.16.0     0xffffff00
N3      1       192.168.22.0     0xfffffffc
...

Router    Interface_n      Network_n
R1        192.168.22.29    N11    192.168.21.4    N5    192.168.22.25    N10
R2        192.168.22.5     N12    192.168.16.1    N0    192.168.22.1     N6
R3        192.168.22.10    N13    192.168.21.2    N5    192.168.22.6     N12
...

Router    Connected Routers
R0        R2    R4    R6    R7    R8
R1        R2    R4
R2        R0    R1    R4    R8
...
\end{lstlisting}
%\end{tabular}
\end{figure}

\subsubsection{dbd}
If {\tt OSPF\_OUTPUT\_DBD} is activated (\refs{s:ospf-of}), database description messages are stored in a file {\tt PREFIX\_ospfDBD.txt}.
The {\tt dbd} script formats this file to produce an output similar to that of standard routers:
\begin{center}
    {\tt ./scripts/dbd PREFIX\_ospfDBD.txt}
\end{center}

\begin{lstlisting}
OSPF Router with ID (192.168.22.10)

Router Link States (Area 1)

Link ID          ADV Router       Age    Seq#          Checksum
192.168.22.5     192.168.22.5     4      0x80000002    0x38ce
192.168.22.10    192.168.22.10    837    0x80000002    0x6b0f
192.168.22.9     192.168.22.9     837    0x80000002    0x156c

Net Link States (Area 1)

Link ID         ADV Router       Age    Seq#          Checksum
192.168.22.6    192.168.22.10    4      0x80000001    0x150b
192.168.22.9    192.168.22.9     838    0x80000001    0x39e0

Summary Net Link States (Area 1)

Link ID         ADV Router       Age    Seq#           Checksum
192.168.17.0    192.168.22.9     735    0x80000001     0x5dd9
192.168.17.0    192.168.22.10    736    0x80000001     0x57de
192.168.18.0    192.168.22.9     715    0x80000001     0x52e3
...
\end{lstlisting}

\end{document}
