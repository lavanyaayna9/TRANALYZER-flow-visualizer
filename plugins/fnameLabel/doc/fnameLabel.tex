\IfFileExists{t2doc.cls}{
    \documentclass[documentation]{subfiles}
}{
    \errmessage{Error: could not find 't2doc.cls'}
}

\begin{document}

\trantitle
    {fnameLabel} % Plugin name
    {Filename Labeller} % Short description
    {Tranalyzer Development Team} % author(s)

\section{fnameLabel}\label{s:fnameLabel}

\subsection{Description}
The fnameLabel plugin tags every flow with the name of the file or interface from which the flow originates.
Moreover, it adds a hash value or a label which represents the number contained in a file or a specific letter.
It is predominantly used to automatically separate flows or packets created by the {\tt --R} or {\tt --D} option.
It can be used, e.g., for training classifiers.

\subsection{Configuration Flags}
The following flags can be used to control the output of the plugin:
\begin{longtable}{>{\tt}lcl>{\tt\small}l}
    \toprule
    {\bf Name} & {\bf Default} & {\bf Description} & {\bf Flags}\\
    \midrule\endhead%
    FNL\_LBL     &    1 & 1: Output label derived from input                                                          & \\
                 &      & (Use {\tt fileNum} for Tranalyzer {\tt --D} option, otherwise, refer to {\tt FNL\_IDX})     & \\
    FNL\_IDX     &    1 & Use the {\tt FNL\_IDX} letter of the filename as label                                      & FBL\_LBL=1\\
                 &      & (Tranalyzer {\tt -R}/{\tt --i}/{\tt --r} options)                                           & \\
    FNL\_HASH    &    0 & 1: Output hash of filename                                                                  & \\
    FNL\_FLNM    &    1 & 1: Output filename                                                                          & \\
    FNL\_FREL    &    1 & Use absolute (0) or relative (1) filenames for {\tt fnLabel}, {\tt fnHash} and {\tt fnName} & \\
    FNL\_NAMELEN & 1024 & Max length for filename                                                                     & \\
    \bottomrule
\end{longtable}

\subsection{Flow File Output}
The fnameLabel plugin outputs the following columns:
\begin{longtable}{>{\tt}lll>{\tt\small}l}
    \toprule
    {\bf Column} & {\bf Type} & {\bf Description} & {\bf Flags}\\
    \midrule\endhead%
    fnLabel & U32 & {\tt FNL\_IDX} letter of the filename/interface & FNL\_LBL=1\\
    fnHash  & U64 & Hash of the filename/interface                  & FNL\_HASH=1\\
    fnName  & S   & Filename                                        & FNL\_FLNM=1\\
    \bottomrule
\end{longtable}
Note that the filename refers to the file in which the flow was created.

\subsection{Packet File Output}
In packet mode ({\tt --s} option), the fnameLabel plugin outputs the same columns as in the flow file.

\end{document}
