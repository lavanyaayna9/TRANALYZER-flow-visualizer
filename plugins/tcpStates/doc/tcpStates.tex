\IfFileExists{t2doc.cls}{
    \documentclass[documentation]{subfiles}
}{
    \documentclass[a4paper,10pt]{article}
}

\begin{document}

\trantitle
    {tcpStates}
    {TCP Connection Tracker}
    {Tranalyzer Development Team} % author(s)

\section{tcpStates}\label{s:tcpStates}

\subsection{Description}
The tcpStates plugin tracks the actual state of a TCP connection, by analyzing the flags set in the packet header. The plugin recognizes and reports non-compliant behavior.

\subsection{Configuration Flags}
None.
%The following flags can be used to control the output of the plugin:
%\begin{longtable}{ccl}
%    \toprule
%    {\bf Name} & {\bf Default} & {\bf Description} \\
%    \midrule\endhead%
%    {\tt TCPSTATES\_PRINT} & 1 & Whether (1) or not (0) to output the state\\
%    \bottomrule
%\end{longtable}

\subsection{Flow File Output}
The tcpStates plugin outputs the following columns:
\begin{longtable}{>{\tt}lll>{\tt\small}l}
    \toprule
    {\bf Column} & {\bf Type} & {\bf Description} & {\bf Flags}\\
    \midrule\endhead%
    \nameref{tcpStatesAFlags} & H8 & TCP state machine anomalies & \\
    \bottomrule
\end{longtable}

\subsubsection{tcpStatesAFlags}\label{tcpStatesAFlags}
The {\tt tcpStatesAFlags} column is to be interpreted as follows:
\begin{longtable}{>{\tt}rl}
    \toprule
    {\bf tcpStatesAFlags} & {\bf Description} \\
    \midrule\endhead%
    0x01 & Malformed connection establishment \\
    0x02 & Malformed teardown \\
    0x04 & Malformed flags during established connection \\
    0x08 & Packets detected after teardown \\
    \\
    0x10 & Packets detected after reset \\
    0x20 & ---\\
    0x40 & Reset from sender \\
    0x80 & Potential evil behavior (scan)\\
    \bottomrule
\end{longtable}

\subsubsection{Flow Timeouts}
The tcpStates plugin also changes the timeout values of a flow according to its recognized state:
\begin{longtable}{llr}
    \toprule
    {\bf State} & {\bf Description}                            & {\bf Timeout (seconds)} \\
    \midrule\endhead%
    New         & Three way handshake is encountered           & 120 \\
    Established & Connection established                       & 610 \\
    Closing     & Hosts are about to close the connection      & 120 \\
    Closed      & Connection closed                            & 10 \\
    Reset       & Connection reset encountered by one of hosts & 0.1 \\
    \bottomrule
\end{longtable}

\subsubsection{Differences to the Host TCP State Machines}
The plugin state machine (\reff{fig:tcpstates}) and the state machines usually implemented in hosts differ in some cases. Major differences are caused by the benevolence of the plugin. For example, if a connection has not been established in a correct way, the plugin treats the connection as established, but sets the {\em malformed connection establishment} flag. The reasons for this benevolence are the following:
\begin{itemize}
    \item A flow might have been started before invocation of Tranalyzer2.
    \item A flow did not finish before Tranalyzer2 terminated.
    \item Tranalyzer2 did not detect every packet of a connection, for example due to a router misconfiguration.
    \item Flows from malicious programs may show suspicious behavior.
    \item Packets may be lost {\bf after} being captured by Tranalyzer2 but {\bf before} they reached the opposite host.
\end{itemize}

\begin{figure}[!ht]
    \centering
    \tranimg[height=0.8\textheight]{tcp_states_plugin.png}
    \caption{State machine of the tcpState plugin}
    \label{fig:tcpstates}
\end{figure}

\subsection{Packet File Output}
In packet mode ({\tt --s} option), the tcpStates plugin outputs the following columns:
\begin{longtable}{>{\tt}lll>{\tt\small}l}
    \toprule
    {\bf Column} & {\bf Type} & {\bf Description} & {\bf Flags}\\
    \midrule\endhead%
    \nameref{tcpStatesAFlags} & H8 & TCP state machine anomalies & \\
    \bottomrule
\end{longtable}

\subsection{Plugin Report Output}
The aggregated \nameref{tcpStatesAFlags} anomalies is reported.

\end{document}
