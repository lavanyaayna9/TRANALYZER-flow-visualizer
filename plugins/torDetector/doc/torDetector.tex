\IfFileExists{t2doc.cls}{
    \documentclass[documentation]{subfiles}
}{
    \errmessage{Error: could not find t2doc.cls}
}

\overfullrule=0pt
\begin{document}

\trantitle
    {torDetector}
    {Detect Tor flows}
    {Tranalyzer Development Team}

\setlength{\parindent}{0cm}

\section{torDetector}\label{s:torDetector}

\subsection{Description}
This plugin detects Tor flows.

\subsection{Dependencies}
This plugin requires the {\bf libssl}.
\begin{table}[!ht]
    \centering
    \begin{tabular}{>{\bf}r>{\tt}l>{\tt}l>{\tt}l}
        \toprule
        %                             &                      &                 \\
        %\midrule
        Ubuntu:                      & sudo apt-get install & libssl-dev      \\
        Arch:                        & sudo pacman -S       & openssl         \\
        %Gentoo:                      & sudo emerge          & openssl         \\
        openSUSE:                    & sudo zypper install  & libopenssl-devel\\
        Red Hat/Fedora\tablefootnote{If the {\tt dnf} command could not be found, try with {\tt yum} instead}:
                                     & sudo dnf install     & openssl-devel   \\
        macOS\tablefootnote{Brew is a packet manager for macOS that can be found here: \url{https://brew.sh}}:
                                     & brew install         & openssl@1.1     \\
        \bottomrule
    \end{tabular}
\end{table}

\subsection{Configuration Flags}
\begin{longtable}{>{\tt}lcl}
    \toprule
    {\bf Name} & {\bf Default} & {\bf Description} \\
    \midrule\endhead%
    TOR\_DETECT\_OBFUSCATION & 1 & Detect obfuscation protocols\\
    TOR\_DEBUG\_MESSAGES     & 0 & Activate debug output\\
    TOR\_PKTL                & 1 & Activate packet length modulo 8 heuristic\\
    \bottomrule
\end{longtable}

The obfuscation detection method has a pretty high rate of false positives when only one direction
of the traffic is captured. It should therefore be disabled (or low confidence should be given to
the resulting output) when analyzing Tor traffic which was only captured in one direction.

\subsection{Flow File Output}
The torDetector plugin outputs the following columns:
\begin{longtable}{>{\tt}lll}%>{\tt\small}l}
    \toprule
    {\bf Column} & {\bf Type} & {\bf Description}\\% & {\bf Flags}\\
    \midrule\endhead%
    \nameref{torStat} & H8 & Tor status\\
    \bottomrule
\end{longtable}

\subsubsection{torStat}\label{torStat}
The {\tt torStat} column is to be interpreted as follows:
\begin{longtable}{>{\tt}rl}
    \toprule
    {\bf torStat} & {\bf Description}\\
    \midrule\endhead%
    0x0\textcolor{magenta}{1} & Tor flow\\
    0x0\textcolor{magenta}{2} & Obfuscated Tor flow ({\tt TOR\_DETECT\_OBFUSCATION=1})\\
    0x0\textcolor{magenta}{4} & Tor address detected\\
    0x0\textcolor{magenta}{8} & Tor pktlen modulo 8 detected\\
    \\
    0x\textcolor{magenta}{1}0 & Internal state: SYN detected\\
    0x\textcolor{magenta}{2}0 & Internal state: obfuscation checked\\
    0x\textcolor{magenta}{4}0 & ---\\
    0x\textcolor{magenta}{8}0 & Packet snapped or decoding failed\\
    \bottomrule
\end{longtable}

\subsection{Packet File Output}
In packet mode ({\tt --s} option), the torDetector plugin outputs the following columns:
\begin{longtable}{>{\tt}lll>{\tt\small}l}
    \toprule
    {\bf Column} & {\bf Type} & {\bf Description} & {\bf Flags}\\
    \midrule\endhead%
    \nameref{torStat} & H8 & Status & \\
    \bottomrule
\end{longtable}

\subsection{Plugin Report Output}
The following information is reported:
\begin{itemize}
    \item Aggregated {\tt\nameref{torStat}}
    \item Number of Tor packets
\end{itemize}

\subsection{Plugin Detection Method}

This subsection briefly present the detection methods used by this plugin.
Tor version 0.2.7.6 and 0.4.5.10 were used to test these detection methods, they might become invalid in the future.

\subsubsection{TLS Certificate}

On older versions of the Tor client (< 0.2.9.15) the TLS certificate can be extracted from the traffic.
The following conditions are used to determine if it belongs to a Tor flow:

\begin{description}
    \item[Size of the certificate] Tor certificates are very minimalist and are always smaller than
        500 bytes.
    \item[Public key algorithm] Currently Tor uses RSA-1024 certificates. Proposal 220,
        \url{https://gitweb.torproject.org/torspec.git/plain/proposals/220-ecc-id-keys.txt}, defines
        how to support Ed25519 in addition to RSA-1024. This proposal was implemented in Tor 0.2.7.5
        according to the changelog:
        \url{https://gitweb.torproject.org/tor.git/tree/ReleaseNotes?h=release-0.2.8#n96}.
        However tests with Tor 0.2.7.6 (client and entry node) showed that RSA-1024 was still used for
        the TLS link key.
    \item[Validity period]\hfill
        \begin{description}
            \item[before 0.2.4.11] Tor certificate are always valid for exactly one year (60*60*24*365
                seconds).
            \item[0.2.4.11 and after] In March 2013, Tor changed the way it generates the validity period
                in certificates. Certificates are valid for a random period of time but validity periods
                always start at exactly midnight.
        \end{description}
    \item[Certificate issuer] The certificate issuer is only defined by its common name (no organization,
        country, \ldots). This common name has the following format: {\tt www.RAND.com} where {\tt RAND}
        is between 8 and 20 (inclusive) base32 characters. If Tor is compiled with the
        {\tt DISABLE\_V3\_LINKPROTO\_SERVERSIDE} flag, the common name ends in {\tt .net} instead of
        {\tt .com}
    \item[Certificate subject] The certificate subject is only defined by its common name
        (no organization, country, \ldots). This common name has the following format: {\tt www.RAND.net}
        where {\tt RAND} is between 8 and 20 (inclusive) base32 characters.
\end{description}

\subsubsection{TLS Client Hello}

Recent version of the Tor client (>= 0.2.9.15) use TLS 1.3 which encrypts the certificate.
On these versions, only the TLS handshake Client Hello and Server Hello can be used.

\begin{description}
    \item[Cipher list] The {\tt TLS\_EMPTY\_RENEGOTIATION\_INFO\_SCSV} is always the last element in the supported cipher list sent by Tor.
        Recent versions of Firefox and Chrome do not send this cipher.
    \item[TLS extensions] The {\tt renegotiation\_info}, the {\tt Application-Layer Protocol Negotiation (ALPN)} and the {\tt Next Protocol Negotiation (NPN)} extensions are never present in Tor Client Hello messages.
        These extensions are almost always present in modern browsers.
    \item[Server name extension] The {\tt server\_name} extension contains a hostname with the following format:
        {\tt www.RAND.com} where {\tt RAND} is between 4 and 25 (inclusive) base32 characters.
\end{description}

\subsubsection{TLS Server Hello}

The following fields are checked in the Server Hello to differentiate Tor from other TLS traffic:

\begin{description}
    \item[TLS extensions] The {\tt Application-Layer Protocol Negotiation (ALPN)} and the {\tt Next Protocol Negotiation (NPN)} extensions are never present in Tor Server Hello messages.
        These extensions are sometimes present in traffic sent by web servers.
\end{description}


\subsubsection{Obfuscation detection}

Different pluggable transport protocols can be used to obfuscate the Tor traffic between a client
and a bridge. For more info: \url{https://www.torproject.org/docs/pluggable-transports.html.en}.\\

This plugin tries to detect the {\bf obfs3} and {\bf obfs4} obfuscation methods. The goal of
these obfuscation methods is to make the traffic look completely random from the first byte
(including the (EC) Diffie-Hellman key exchange). This characteristic is used to detect flows with high
entropy in their first packets. Indeed, most encrypted protocols (TLS or SSH for instance) start with
an unencrypted handshake phase.

\subsection{Other Detection Methods}

This subsection describes other possible detection methods not implemented in this plugin.

\subsubsection{Relay blacklist}

The list of all relays can be queried from the Tor directory. This list contain the relay IP address and
OR port. Tor flows can easily be identified by comparing their IP addresses and ports against this list.
A CSV list can be downloaded from \url{https://torstatus.blutmagie.de/}.\\

The disadvantages of this method are:
\begin{itemize}
    \item It will not detect Bridge relays (\url{https://www.torproject.org/docs/bridges}) because
        it is not possible to query a list of Bridges from the Tor directory.
        Bridges addresses can only be retrieved three at a time from
        \url{https://bridges.torproject.org/} unless the bridge was specifically configured to not
        publish its descriptor.
    \item As this list is constantly changing, we need to regularly fetch it and keep all possible
        versions to use the version with the date closest to the analyzed PCAP.
\end{itemize}

The advantage of this method is:
\begin{itemize}
    \item The traffic payload is not necessary, this method only needs the flow ports and IP addresses.
\end{itemize}

\subsubsection{Packet size distribution}

Tor always pack data in cells of 512 bytes. This means that when a client or a relay only has a few
bytes to transmit, a packet with a size a bit superior to 512 bytes (because of TLS overhead)
will be sent. This results in a peak in the packet size distribution around 550 bytes
(543 bytes in the tests done with Tor version 0.2.7.6) and very few packets with smaller sizes.
Figures~\ref{fig:tor_https_distrib} and \ref{fig:tor_tor_distrib} show the difference in the packet
size distribution between HTTPS traffic and Tor traffic.\\

\begin{figure}[!ht]
  \centering
  \tranimg[width=0.9\textwidth]{https-distrib}
  \caption{Packet size distribution of HTTPS traffic.}
  \label{fig:tor_https_distrib}
\end{figure}

\begin{figure}[!ht]
  \centering
  \tranimg[width=0.9\textwidth]{tor-distrib}
  \caption{Packet size distribution of Tor traffic.}
  \label{fig:tor_tor_distrib}
\end{figure}

The disadvantage of this method is:
\begin{itemize}
    \item False positives and negatives are way more frequent than with certificate analysis.
\end{itemize}

The advantage of this method is:
\begin{itemize}
    \item The traffic payload is not necessary, this method only needs the packet size and
        the 5 (or 6 with VLAN) tuple necessary to aggregate packets in flows.
\end{itemize}

\clearpage

\end{document}
