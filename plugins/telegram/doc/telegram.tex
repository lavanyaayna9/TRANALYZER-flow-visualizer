\IfFileExists{t2doc.cls}{
    \documentclass[documentation]{subfiles}
}{
    \errmessage{Error: could not find 't2doc.cls'}
}

\begin{document}

\trantitle
    {telegram} % Plugin name
    {Telegram} % Short description
    {Tranalyzer Development Team} % author(s)

\section{telegram}\label{s:telegram}

\subsection{Description}
The telegram plugin detects Telegram UDP and TCP traffic using heuristics and
the IP labeling of basicFlow. Moreover it can extract L7 content of flows for
further deobfuscation experiments. It is a rudimentary version, lots to improve.

%\subsection{Dependencies}

\subsection{Configuration Flags}
The following flags can be used to control the output of the plugin:
\begin{longtable}{>{\tt}lcl}
    \toprule
    {\bf Name} & {\bf Default} & {\bf Description}\\
    \midrule\endhead%
    TG\_SAVE            & 0 & 1: save telegram flows\\
    TG\_DEOBFUSCATE     & 0 & 1: deobfuscate telegram flows\\
    TG\_4\_9\_OR\_NEWER & 1 & 1: reduce deobfuscation false positives for Telegram >= 4.9.0\\
                        &   & 0: needed when deobfuscating traffic generated by Telegram < 4.9.0\\
    TG\_DEBUG\_MESSAGES & 0 & 1: print debug messages\\
    \bottomrule
\end{longtable}

\subsection{Flow File Output}
The telegram plugin outputs the following columns:
\begin{longtable}{>{\tt}lll>{\tt\small}l}
    \toprule
    {\bf Column} & {\bf Type} & {\bf Description} & {\bf Flags}\\
    \midrule\endhead%
    \nameref{tgStat} & H16 & Status                & \\
    tgAuthKeyId      & H64 & Authentication key ID & TG\_DEOBFUSCATE!=0\\
    \bottomrule
\end{longtable}

\subsubsection{tgStat}\label{tgStat}
The {\tt tgStat} column is to be interpreted as follows:
\begin{longtable}{>{\tt}rl}
    \toprule
    {\bf tgStat} & {\bf Description}\\
    \midrule\endhead%
    $2^{0}$  (=0x0001) & Telegram detected by heuristics\\
    $2^{1}$  (=0x0002) & Control channel\\
    $2^{1}$  (=0x0004) & Voice\\
    $2^{3}$  (=0x0008) & Telegram detected by IP\\
    $2^{4}$  (=0x0010) & File save\\
    $2^{5}$  (=0x0020) & Bot app\\
    $2^{6}$  (=0x0040) & ---\\
    $2^{7}$  (=0x0080) & ---\\
    $2^{8}$  (=0x0100) & Write error\\
    $2^{9}$  (=0x0200) & ---\\
    $2^{10}$ (=0x0400) & ---\\
    $2^{11}$ (=0x0800) & ---\\
    $2^{12}$ (=0x1000) & Internal state machine \\
    $2^{13}$ (=0x2000) & Internal state machine \\
    $2^{14}$ (=0x4000) & Internal state machine \\
    $2^{15}$ (=0x8000) & Internal state machine init\\
    \bottomrule
\end{longtable}

\subsection{Packet File Output}
In packet mode ({\tt --s} option), the telegram plugin outputs the following columns:
\begin{longtable}{>{\tt}lll>{\tt\small}l}
    \toprule
    {\bf Column} & {\bf Type} & {\bf Description} & {\bf Flags}\\
    \midrule\endhead%
    \nameref{tgStat} & H16 & Status & \\
    \bottomrule
\end{longtable}

\subsection{Plugin Report Output}
The following information is reported:
\begin{itemize}
    \item Aggregated {\tt\nameref{tgStat}}
    \item Number of Telegram packets
\end{itemize}

\subsection{TODO}
\begin{itemize}
    \item Shift IP labeling to on flow generated in \tranrefpl{basicFlow}
\end{itemize}

\end{document}
