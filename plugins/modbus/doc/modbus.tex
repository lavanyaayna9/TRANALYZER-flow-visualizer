\IfFileExists{t2doc.cls}{
    \documentclass[documentation]{subfiles}
}{
    \errmessage{Error: could not find 't2doc.cls'}
}

\begin{document}

\trantitle
    {modbus} % Plugin name
    {Modbus} % Short description
    {Tranalyzer Development Team} % author(s)

\section{modbus}\label{s:modbus}

\subsection{Description}
The modbus plugin analyzes Modbus traffic.

\subsection{Configuration Flags}
The following flags can be used to control the output of the plugin:
\begin{longtable}{>{\tt}lcl}
    \toprule
    {\bf Name} & {\bf Default} & {\bf Description}\\
    \midrule\endhead%
    MB\_DEBUG      & 0 & Activate debug output\\
    \\
    MB\_FE\_FRMT   & 0 & Function/Exception codes representation: 0: hex, 1: int\\
    \\
    MB\_NUM\_FUNC  & 0 & Number of function codes to store (0 to hide \hyperref[modbusFC]{modbusFC})\\
    MB\_UNIQ\_FUNC & 0 & Aggregate multiply defined function codes\\
    \\
    MB\_NUM\_FEX   & 0 & Number of function codes causing exceptions to store (0 to hide \hyperref[modbusFEx]{modbusFEx})\\
    MB\_UNIQ\_FEX  & 0 & Aggregate multiply defined function codes causing exceptions\\
    \\
    MB\_NUM\_EX    & 0 & Number of exception codes to store (0 to hide \hyperref[modbusExC]{modbusExC})\\
    MB\_UNIQ\_EX   & 0 & Aggregate multiply defined exception codes\\
    \bottomrule
\end{longtable}

\subsection{Flow File Output}
The modbus plugin outputs the following columns:
\begin{longtable}{>{\tt}lll>{\tt\small}l}
    \toprule
    {\bf Column}                      & {\bf Type} & {\bf Description}                                 & {\bf Flags}\\
    \midrule\endhead%
    \nameref{modbusStat}              & H16        & Status                                            & \\
    modbusUID                         & U8         & Unit identifier                                   & \\
    modbusNPkts                       & U32        & Number of Modbus packets                          & \\
    modbusNumEx                       & U16        & Number of exceptions                              & \\
    \hyperref[modbusFC]{modbusFCBF}   & H64        & Aggregated function codes                         & \\
    \hyperref[modbusFC]{modbusFC}     & RH8        & List of function codes                            & MB\_NUM\_FUNC>0\\
    \hyperref[modbusFEx]{modbusFExBF} & H64        & Aggregated function codes which caused exceptions & \\
    \hyperref[modbusFEx]{modbusFEx}   & RH8        & List of function codes which caused exceptions    & MB\_NUM\_FEX>0\\
    \hyperref[modbusExC]{modbusExCBF} & H16        & Aggregated exception codes                        & \\
    \hyperref[modbusExC]{modbusExC}   & RH8        & List of exception codes                           & MB\_NUM\_EX>0\\
    \bottomrule
\end{longtable}

\clearpage

\subsubsection{modbusStat}\label{modbusStat}
The {\tt modbusStat} column is to be interpreted as follows:
\begin{longtable}{>{\tt}rl}
    \toprule
    {\bf modbusStat} & {\bf Description}\\
    \midrule\endhead%
    0x0001 & Flow is Modbus\\
    0x0002 & Non-modbus protocol identifier\\
    0x0004 & Unknown function code\\
    0x0008 & Unknown exception code\\
    \\
    0x0010 & Multiple unit identifiers\\
    0x0020 & ---\\
    0x0040 & ---\\
    0x0080 & ---\\
    \\
    0x0100 & List of function codes truncated\ldots increase {\tt MB\_NUM\_FUNC}\\
    0x0200 & List of function codes which caused exceptions truncated\ldots increase {\tt MB\_NUM\_FEX}\\
    0x0400 & List of exception codes truncated\ldots increase {\tt MB\_NUM\_EX}\\
    0x0800 & ---\\
    \\
    0x1000 & ---\\
    0x2000 & ---\\
    0x4000 & Snapped packet\\
    0x8000 & Malformed packet\\
    \bottomrule
\end{longtable}

\subsubsection{modbusFC and modbusFCBF}\label{modbusFC}
The {\tt modbusFC} and {\tt modbusFCBF} columns are to be interpreted as follows:
\begin{longtable}{>{\tt}r>{\tt}ll}
    \toprule
    {\bf modbusFC} & {\bf modbusFCBF} & {\bf Description}\\
    \midrule\endhead%
    % Data Access
     1 = 0x01 & 0x0000 0000 0000 0002 & Read Coils\\
     2 = 0x02 & 0x0000 0000 0000 0004 & Read Discrete Inputs\\
     3 = 0x03 & 0x0000 0000 0000 0008 & Read Multiple Holding Registers\\
     \\
     4 = 0x04 & 0x0000 0000 0000 0010 & Read Input Registers\\
     5 = 0x05 & 0x0000 0000 0000 0020 & Write Single Coil\\
     6 = 0x06 & 0x0000 0000 0000 0040 & Write Single Holding Register\\
     7 = 0x07 & 0x0000 0000 0000 0080 & Read Exception Status\\
     \\
     8 = 0x08 & 0x0000 0000 0000 0100 & Diagnostic\\
    11 = 0x0b & 0x0000 0000 0000 0800 & Get Com Event Counter\\
    \\
    12 = 0x0c & 0x0000 0000 0000 1000 & Get Com Event Log\\
    15 = 0x0f & 0x0000 0000 0000 8000 & Write Multiple Coils\\
    \\
    16 = 0x10 & 0x0000 0000 0001 0000 & Write Multiple Holding Registers\\
    17 = 0x11 & 0x0000 0000 0002 0000 & Report Slave ID\\
    \\
    20 = 0x14 & 0x0000 0000 0010 0000 & Read File Record\\
    21 = 0x15 & 0x0000 0000 0020 0000 & Write File Record\\
    22 = 0x16 & 0x0000 0000 0040 0000 & Mask Write Register\\
    23 = 0x17 & 0x0000 0000 0080 0000 & Read/Write Multiple Registers\\
    \\
    24 = 0x18 & 0x0000 0000 0100 0000 & Read FIFO Queue\\
    \\
    43 = 0x2b & 0x0000 0800 0000 0000 & Read Decide Identification\\
    \bottomrule
\end{longtable}

\subsubsection{modbusFEx and modbusFExBF}\label{modbusFEx}
The {\tt modbusFEx} and {\tt modbusFExBF} columns are to be interpreted as {\tt\hyperref[modbusFC]{modbusFC}} and {\tt\hyperref[modbusFC]{modbusFCBF}}, respectively.

\subsubsection{modbusExC and modbusExCBF}\label{modbusExC}
The {\tt modbusExC} and {\tt modbusExCBF} column are to be interpreted as follows:
\begin{longtable}{>{\tt}r>{\tt}rl}
    \toprule
    {\bf modbusExC} & {\bf modbusExCBF} & {\bf Description}\\
    \midrule\endhead%
     1 = 0x01 & 0x0002 & Illegal function code\\
     2 = 0x02 & 0x0004 & Illegal data address\\
     3 = 0x03 & 0x0008 & Illegal data value\\
     \\
     4 = 0x04 & 0x0010 & Slave device failure\\
     5 = 0x05 & 0x0020 & Acknowledge\\
     6 = 0x06 & 0x0040 & Slave device busy\\
     7 = 0x07 & 0x0080 & Negative acknowledge\\
     \\
     8 = 0x08 & 0x0100 & Memory parity error\\
    10 = 0x0a & 0x0400 & Gateway path unavailable\\
    11 = 0x0b & 0x0800 & Gateway target device failed to respond\\
    \bottomrule
\end{longtable}

\subsection{Packet File Output}
In packet mode ({\tt --s} option), the modbus plugin outputs the following columns:
\begin{longtable}{>{\tt}lll>{\tt\small}l}
    \toprule
    {\bf Column}         & {\bf Type} & {\bf Description} & {\bf Flags}\\
    \midrule\endhead%
    mbTranId             & U16        & Transaction Identifier & \\
    mbProtId             & U16        & Protocol Identifier    & \\
    mbLen                & U16        & Length                 & \\
    mbUnitId             & U8         & Unit identifier        & \\
    \nameref{mbFuncCode} & H8         & Function code          & MB\_FE\_FRMT=0\\
    \nameref{mbFuncCode} & U8         & Function code          & MB\_FE\_FRMT=1\\
    \bottomrule
\end{longtable}

\subsubsection{mbFuncCode}\label{mbFuncCode}
If {\tt mbFuncCode} column is to be interpreted as follows:
\begin{longtable}{>{\tt}ll}
    \toprule
    {\bf mbFuncCode} & {\bf Description}\\
    \midrule\endhead%
    $< 128$ (=0x80)    & refer to \nameref{modbusFC}\\
    $\geq 128$ (=0x80) & subtract 128 (={\tt 0x80}) and refer to \nameref{modbusFEx}\\
    \bottomrule
\end{longtable}

\subsection{Monitoring Output}
In monitoring mode, the modbus plugin outputs the following columns:
\begin{longtable}{>{\tt}lll>{\tt\small}l}
    \toprule
    {\bf Column} & {\bf Type} & {\bf Description} & {\bf Flags}\\
    \midrule\endhead%
    modbusNPkts          & U64 & Number of Modbus packets & \\
    \nameref{modbusStat} & H16 & Status                   & \\
    \bottomrule
\end{longtable}

\subsection{Plugin Report Output}
The following information is reported:
\begin{itemize}
    \item Aggregated {\tt\nameref{modbusStat}}
    \item Number of Modbus packets
\end{itemize}

\end{document}
