\IfFileExists{t2doc.cls}{
    \documentclass[documentation]{subfiles}
}{
    \errmessage{Error: could not find 't2doc.cls'}
}

\begin{document}

\trantitle
    {regexHyperscan} % Plugin name
    {Traffic pattern matching using the Hyperscan library.} % Short description
    {Tranalyzer Development Team} % author(s)

\section{regexHyperscan}\label{s:regexHyperscan}

\subsection{Description}
This plugin applies regexes on the network traffic using the
\href{https://github.com/intel/hyperscan}{Hyperscan library}%
\footnote{\url{https://github.com/intel/hyperscan}}.
The regexes can be applied on the whole flow or per packet from layer 7.

\subsection{Dependencies}

\subsubsection{External Libraries}

This plugin depends on the Hyperscan library which is included in this plugin.
In order to compile it, the following tools and libraries are needed.
\begin{table}[!ht]
    \centering
    \begin{tabular}{>{\bf}r>{\tt}l>{\tt}l}
        \toprule
        %                             &                      & {\bf OPT1=1}\\
        %\midrule
        Ubuntu:                      & sudo apt-get install & cmake g++ libboost-dev ragel\\
        Arch:                        & sudo pacman -S       & boost cmake ragel\\
        %Gentoo:                      & sudo emerge          & boost cmake ragel\\
        %openSUSE:                    & sudo zypper install  & boost-devel cmake gcc-c++ ragel\\
        Red Hat/Fedora\tablefootnote{If the {\tt dnf} command could not be found, try with {\tt yum} instead}:
                                     & sudo dnf install     & boost-devel cmake gcc-c++ ragel\\
        %macOS\tablefootnote{Brew is a packet manager for macOS that can be found here: \url{https://brew.sh}}:
        %                             & brew install         & boost cmake ragel\\
        \bottomrule
    \end{tabular}
\end{table}

\subsubsection{Required Files}
\label{s:hsRequiredFiles}
The file {\tt hsregexes.txt} contains the regexes and their corresponding ID.
The lines starting with {\tt \%} are comments. The other lines must contain two or three columns:
\begin{longtable}{cl}
    \toprule
    {\bf Column} & {\bf Description}\\
    \midrule\endhead%
    1 & A string ID which will appear in the \hyperref[s:hsFlowOutput]{flow output} if the flow matches
    the regex in column 2.\\
    2 & A regex in the Hyperscan format describe in \refs{s:hsRegFormat}.\\
    3 & Optional. Whether (1) to extract flows matching regex using the \tranrefpl{liveXtr} plugin, or not (0).\\
    \bottomrule
\end{longtable}

\subsection{Hyperscan regex format}
\label{s:hsRegFormat}

Each regex must have the following format: {\tt /pattern/flags}\\

The {\bf pattern} use the PCRE syntax with some limitation explained in the
\href{https://intel.github.io/hyperscan/dev-reference/compilation.html#pattern-support}{Hyperscan documentation}%
\footnote{\url{https://intel.github.io/hyperscan/dev-reference/compilation.html\#pattern-support}}.\\

The {\bf flags} are optional and are described in the
\href{https://intel.github.io/hyperscan/dev-reference/api_constants.html#pattern-flags}{Hyperscan documentation}%
\footnote{\url{https://intel.github.io/hyperscan/dev-reference/api_constants.html\#pattern-flags}}.
The following table provides a correspondence between the letters used in this plugin regex format
and the values in the Hyperscan documentation.

\begin{longtable}{>{\tt}cl}
    \toprule
    {\bf Flag} & {\bf Description}\\
    \midrule\endhead%
    i & {\tt HS\_FLAG\_CASELESS}\\
    s & {\tt HS\_FLAG\_DOTALL}\\
    m & {\tt HS\_FLAG\_MULTILINE}\\
    H & {\tt HS\_FLAG\_SINGLEMATCH} (enabled by default)\\
    V & {\tt HS\_FLAG\_ALLOWEMPTY}\\
    8 & {\tt HS\_FLAG\_UTF8}\\
    W & {\tt HS\_FLAG\_UCP}\\
    \bottomrule
\end{longtable}

\subsection{Configuration Flags}
The following flags can be used to control the output of the plugin:
\begin{longtable}{>{\tt}lcl}
    \toprule
    {\bf Name} & {\bf Default} & {\bf Description}\\
    \midrule\endhead%
    RHS\_STREAMING & 1 &
        \begin{tabular}{@{}l@{}}
        1: Apply the regexes on the whole flow as a stream.\\
        0: Apply the regexes per packet.\\
        \end{tabular}\\
    RHS\_RELOADING         & 1  & Automatically reload the regex file when modified.\\
    RHS\_EXTRACT\_OPPOSITE & 1  & Also extract the opposite flow when regex match.\\
    RHS\_MAX\_FLOW\_MATCH  & 16 & Max. number of regexes which can match on a flow.\\
    RHS\_REGEX\_FILE       & {\tt\small "hsregexes.txt"} & The name of the file described in \refs{s:hsRequiredFiles}.\\
    \bottomrule
\end{longtable}

\subsection{Flow File Output}
\label{s:hsFlowOutput}
The regexHyperscan plugin outputs the following columns:
\begin{longtable}{>{\tt}lll>{\tt\small}l}
    \toprule
    {\bf Column} & {\bf Type} & {\bf Description}\\
    \midrule\endhead%
    hsregexes & RS & IDs of all regexes matching this flow & \\
    \bottomrule
\end{longtable}

\end{document}
