\IfFileExists{t2doc.cls}{
    \documentclass[documentation]{subfiles}
}{
    \errmessage{Error: could not find t2doc.cls}
}

\begin{document}

\trantitle
    {httpSniffer}
    {HyperText Transfer Protocol (HTTP)}
    {Tranalyzer Development Team}

\section{httpSniffer}\label{s:httpSniffer}

\subsection{Description}
The httpSniffer plugin processes HTTP header and content information of a flow. The idea is to identify
certain HTTP features using flow parameters and to extract certain content such as text or
images for further investigation. The httpSniffer plugin requires no dependencies and produces
only output to the flow file. User defined compiler switches in {\em httpSniffer.h} produce
optimized code for the specific application.

\subsection{Configuration Flags}
The flow based output and the extracted information can be controlled by switches and constants listed in the table below.
They control the output of host, URL and method counts, names and cookies and the function of content storage.
{\bf WARNING:} The amount of being stored on disk can be substantial, make sure that the number of concurrent file handles
is large enough, use {\tt ulimit -n}.

\begin{longtable}{>{\tt}lcl>{\tt\small}l}
    \toprule
    {\bf Name} & {\bf Default} & {\bf Description} & {\bf Flags}\\
    \midrule\endhead%
    HTTP\_MIME           & 1 & Mime types                                      & \\
    HTTP\_STAT           & 1 & Status codes                                    & \\
    HTTP\_MCNT           & 1 & Mime count: GET, POST                           & \\
    HTTP\_HOST           & 1 & Hosts                                           & \\
    HTTP\_URL            & 1 & URLs                                            & \\
    HTTP\_COOKIE         & 1 & Cookies                                         & \\
    HTTP\_IMAGE          & 1 & Image names                                     & \\
    HTTP\_VIDEO          & 1 & Video names                                     & \\
    HTTP\_AUDIO          & 1 & Audio names                                     & \\
    HTTP\_MSG            & 1 & Message names                                   & \\
    HTTP\_APPL           & 1 & Application names                               & \\
    HTTP\_TEXT           & 1 & Text names                                      & \\
    HTTP\_PUNK           & 1 & POST/else/unknown names                         & \\
    HTTP\_BODY           & 1 & Analyze body and print anomalies                & \\
    HTTP\_BDURL          & 1 & Refresh and set-cookie URLs                     & HTTP\_BODY=1\\
    HTTP\_USRAG          & 1 & User-Agents                                     & \\
    HTTP\_XFRWD          & 1 & X-Forwarded-For                                 & \\
    HTTP\_REFRR          & 1 & Referer                                         & \\
    HTTP\_VIA            & 1 & Via                                             & \\
    HTTP\_LOC            & 1 & Location                                        & \\
    HTTP\_SERV           & 1 & Server                                          & \\
    HTTP\_PWR            & 1 & X-Powered-By                                    & \\
    \\
    HTTP\_ANTVIR         & 0 & Antivirus Info                                  & \\
    HTTP\_AVAST\_CID     & 0 & Avast client ID                                 & \\
    HTTP\_ESET\_UID      & 0 & ESET update ID                                  & \\
    \\
    HTTP\_STATAGA        & 1 & Aggregate status response                       & \\
    HTTP\_MIMEAGA        & 1 & Aggregate mime response                         & \\
    HTTP\_HOSTAGA        & 1 & Aggregate Hosts                                 & \\
    HTTP\_URLAGA         & 1 & Aggregate URLs                                  & \\
    HTTP\_USRAGA         & 1 & Aggregate User-Agents                           & \\
    HTTP\_XFRWDA         & 1 & Aggregate X-Forwarded-For                       & \\
    HTTP\_REFRRA         & 1 & Aggregate Referer                               & \\
    HTTP\_VIAA           & 1 & Aggregate Via                                   & \\
    HTTP\_LOCA           & 1 & Aggregate Location                              & \\
    HTTP\_SERVA          & 1 & Aggregate Server                                & \\
    HTTP\_PWRA           & 1 & Aggregate X-Powered-By                          & \\
    \\
    HTTP\_SAVE\_IMAGE    & 0 & Save all images                                 & \\
    HTTP\_SAVE\_VIDEO    & 0 & Save all videos                                 & \\
    HTTP\_SAVE\_AUDIO    & 0 & Save all audios                                 & \\
    HTTP\_SAVE\_MSG      & 0 & Save all messages                               & \\
    HTTP\_SAVE\_TEXT     & 0 & Save all texts                                  & \\
    HTTP\_SAVE\_APPL     & 0 & Save all applications                           & \\
    HTTP\_SAVE\_PUNK     & 0 & Save all else                                   & \\
    \\
    HTTP\_PUNK\_AV\_ONLY & 0 & Save PUT/else only for antivirus                & HTTP\_SAVE\_PUNK=1\\
    \\
    HTTP\_RMDIR          & 1 & Empty {\tt\small HTTP\_*\_PATH} before starting & HTTP\_SAVE=1\\
    \bottomrule
\end{longtable}

Note that {\tt HTTP\_SAVE\_*} refers to the {\em Content-Type}, e.g., {\tt HTTP\_SAVE\_APPL}, will save all payload whose {\em Content-Type} starts with {\tt application/} (including forms, such as {\tt application/x-www-form-urlencoded}).
The maximum memory allocation per item is defined by {\tt HTTP\_DATA\_C\_MAX} listed below.
The path of each extracted HTTP content can be set by the {\tt HTTP\_XXXX\_PATH} constants.
HTTP content having no name is assigned a default name defined by {\tt HTTP\_NONAME}. Each name is
prepended the findex, packet number and an index to facilitate the mapping between flows and its content.
The latter constant has to be chosen carefully because for each item (mime, cookie, image, \ldots),
{\tt HTTP\_MXFILE\_LEN * HTTP\_DATA\_C\_MAX * HASHCHAINTABLE\_SIZE * HASHFACTOR} bytes are allocated.

The filenames are defined as follows:
\begin{center}
    {\tt Filename\_findex\_Flow-Dir(A/B)\_\#Packet-in-Flow\_\#Mimetype-in-Flow}
\end{center}
So they can easily being matched with the flow or packet file.

\begin{longtable}{>{\tt}lcl}
    \toprule
    {\bf Name}          & {\bf Default}             & {\bf Description} \\
    \midrule\endhead%
    HTTP\_PATH          & {\tt\small "/tmp"}        & Root path for extracted content \\
    HTTP\_IMAGE\_PATH   & {\tt\small "httpPicture"} & Path for pictures \\
    HTTP\_VIDEO\_PATH   & {\tt\small "httpVideo"}   & Path for videos \\
    HTTP\_AUDIO\_PATH   & {\tt\small "httpAudio"}   & Path for audios \\
    HTTP\_MSG\_PATH     & {\tt\small "httpMSG"}     & Path for messages \\
    HTTP\_TEXT\_PATH    & {\tt\small "httpText"}    & Path for texts \\
    HTTP\_APPL\_PATH    & {\tt\small "httpAppl"}    & Path for applications \\
    HTTP\_PUNK\_PATH    & {\tt\small "httpPunk"}    & Path for PUT/else \\
    HTTP\_NONAME        & {\tt\small "nudel"}       & File name for unnamed content \\
    HTTP\_DATA\_C\_MAX  &  20                       & Maximum dim of all storage array: \# / flow \\
    HTTP\_CNT\_LEN      &  13                       & Max \# of cnt digits attached to file name \\
    HTTP\_FINDEX\_LEN   &  20                       & String length of findex in decimal format \\
    HTTP\_MXFILE\_LEN   &  80                       & Maximum image name length in bytes \\
    HTTP\_MXUA\_LEN     & 400                       & Maximum User-Agent name length in bytes \\
    HTTP\_MXXF\_LEN     &  80                       & Maximum X-Forward-For name length in bytes \\
    HTTP\_AVID\_LEN     &  32                       & Maximum antivirus client ID length in bytes \\
    \bottomrule
\end{longtable}

\subsubsection{Environment Variable Configuration Flags}
The following configuration flags can also be configured with environment variables ({\tt ENVCNTRL>0}):
\begin{itemize}
    \item {\tt HTTP\_RMDIR}
    \item {\tt HTTP\_PATH}
    \item {\tt HTTP\_IMAGE\_PATH}
    \item {\tt HTTP\_VIDEO\_PATH}
    \item {\tt HTTP\_AUDIO\_PATH}
    \item {\tt HTTP\_MSG\_PATH}
    \item {\tt HTTP\_TEXT\_PATH}
    \item {\tt HTTP\_APPL\_PATH}
    \item {\tt HTTP\_PUNK\_PATH}
    %\item {\tt HTTP\_NONAME}
\end{itemize}

\subsection{Flow File Output}
The httpSniffer plugin outputs the following columns:
\begin{longtable}{>{\tt}lll>{\tt\small}l}
    \toprule
    {\bf Column} & {\bf Type} & {\bf Description} & {\bf Flags}\\
    \midrule\endhead%
    \nameref{httpStat}        & H16   & Status                                            & \\
    \nameref{httpAFlags}      & H16   & Anomaly flags                                     & \\
    \nameref{httpMethods}     & H8    & HTTP methods                                      & \\
    \nameref{httpHeadMimes}   & H16   & HEADMIME-TYPES                                    & \\
    \nameref{httpCFlags}      & H8    & HTTP content body info                            & HTTP\_BODY=1\\
    httpGet\_Post             & 2U16  & Number of GET and POST requests                   & HTTP\_MCNT=1\\
    httpRSCnt                 & U16   & Response status count                             & HTTP\_STAT=1\\
    httpRSCode                & RU16  & Response status code                              & HTTP\_STAT=1\\
    httpURL\_Via\_Loc\_Srv\_  & 10U16 & Number of URL, Via, Location, Server,             & \\
    \qquad Pwr\_UAg\_XFr\_    &       & \qquad X-Powered-By, User-Agent, X-Forwarded-For, & \\
    \qquad Ref\_Cky\_Mim      &       & \qquad Referer, Cookie and Mime-Type              & \\
    httpImg\_Vid\_Aud\_Msg\_  & 7U16  & Number of images, videos, audios, messages,       & \\
    \qquad Txt\_App\_Unk      &       & \qquad texts, applications and unknown            & \\
    httpHosts                 & RS    & Host names                                        & HTTP\_HOST=1\\
    httpURL                   & RS    & URLs (including parameters)                       & HTTP\_URL=1\\
    httpMimes                 & RS    & MIME-types                                        & HTTP\_MIME=1\\
    httpCookies               & RS    & Cookies                                           & HTTP\_COOKIE=1\\
    httpImages                & RS    & Images                                            & HTTP\_IMAGE=1\\
    httpVideos                & RS    & Videos                                            & HTTP\_VIDEO=1\\
    httpAudios                & RS    & Audios                                            & HTTP\_AUDIO=1\\
    httpMsgs                  & RS    & Messages                                          & HTTP\_MSG=1\\
    httpAppl                  & RS    & Applications                                      & HTTP\_APPL=1\\
    httpText                  & RS    & Texts                                             & HTTP\_TEXT=1\\
    httpPunk                  & RS    & Payload unknown                                   & HTTP\_PUNK=1\\
    httpBdyURL                & RS    & Body: Refresh, set\_cookie URL                    & HTTP\_BODY=1\&\&\\
                              &       &                                                   & HTTP\_BDURL=1\\
    httpUsrAg                 & RS    & User-Agent                                        & HTTP\_USRAG=1\\
    httpXFor                  & RS    & X-Forwarded-For                                   & HTTP\_XFRWD=1\\
    httpRefrr                 & RS    & Referer                                           & HTTP\_REFRR=1\\
    httpVia                   & RS    & Via (Proxy)                                       & HTTP\_VIA=1\\
    httpLoc                   & RS    & Location (Redirection)                            & HTTP\_LOC=1\\
    httpServ                  & RS    & Server                                            & HTTP\_SERV=1\\
    httpPwr                   & RS    & X-Powered-By / Application                        & HTTP\_PWR=1\\
    httpAvastCid              & S     & Avast client ID                                   & HTTP\_AVAST\_CID=1\\
    httpEsetUid               & S     & ESET update ID                                    & HTTP\_ESET\_UID=1\\
    \bottomrule
\end{longtable}

\subsubsection{httpStat}\label{httpStat}
The {\tt httpStat} column is to be interpreted as follows:
\begin{longtable}{>{\tt}rl}
    \toprule
    {\bf httpStat}     & {\bf Description}\\
    \midrule\endhead%
    $2^{0}$  (=0x0001) & Warning: {\tt HTTP\_DATA\_C\_MAX} entries in flow name array reached\\
    $2^{1}$  (=0x0002) & Warning: Filename longer than {\tt HTTP\_MXFILE\_LEN}\\
    $2^{2}$  (=0x0004) & Internal state: pending URL name\\
    $2^{3}$  (=0x0008) & HTTP flow\\
    \\
    $2^{4}$  (=0x0010) & Internal state: Chunked transfer\\
    $2^{5}$  (=0x0020) & Internal state: HTTP flow detected\\
    $2^{6}$  (=0x0040) & Internal state: HTTP header parsing in process\\
    $2^{7}$  (=0x0080) & Internal state: sequence number init\\
    \\
    $2^{8}$  (=0x0100) & Internal state: header shift\\
    $2^{9}$  (=0x0200) & Internal state: PUT payload sniffing\\
    $2^{10}$ (=0x0400) & Internal state: Image payload sniffing\\
    $2^{11}$ (=0x0800) & Internal state: video payload sniffing\\
    \\
    $2^{12}$ (=0x1000) & Internal state: audio payload sniffing\\
    $2^{13}$ (=0x2000) & Internal state: message payload sniffing\\
    $2^{14}$ (=0x4000) & Internal state: text payload sniffing\\
    $2^{15}$ (=0x8000) & Internal state: application payload sniffing\\
    \bottomrule
\end{longtable}

\subsubsection{httpAFlags}\label{httpAFlags}
The {\tt httpAFlags} column denotes HTTP anomalies regarding the protocol and the security.
It is to be interpreted as follows:
\begin{longtable}{>{\tt}rl}
    \toprule
    {\bf httpAFlags}   & {\bf Description}\\
    \midrule\endhead%
    $2^{0}$  (=0x0001) & Warning: POST query with parameters, possible malware \\
    $2^{1}$  (=0x0002) & Warning: Host is IPv4\\
    $2^{2}$  (=0x0004) & Warning: Possible DGA\\
    $2^{3}$  (=0x0008) & Warning: Mismatched content-type\\
    \\
    $2^{4}$  (=0x0010) & Warning: Sequence number mangled or error retry detected\\
    $2^{5}$  (=0x0020) & Warning: Parse Error\\
    $2^{6}$  (=0x0040) & Warning: header without value, e.g., {\tt Content-Type:\ [missing]}\\
    $2^{7}$  (=0x0080) & ---\\
    \\
    $2^{8}$  (=0x0100) & Info: X-Site Scripting protection\\
    $2^{9}$  (=0x0200) & Info: Content Security Policy\\
    $2^{10}$ (=0x0400) & Info: Do not track\\
    $2^{11}$ (=0x0800) & ---\\
    \\
    $2^{12}$ (=0x1000) & Warning: possible EXE download\\
    $2^{13}$ (=0x2000) & Warning: possible ELF download\\
    $2^{14}$ (=0x4000) & Warning: HTTP 1.0 legacy protocol, often used by malware\\
    $2^{15}$ (=0x8000) & ---\\
    \bottomrule
\end{longtable}

\subsubsection{httpMethods}\label{httpMethods}
The {\tt httpMethods} column is to be interpreted as follows:
\begin{longtable}{>{\tt}rll}
    \toprule
    {\bf httpMethods} & {\bf Type} & {\bf Description}\\
    \midrule\endhead%
    $2^0$ (=0x01)     & OPTIONS    & Return HTTP methods that server supports for specified URL\\
    $2^1$ (=0x02)     & GET        & Request of representation of specified resource\\
    $2^2$ (=0x04)     & HEAD       & Request of representation of specified resource without body\\
    $2^3$ (=0x08)     & POST       & Request to accept enclosed entity as new subordinate of resource identified by URI\\
    \\
    $2^4$ (=0x10)     & PUT        & Request to store enclosed entity under supplied URI\\
    $2^5$ (=0x20)     & DELETE     & Delete specified resource\\
    $2^6$ (=0x40)     & TRACE      & Echo back received request\\
    $2^7$ (=0x80)     & CONNECT    & Convert request connection to transparent TCP/IP tunnel \\
    \bottomrule
\end{longtable}

\subsubsection{httpHeadMimes}\label{httpHeadMimes}
The {\tt httpHeadMimes} column is to be interpreted as follows:
\begin{longtable}{>{\tt}rll}
    \toprule
    {\bf httpHeadMimes} & {\bf Mime-Type} & {\bf Description}\\
    \midrule\endhead%
    $2^{0}$  (=0x0001) & application & Multi-purpose files: java or postscript, \ldots\\
    $2^{1}$  (=0x0002) & audio       & Audio file\\
    $2^{2}$  (=0x0004) & image       & Image file\\
    $2^{3}$  (=0x0008) & message     & Instant or email message type\\
    \\
    $2^{4}$  (=0x0010) & model       & 3D computer graphics\\
    $2^{4}$  (=0x0020) & multipart   & Archives and other objects made of more than one part\\
    $2^{5}$  (=0x0040) & text        & Human-readable text and source code\\
    $2^{6}$  (=0x0080) & video       & Video stream: Mpeg, Flash, Quicktime, \ldots\\
    \\
    $2^{8}$  (=0x0100) & vnd         & Vendor-specific files: Word, OpenOffice, \ldots\\
    $2^{9}$  (=0x0200) & x           & Non-standard files: tar, SW packages, \LaTeX, Shockwave Flash, \ldots\\
    $2^{10}$ (=0x0400) & x-pkcs      & public-key cryptography standard files\\
    $2^{11}$ (=0x0800) & ---         & ---\\
    \\
    $2^{12}$ (=0x1000) & ---         & ---\\
    $2^{13}$ (=0x2000) & ---         & ---\\
    $2^{14}$ (=0x4000) & ---         & ---\\
    $2^{15}$ (=0x8000) & *           & All else\\
    \bottomrule
\end{longtable}

\subsubsection{httpCFlags}\label{httpCFlags}
The {\tt httpCFlags} column is to be interpreted as follows:
\begin{longtable}{>{\tt}rl}
    \toprule
    {\bf httpCFlags} & {\bf Description}\\
    \midrule\endhead%
    $2^{0}$  (=0x0001) & HTTP set cookie\\
    $2^{1}$  (=0x0002) & HTTP refresh detected\\
    $2^{2}$  (=0x0004) & Hostname detected\\
    $2^{3}$  (=0x0008) & POST Boundary marker\\
    \\
    $2^{4}$  (=0x0010) & Potential HTTP content\\
    $2^{5}$  (=0x0020) & Stream\\
    $2^{6}$  (=0x0040) & Quarantine virus upload\\
    $2^{7}$  (=0x0080) & Antivirus sample upload\\
    \\
    $2^{8}$  (=0x0100) & Antivirus Avira detected\\
    $2^{9}$  (=0x0200) & Antivirus Avast detected\\
    $2^{10}$ (=0x0400) & Antivirus AVG detected\\
    $2^{11}$ (=0x0800) & Antivirus Bit Defender detected\\
    \\
    $2^{12}$ (=0x1000) & Antivirus ESET detected\\
    $2^{13}$ (=0x2000) & Antivirus Microsoft sec detected\\
    $2^{14}$ (=0x4000) & Antivirus Symantec, Norton, \ldots\\
    $2^{15}$ (=0x8000) & Stream1\\
    \bottomrule
\end{longtable}

\subsection{Packet File Output}
In packet mode ({\tt --s} option), the httpSniffer plugin outputs the following columns:
\begin{longtable}{>{\tt}lll>{\tt\small}l}
    \toprule
    {\bf Column} & {\bf Type} & {\bf Description} & {\bf Flags}\\
    \midrule\endhead%
    \nameref{httpStat}        & H16   & Status                 & \\
    \nameref{httpAFlags}      & H16   & Anomaly flags          & \\
    \nameref{httpMethods}     & H8    & HTTP methods           & \\
    \nameref{httpHeadMimes}   & H16   & HEADMIME-TYPES         & \\
    \nameref{httpCFlags}      & H8    & HTTP content body info & HTTP\_BODY=1\\
    \bottomrule
\end{longtable}

\subsection{Monitoring Output}
In monitoring mode, the httpSniffer plugin outputs the following columns:
\begin{longtable}{>{\tt}lll>{\tt\small}l}
    \toprule
    {\bf Column} & {\bf Type} & {\bf Description} & {\bf Flags}\\
    \midrule\endhead%
    httpPkts & U64 & Number of HTTP packets & \\
    \bottomrule
\end{longtable}

\subsection{Plugin Report Output}
The following information is reported:
\begin{itemize}
    \item Max number of file handles (only if {\tt HTTP\_SAVE=1})
    \item Number of HTTP packets
    \item Number of HTTP \#GET, \#POST, \#GET/\#POST ratio
    \item Aggregated {\tt\nameref{httpStat}}
    \item Aggregated {\tt\nameref{httpHeadMimes}}
    \item Aggregated {\tt\nameref{httpAFlags}}
    \item Aggregated {\tt\nameref{httpCFlags}} ({\tt HTTP\_BODY=1})
\end{itemize}

The GET/POST ratio is very helpful in detecting malware operations, if you know the normal ratio of your
machines in the network. The file descriptor gives you an indication of the maximum file handles the
present pcap will produce. You can increase it by invoking {\tt uname -n mylimit}, but it should not
be necessary as we manage the number of handle being open to be always below the max limit.

\end{document}
