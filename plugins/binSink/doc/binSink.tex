\IfFileExists{t2doc.cls}{
    \documentclass[documentation]{subfiles}
}{
    \errmessage{Error: could not find 't2doc.cls'}
}

\begin{document}

\trantitle
    {binSink}
    {Binary Output}
    {Tranalyzer Development Team}

\section{binSink}\label{s:binSink}

\subsection{Description}
The binSink plugin is one of the basic output plugin for Tranalyzer2.
It uses the output prefix ({\tt --w} option) to generate a binary flow file with suffix {\tt \_flows.bin}.
All standard output from every plugin is stored in binary format in this file.

\subsection{Dependencies}

\subsubsection{External Libraries}
If gzip compression is activated ({\tt BFS\_GZ\_COMPRESS=1}), then {\bf zlib} must be installed.
\begin{table}[!ht]
    \centering
    \begin{tabular}{>{\bf}r>{\tt}l>{\tt}l}
        \toprule
                                     &                      & {\bf BFS\_GZ\_COMPRESS=1}\\
        \midrule
        Ubuntu:                      & sudo apt-get install & zlib1g-dev\\
        Arch:                        & sudo pacman -S       & zlib\\
        Gentoo:                      & sudo emerge          & zlib\\
        openSUSE:                    & sudo zypper install  & zlib-devel\\
        Red Hat/Fedora\tablefootnote{If the {\tt dnf} command could not be found, try with {\tt yum} instead}:
                                     & sudo dnf install     & zlib-devel\\
        macOS\tablefootnote{Brew is a packet manager for macOS that can be found here: \url{https://brew.sh}}:
                                     & brew install         & zlib\\
        \bottomrule
    \end{tabular}
\end{table}

\subsubsection{Core Configuration}
This plugin requires the following core configuration:
\begin{itemize}
    \item {\em \$T2HOME/tranalyzer2/src/tranalyzer.h}:
        \begin{itemize}
            \item {\tt BLOCK\_BUF=0}
        \end{itemize}
\end{itemize}

\subsection{Configuration Flags}

The following flags can be used to control the output of the plugin:

\begin{longtable}{>{\tt}lcl}
    \toprule
    {\bf Name} & {\bf Default} & {\bf Description} \\
    \midrule\endhead%
    BFS\_GZ\_COMPRESS  & 0                         & Compress (gzip) the output\\
    BFS\_SFS\_SPLIT    & 1                         & Split the output file (Tranalyzer {\tt --W} option)\\\\
    BFS\_FLOWS\_SUFFIX & {\tt\small "\_flows.bin"} & Suffix to use for the output file\\
    \bottomrule
\end{longtable}

\subsubsection{Environment Variable Configuration Flags}
The following configuration flags can also be configured with environment variables ({\tt ENVCNTRL>0}):
\begin{itemize}
    \item {\tt BFS\_FLOWS\_SUFFIX}
\end{itemize}

\subsection{Post-Processing}

\subsection{t2b2t}
The program {\tt t2b2t} can be used to transform binary Tranalyzer files generated by the \tranrefpl{binSink} or \tranrefpl{socketSink} plugin into text or JSON files.
The converted files use the same format as the ones generated by the \tranrefpl{txtSink} or \tranrefpl{jsonSink} plugin.\\

The program can be found in {\tt\$T2HOME/utils/t2b2t} and can be compiled by typing {\tt make}.\\

The use of the program is straightforward:
\begin{itemize}
    \item bin$\rightarrow$txt: {\tt t2b2t -r FILE\_flows.bin -w FILE\_flows.txt}
    \item bin$\rightarrow$JSON: {\tt t2b2t -r FILE\_flows.bin -j -w FILE\_flows.json}
    \item bin$\rightarrow$compressed txt: {\tt t2b2t -r FILE\_flows.bin -c -w FILE\_flows.txt.gz}
    \item bin$\rightarrow$compressed JSON: {\tt t2b2t -r FILE\_flows.bin -c -j -w FILE\_flows.json.gz}
\end{itemize}

If the {\tt --w} option is omitted, the destination is inferred from the input file, e.g., the examples above would produce the same output files with or without the {\tt --w} option. Note that {\tt --w --} can be used to output to stdout.\\
Additionally, the {\tt --n} option can be used {\bf not} to print the name of the columns as the first row.\\
Try {\tt t2b2t -h} for more information.

\subsection{Custom File Output}
\begin{itemize}
    \item {\tt PREFIX\_flows.bin}: Binary representation of Tranalyzer output
\end{itemize}

\end{document}
