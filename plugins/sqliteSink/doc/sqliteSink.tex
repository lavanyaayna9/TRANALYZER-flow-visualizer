\IfFileExists{t2doc.cls}{
    \documentclass[documentation]{subfiles}
}{
    \errmessage{Error: could not find 't2doc.cls'}
}

\begin{document}

\trantitle
    {sqliteSink} % Plugin name
    {SQLite} % Short description
    {Tranalyzer Development Team} % author(s)

\section{sqliteSink}\label{s:sqliteSink}

\subsection{Description}
The sqliteSink plugin outputs flows to a SQLite database.

\subsection{Dependencies}

\subsubsection{External Libraries}
This plugin depends on the {\bf sqlite} library.
\begin{table}[!ht]
    \centering
    \begin{tabular}{>{\bf}r>{\tt}l>{\tt}l>{\tt}l}
        \toprule
        %                             &                      &              \\
        %\midrule
        Ubuntu:                      & sudo apt-get install & libsqlite3-dev\\
        Arch:                        & sudo pacman -S       & sqlite        \\
        %Gentoo:                      & sudo emerge          & sqlite        \\
        openSUSE:                    & sudo zypper install  & sqlite3-devel \\
        Red Hat/Fedora\tablefootnote{If the {\tt dnf} command could not be found, try with {\tt yum} instead}:
                                     & sudo dnf install     & sqlite-devel  \\
        macOS\tablefootnote{Brew is a packet manager for macOS that can be found here: \url{https://brew.sh}}:
                                     & brew install         & sqlite        \\
        \bottomrule
    \end{tabular}
\end{table}

\subsubsection{Core Configuration}
This plugin requires the following core configuration:
\begin{itemize}
    \item {\em \$T2HOME/tranalyzer2/src/tranalyzer.h}:
        \begin{itemize}
            \item {\tt BLOCK\_BUF=0}
        \end{itemize}
\end{itemize}

\subsection{Configuration Flags}
The following flags can be used to control the output of the plugin:
\begin{longtable}{>{\tt}lcl}
    \toprule
    {\bf Name} & {\bf Default} & {\bf Description}\\
    \midrule\endhead%
    SQLITE\_OVERWRITE                              & 2                                & 0: abort if table already exists\\
                                                   &                                  & 1: overwrite table if it already exists\\
                                                   &                                  & 2: append to table if it already exists\\
    SQLITE\_HEX\_AS\_INT                           & 0                                & 0: store hex numbers (bitfields) as text\\
                                                   &                                  & 1: store hex numbers (bitfields) as int\\
    SQLITE\_TRANSACTION\_NFLOWS                    & 40000                            & 0: one transaction\\
                                                   &                                  & > 0: one transaction every $n$ flows\\
    SQLITE\_QRY\_LEN                               & 32768                            & Initial length for query\\
    SQLITE\_QRY\_MAXLEN                            & 4194304                          & Maximal length for query\\

    \hyperref[sqlite:dbname]{SQLITE\_DB\_SUFFIX}   & {\tt\small ".db"}                & Suffix for the database name\\
    \hyperref[sqlite:dbname]{SQLITE\_DBNAME}       & {\tt\small "/tmp/t2.db"}         & Name of the database\\
    SQLITE\_TABLE\_NAME                            & {\tt\small "flow"}               & Name of the table\\

    \hyperref[sqlite:select]{T2\_SQLITE\_SELECT}   & 0                                & Only insert specific fields into the DB\\
    \hyperref[sqlite:select]{SQLITE\_SELECT\_FILE} & {\tt\small "sqlite-columns.txt"} & Filename of the field selector\\
                                                   &                                  & (one column name per line)\\
    \bottomrule
\end{longtable}

\subsubsection{Environment Variable Configuration Flags}
The following configuration flags can also be configured with environment variables ({\tt ENVCNTRL>0}):
\begin{itemize}
    \item {\tt SQLITE\_QRY\_LEN}
    \item {\tt SQLITE\_QRY\_MAXLEN}
    \item {\tt SQLITE\_DB\_SUFFIX}
    \item {\tt SQLITE\_TABLE\_NAME}
    \item {\tt SQLITE\_SELECT\_FILE}
\end{itemize}

\subsubsection{Database Name}\label{sqlite:dbname}
The database name is extracted from Tranalyzer input and/or {\tt --w/--W} option.
{\tt SQLITE\_DB\_SUFFIX} is simply appended.
Alternatively, an absolute path may be provided by uncommenting the {\tt SQLITE\_DBNAME} macro in {\em src/sqliteSink.h}.

\subsection{Insertion of Selected Fields Only}\label{sqlite:select}

When {\small\tt T2\_SQLITE\_SELECT=1}, the columns to insert into the DB can be customized with the help of {\small\tt SQLITE\_SELECT\_FILE}.
The filename defaults to {\tt sqlite-columns.txt} in the user plugin folder, e.g., {\em \textasciitilde{}/.tranalyzer/plugins}.
The format of the file is simply one field name per line with lines starting with a {\tt `\#'} being ignored.
For example, to only insert source and destination addresses and ports, create the following file:

\begin{verbatim}
# Lines starting with a '#' are ignored and can be used to add comments
srcIP
srcPort
dstIP
dstPort
\end{verbatim}

\subsection{Plugin Report Output}
The following information is reported:
\begin{itemize}
    \item Number of flows discarded due to main buffer problems
\end{itemize}

\subsection{Example}
{\tt\color{blue} \# Run Tranalyzer}\\
{\tt \$ t2 -r file.pcap}\\
{\tt\color{blue} \# Connect to the SQLite database}\\
{\tt \$ sqlite3 file.db}\\
{\tt\color{blue} \# Number of flows}\\
{\tt sqlite> select count(*) from flow;}\\
{\tt\color{blue} \# 10 first srcIP/dstIP pairs}\\
{\tt sqlite> select "srcIP", "dstIP" from flow limit 10;}\\
{\tt\color{blue} \# All flows from 1.2.3.4 to 1.2.3.5}\\
{\tt sqlite> select * from flow where "srcIP" = '1.2.3.4' and "dstIP" = '1.2.3.5';}\\

\end{document}
