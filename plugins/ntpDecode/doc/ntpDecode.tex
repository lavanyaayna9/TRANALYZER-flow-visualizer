\IfFileExists{t2doc.cls}{
    \documentclass[documentation]{subfiles}
}{
    \errmessage{Error: could not find t2doc.cls}
}

\begin{document}

\trantitle
    {ntpDecode}
    {Network Time Protocol (NTP)}
    {Tranalyzer Development Team}

\section{ntpDecode}\label{s:ntpDecode}

\subsection{Description}
The ntpDecode plugin produces a flow based view of NTP operations between computers for anomaly detection
and troubleshooting.

\subsection{Configuration Flags}
The following flags can be used to control the output of the plugin:
\begin{longtable}{>{\tt}lcl>{\tt\small}l}
    \toprule
    {\bf Name} & {\bf Default} & {\bf Description} & {\bf Flags}\\
    \midrule\endhead%
    NTP\_TS        & 1 & 0: no time stamps,                 & \\
                   &   & 1: print NTP time stamps           & \\
    NTP\_LIVM\_HEX & 0 & Leap indicator, version and mode:  & \\
                   &   & \qquad 0: split into three values, & \\
                   &   & \qquad 1: aggregated hex number    & \\
    \bottomrule
\end{longtable}

\subsection{Flow File Output}
The ntpDecode plugin outputs the following columns:
\begin{longtable}{>{\tt}llll>{\tt\small}ll}
    \toprule
    {\bf Name}                      & {\bf Type} & {\bf Description}                              & {\bf Flags}\\
    \midrule\endhead%
    \nameref{ntpStat}               & H8         & NTP status, warnings and errors                & \\
    \nameref{ntpLiVM}               & H8         & NTP leap indicator, version number and mode    & NTP\_LIVM\_HEX=1\\
    \hyperref[ntpLiVM]{ntpLi\_V\_M} & U8\_U8\_U8 & NTP leap indicator, version number and mode    & NTP\_LIVM\_HEX=0\\
    \nameref{ntpStrat}              & H8         & NTP stratum                                    & \\
    ntpRefClkId                     & IP4        & NTP root reference clock ID (stratum $\geq$ 2) & \\
    \nameref{ntpRefStrId}           & SC         & NTP root reference string (stratum $\leq$ 1)   & \\
    ntpPollInt                      & U32        & NTP poll interval                              & \\
    ntpPrec                         & F          & NTP precision                                  & \\
    ntpRtDelMin                     & F          & NTP root delay minimum                         & \\
    ntpRtDelMax                     & F          & NTP root delay maximum                         & \\
    ntpRtDispMin                    & F          & NTP root dispersion minimum                    & \\
    ntpRtDispMax                    & F          & NTP root dispersion maximum                    & \\
    ntpRefTS                        & TS         & NTP reference timestamp                        & NTP\_TS=1\\
    ntpOrigTS                       & TS         & NTP originate timestamp                        & NTP\_TS=1\\
    ntpRecTS                        & TS         & NTP receive timestamp                          & NTP\_TS=1\\
    ntpTranTS                       & TS         & NTP transmit timestamp                         & NTP\_TS=1\\
    \bottomrule
\end{longtable}

\clearpage

\subsubsection{ntpStat}\label{ntpStat}
The {\tt ntpStat} column is to be interpreted as follows:
\begin{longtable}{>{\tt}rl}
    \toprule
    {\bf ntpStat} & {\bf Description}\\
    \midrule\endhead%
    $2^0$ (=0x01) & NTP port detected \\
    $2^1$ (=0x02) & --- \\
    $2^2$ (=0x04) & --- \\
    $2^3$ (=0x08) & --- \\
    \\
    $2^4$ (=0x10) & --- \\
    $2^5$ (=0x20) & --- \\
    $2^6$ (=0x40) & --- \\
    $2^7$ (=0x80) & --- \\
    \bottomrule
\end{longtable}

\subsubsection{ntpLiVM}\label{ntpLiVM}
The {\tt ntpLiVM} column is to be interpreted as follows (refer to \refs{ntp:examples} for some examples):
\begin{longtable}{>{\tt}rl}
    \toprule
    {\bf ntpLiVM} & {\bf Description}\\
    \midrule\endhead%
    xx.. ....  & Leap indicator\\
    ..xx~ x... & Version number\\ % XXX tilda is a hack to fix the output
    .... .xxx  & Mode\\
    \bottomrule
\end{longtable}

The {\tt Leap Indicator} bits are to be interpreted as follows:
\begin{longtable}{>{\tt}rl}
    \toprule
    {\bf Leap Indicator} & {\bf Description}\\
    \midrule\endhead%
    0x0 & No warning\\
    0x1 & Last minute has 61 seconds\\
    0x2 & Last minute has 59 seconds\\
    0x3 & Alarm condition, clock not synchronized\\
    \bottomrule
\end{longtable}

The {\tt Mode} bits are to be interpreted as follows:
\begin{longtable}{>{\tt}rl}
    \toprule
    {\bf Mode} & {\bf Description}\\
    \midrule\endhead%
    0x0 & Reserved\\
    0x1 & Symmetric active\\
    0x2 & Symmetric passive\\
    0x3 & Client\\
    0x4 & Server\\
    0x5 & Broadcast\\
    0x6 & NTP control message\\
    0x7 & Private use\\
    \bottomrule
\end{longtable}

\subsubsection{ntpStrat}\label{ntpStrat}
The {\tt ntpStrat} column is to be interpreted as follows:
\begin{longtable}{>{\tt}rl}
    \toprule
    {\bf ntpStrat} & {\bf Description}\\
    \midrule\endhead%
    0x00       & Unspecified \\
    0x01       & Primary reference\\
    0x02--0xff & Secondary reference\\
    \bottomrule
\end{longtable}

\subsubsection{ntpRefStrId}\label{ntpRefStrId}
The interpretation of the {\tt ntpRefStrId} column depends on the value of \nameref{ntpStrat}.
The following table lists some suggested identifiers:
\begin{longtable}{>{\tt}r>{\tt}rl}
    \toprule
    {\bf ntpStrat} & {\bf ntpRefStrId} & {\bf Description}\\
    \midrule\endhead%
    0x00 & DCN      & DCN routing protocol\\
    0x00 & NIST     & NIST public modem\\
    0x00 & TSP      & TSP time protocol\\
    0x00 & DTS      & Digital Time Service\\
    0x01 & ATOM     & Atomic clock (calibrated)\\
    0x01 & VLF      & VLF radio\\
    0x01 & callsign & Generic radio\\
    0x01 & LORC     & LORAN-C\\
    0x01 & GOES     & GOES UHF environment satellite\\
    0x01 & GPS      & GPS UHF positioning satellite\\
    \bottomrule
\end{longtable}

\subsection{Monitoring Output}
In monitoring mode, the ntpDecode plugin outputs the following columns:
\begin{longtable}{>{\tt}lll>{\tt\small}l}
    \toprule
    {\bf Column} & {\bf Type} & {\bf Description} & {\bf Flags}\\
    \midrule\endhead%
    ntpPkts & U64 & Number of NTP packets & \\
    \bottomrule
\end{longtable}

\subsection{Plugin Report Output}
The following information is reported:
\begin{itemize}
    \item Aggregated {\tt\nameref{ntpStat}}
    \item Number of NTP packets
\end{itemize}

\subsection{Examples}\label{ntp:examples}
\begin{itemize}
    \item Extract the NTP leap indicator:\\
          {\tt tawk 'NR > 1 \{ print rshift(and(strtonum(\$ntpLiVM), 0xc0), 6) \}' out\_flows.txt}
    \item Extract the NTP version:\\
          {\tt tawk 'NR > 1 \{ print rshift(and(strtonum(\$ntpLiVM), 0x38), 3) \}' out\_flows.txt}
    \item Extract the NTP mode:\\
            {\tt tawk 'NR > 1 \{ printf "\%\#x\textbackslash{}n", and(strtonum(\$ntpLiVM), 0x7) \}' out\_flows.txt}
\end{itemize}

\end{document}
