\IfFileExists{t2doc.cls}{
    \documentclass[documentation]{subfiles}
}{
    \errmessage{Error: could not find t2doc.cls}
}

\begin{document}

\trantitle
    {centrality}
    {Centrality}
    {Tranalyzer Development Team}

\section{centrality}\label{s:centrality}

\subsection{Description}
This plugin produces a connection matrix from pcap files and calculates the centrality
of each IP address. The centrality is defined by the corresponding entry in the eigenvector to the largest eigenvalue of the adjacency matrix (\refs{s:eigenvector-centrality}).

\subsection{Dependencies}

\subsubsection{Other Plugins}
This plugin requires the \tranrefpl{basicStats} plugin if {\tt CENTRALITY\_MATRIXENTRIES} $\geq 2$.

\subsection{Eigenvector centrality}\label{s:eigenvector-centrality}
Let $G=(V,E)$ be a graph, where $V$ is a set of vertices $v_i$ and $E$ a set of edges connecting those vertices. We define the \emph{adjacency matrix} $A \in\mathbb{N}_0^{n\times n}$ matrix labelled by the vertices of $G$, where the entry $A_{i,j}=1$, if there is a connecting edge between $v_i$ and $v_j$.\\
Eigenvalues are defined as roots of the characteristic polynomial $p(\lambda) = det(A-\lambda*I)$, where $I$ is the identity matrix and $A$ the adjacency matrix.\\
Each eigenvalue $\lambda \in \mathbb{C}$ has a corresponding eigenvector $\mathbf{x} \in \mathbb{C}^{n\times1}$ which satisfies the equation $A*\mathbf{x}=\lambda*\mathbf{x}$.\\
The centrality of vertex $i$ is defined by: $centrality(v_i):=\mathbf{x}_i$, where $\mathbf{x}$ is the eigenvector to $\lambda_{max} := max(\lambda \in \mathbb{R})$.
All values of $\mathbf{x}$ are strictly positive.\\
This Plugin uses a simple version of the power iteration to calculate this eigenvector: \\
$\mathbf{x}_{n+1} = \frac{A*\mathbf{x}_n}{\|A*\mathbf{x}_n\|}$. For $n \to \infty$, $\mathbf{x}_n$ converges to $\mathbf{x}$.

\subsection{Configuration Flags}
The following flags can be used to control the output of the plugin:
\begin{center}
    \begin{tabular}{>{\tt}lcl}
        \toprule
        {\bf Variable} & {\bf Default} & {\bf Description}\\
        \midrule
        CENTRALITY\_MATRIXENTRIES & 1 & 0: $A_{i,j}\in\{0,1\}$\\
                                  &   & 1: $A_{i,j}=$ number of flows between $v_i$ and $v_j$\\
                                  &   & 2: $A_{i,j}=$ number of bytes sent between $v_i$ and $v_j$\\
                                  &   & 3: $A_{i,j}=$ bytes asymmetry between $v_i$ and $v_j$\\
                                  &   & 4: $A_{i,j}=$ number of packets sent between $v_i$ and $v_j$\\
                                  &   & 5: $A_{i,j}=$ packets asymmetry between $v_i$ and $v_j$\\
        CENTRALITY\_TIME\_CALC    & 1 & 0: centrality calculated at application termination,\\
                                  &   & $\mathbb{N} \ni n\neq0$: centrality calculated every $n$ seconds (dump time)\\
        CENTRALITY\_IP\_FORMAT    & 1 & 0: IPs as unsigned integer\\
                                  &   & 1: IPs as hex\\
                                  &   & 2: IPs in compressed format, e.g., 1.2.3.4\\
        CENTRALITY\_MATRIXFILE    & 0 & 1: Write a file with triplet matrix\\
        CENTRALITY\_TRAVIZ        & 0 & 1: Traviz output mode\\
        \bottomrule
    \end{tabular}
\end{center}

In addition, {\tt CENTRALITY\_SUFFIX} and {\tt MATRIX\_SUFFIX} can be used to control the suffix for the centrality and matrix output files ({\tt "\_centrality.txt"} and {\tt "\_matrix.txt"} respectively).

\subsection{Flow File Output}
There is no output to the flow file.

\subsection{Centrality File Output}

If {\tt CENTRALITY\_TRAVIZ=1}, then the first row is as follows:
\begin{center}
    {\tt \% number\_of\_rows time IP centrality}
\end{center}
Where {\tt number\_of\_rows} is the actual number of entries in the file and {\tt time}, {\tt IP} and {\tt centrality} represent the name of the various columns.
\begin{center}
    \begin{tabular}{rcl>{\tt\small}l}
        \toprule
        {\bf Column} & {\bf Type} & {\bf Description} & {\bf Flags}\\
        \midrule
        1 & U32/U64 & Time (seconds) & CENTRALITY\_TRAVIZ=1\\
        2 & IP4/U32 & IPv4 address   & \\
        3 & D       & Centrality     & \\
        \bottomrule
    \end{tabular}
\end{center}

%In every centrality file, the first host will be:\\
%
%\begin{center}
%    \begin{tabular}{cccc}
%        1 & 0 & 0.0.0.0 & 1.0000000000\\
%    \end{tabular}
%\end{center}
%
%This host is used to normalize the centralities with the maximum possible centrality. He will get an imaginary connection to every new host that appears to the network.

\subsection{Matrix File Output}

This File shows the adjacency matrix of your network in triplet matrix format. You can restore the original matrix $A$ by assigning $A_{row.column}=value$ and $A_{i.j}=0$ for all unassigned indices.\\
The first row and column will be full of ones as this is the maximum centrality host used for normalization.

\begin{center}
    \begin{tabular}{rcl}
        \toprule
        {\bf Column} & {\bf Type} & {\bf Description}\\
        \midrule
        1 & I & Matrix row\\
        2 & I & Matrix column\\
        3 & I & value of $A_{row,column}$\\
        \bottomrule
    \end{tabular}
\end{center}

\subsection{TODO}

\begin{itemize}
    \item Add support for IPv6
    \item Analyzing centralities in known networks
    \item Classification of networks by their centrality
\end{itemize}

\end{document}
