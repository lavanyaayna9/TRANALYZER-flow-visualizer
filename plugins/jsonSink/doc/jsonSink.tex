\IfFileExists{t2doc.cls}{
    \documentclass[documentation]{subfiles}
}{
    \errmessage{Error: could not find 't2doc.cls'}
}

\begin{document}

\trantitle
    {jsonSink}
    {JSON Output}
    {Tranalyzer Development Team} % author(s)

\section{jsonSink}\label{s:jsonSink}

\subsection{Description}
The jsonSink plugin generates JSON output in a file {\tt PREFIX\_flows.json}, where {\tt PREFIX} is provided via Tranalyzer {\tt --w} or {\tt --W} option.

\subsection{Dependencies}

\subsubsection{External Libraries}
If gzip compression is activated ({\tt JSON\_GZ\_COMPRESS=1}), then {\bf zlib} must be installed.
\begin{table}[!ht]
    \centering
    \begin{tabular}{>{\bf}r>{\tt}l>{\tt}l}
        \toprule
                                     &                      & {\bf JSON\_GZ\_COMPRESS=1}\\
        \midrule
        Ubuntu:                      & sudo apt-get install & zlib1g-dev\\
        Arch:                        & sudo pacman -S       & zlib\\
        Gentoo:                      & sudo emerge          & zlib\\
        openSUSE:                    & sudo zypper install  & zlib-devel\\
        Red Hat/Fedora\tablefootnote{If the {\tt dnf} command could not be found, try with {\tt yum} instead}:
                                     & sudo dnf install     & zlib-devel\\
        macOS\tablefootnote{Brew is a packet manager for macOS that can be found here: \url{https://brew.sh}}:
                                     & brew install         & zlib\\
        \bottomrule
    \end{tabular}
\end{table}

\subsubsection{Core Configuration}
This plugin requires the following core configuration:
\begin{itemize}
    \item {\em \$T2HOME/tranalyzer2/src/tranalyzer.h}:
        \begin{itemize}
            \item {\tt BLOCK\_BUF=0}
        \end{itemize}
\end{itemize}

\subsection{Configuration Flags}

The following flags can be used to control the output of the plugin:

\begin{longtable}{>{\tt}lcl>{\tt\small}l}
    \toprule
    {\bf Name} & {\bf Default} & {\bf Description} & {\bf Flags}\\
    \midrule\endhead%
    JSON\_SOCKET\_ON             & 0                              & Output to a socket (1) or to a file (0) & \\
    JSON\_SOCKET\_ADDR           & {\tt\small ``127.0.0.1''}      & Address of the socket                   & JSON\_SOCKET\_ON=1\\
    JSON\_SOCKET\_PORT           & 5000                           & Port of the socket                      & JSON\_SOCKET\_ON=1\\
                                 &                                &                                         & \\
    JSON\_GZ\_COMPRESS           & 0                              & Compress (gzip) the output              & \\
    JSON\_SPLIT                  & 1                              & Split the output file                   & JSON\_SOCKET\_ON=0\\
                                 &                                & (Tranalyzer {\tt --W} option)           & \\
                                 &                                &                                         & \\
    JSON\_ROOT\_NODE             & 0                              & Add a root node (array)                 & \\
    JSON\_SUPPRESS\_EMPTY\_ARRAY & 1                              & Do not output empty fields              & \\
    JSON\_NO\_SPACES             & 1                              & Suppress unnecessary spaces             & \\
    JSON\_BUFFER\_SIZE           & 1048576                        & Size of output buffer                   & \\
                                 &                                &                                         & \\
    \hyperref[json:select]{JSON\_SELECT}
                                 & 0                              & Only output specific fields             & \\
    \hyperref[json:select]{JSON\_SELECT\_FILE}
                                 & {\small\tt "json-columns.txt"} & Filename of the field selector          & JSON\_SELECT=1\\
                                 &                                & (one column name per line)\\
                                 &                                &                                         & \\
    JSON\_SUFFIX                 & {\tt\small "\_flows.json"}     & Suffix for output file                  & JSON\_SOCKET\_ON=0\\
    \bottomrule
\end{longtable}

\subsubsection{Environment Variable Configuration Flags}
The following configuration flags can also be configured with environment variables ({\tt ENVCNTRL>0}):
\begin{itemize}
    \item {\tt JSON\_BUFFER\_SIZE}
    \item {\tt JSON\_SOCKET\_ADDR} (require {\tt JSON\_SOCKET\_ON=1})
    \item {\tt JSON\_SOCKET\_PORT} (require {\tt JSON\_SOCKET\_ON=1})
    \item {\tt JSON\_SUFFIX} (require {\tt JSON\_SOCKET\_ON=0})
    \item {\tt JSON\_SELECT\_FILE}
\end{itemize}

\subsection{Plugin Report Output}
The following information is reported:
\begin{itemize}
    \item Number of flows discarded due to main buffer problems
\end{itemize}

\subsection{Custom File Output}
\begin{itemize}
    \item {\tt PREFIX\_flows.json}: JSON representation of Tranalyzer output
\end{itemize}

\subsection{Output Selected Fields Only}\label{json:select}

When {\tt JSON\_SELECT=1}, the columns to output can be customized with the help of {\tt JSON\_SELECT\_FILE}.
The filename defaults to {\tt json-columns.txt} in the user plugin folder, e.g., {\em \textasciitilde{}/.tranalyzer/plugins}.
The format of the file is simply one field name per line with lines starting with a {\tt `\#'} being ignored.
For example, to only output source and destination addresses and ports, create the following file:

\begin{verbatim}
# Lines starting with a '#' are ignored and can be used to add comments
srcIP
srcPort
dstIP
dstPort
\end{verbatim}

\subsection{Example}
To send compressed data over a socket ({\tt JSON\_SOCKET\_ON=1} and {\tt JSON\_GZ\_COMPRESS=1}):
\begin{enumerate}
    \item {\tt nc -l 127.0.0.1 5000 | gunzip}
    \item {\tt tranalyzer -r file.pcap}
\end{enumerate}

\end{document}
