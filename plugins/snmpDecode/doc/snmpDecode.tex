\IfFileExists{t2doc.cls}{
    \documentclass[documentation]{subfiles}
}{
    \errmessage{Error: could not find 't2doc.cls'}
}

\begin{document}

\trantitle
    {snmpDecode} % Plugin name
    {Simple Network Management Protocol (SNMP)} % Short description
    {Tranalyzer Development Team} % author(s)

\section{snmpDecode}\label{s:snmpDecode}

\subsection{Description}
The snmpDecode plugin analyzes SNMP traffic.

\subsection{Configuration Flags}
The following flags can be used to control the output of the plugin:
\begin{longtable}{>{\tt}lcl}
    \toprule
    {\bf Name} & {\bf Default} & {\bf Description}\\
    \midrule\endhead%
    SNMP\_STRLEN & 64 & Maximum length for strings\\
    \bottomrule
\end{longtable}

\subsection{Flow File Output}
The snmpDecode plugin outputs the following columns:
\begin{longtable}{>{\tt}lll>{\tt\small}l}
    \toprule
    {\bf Column}                   & {\bf Type}   & {\bf Description}       & {\bf Flags}\\
    \midrule\endhead%
    \nameref{snmpStat}             & H8           & Status                  & \\
    \nameref{snmpVersion}          & U8           & Version                 & \\
    snmpCommunity                  & S            & Community (SNMPv1-2)    & \\
    snmpUser                       & S            & Username (SNMPv3)       & \\
    \hyperref[snmpTypes]{snmpMsgT} & H16          & Message types bitfield  & \\
    snmpNumReq\_                   & U64\_        & Number of GetRequest,   & \\
    \qquad Next\_                  & \qquad U64\_ & \qquad GetNextRequest,  & \\
    \qquad Resp\_                  & \qquad U64\_ & \qquad GetResponse,     & \\
    \qquad Set\_                   & \qquad U64\_ & \qquad SetRequest,      & \\
    \qquad Trap1\_                 & \qquad U64\_ & \qquad Trapv1,          & \\
    \qquad Bulk\_                  & \qquad U64\_ & \qquad GetBulkRequest,  & \\
    \qquad Info\_                  & \qquad U64\_ & \qquad InformRequest,   & \\
    \qquad Trap2\_                 & \qquad U64\_ & \qquad Trapv2,          & \\
    \qquad Rep                     & \qquad U64   & \qquad Report packets   & \\
    \bottomrule
\end{longtable}

\subsubsection{snmpStat}\label{snmpStat}
The {\tt snmpStat} column is to be interpreted as follows:
\begin{longtable}{>{\tt}rl}
    \toprule
    {\bf snmpStat} & {\bf Description}\\
    \midrule\endhead%
    0x01 & Flow is SNMP\\
    0x02 & ---\\
    0x04 & ---\\
    0x08 & ---\\
    \\
    0x10 & ---\\
    0x20 & ---\\
    0x40 & String was truncated\ldots increase {\tt SNMP\_STRLEN}\\
    0x80 & Packet was malformed\\
    \bottomrule
\end{longtable}

\subsubsection{snmpVersion}\label{snmpVersion}
The {\tt snmpVersion} column is to be interpreted as follows:
\begin{longtable}{rl}
    \toprule
    {\bf snmpVersion} & {\bf Description}\\
    \midrule\endhead%
    0 & SNMPv1\\
    1 & SNMPv2c\\
    3 & SNMPv3\\
    \bottomrule
\end{longtable}

\subsubsection{snmpMsgT and snmpType}\label{snmpTypes}
The {\tt snmpMsgT} and {\tt snmpType} columns are to be interpreted as follows:
\begin{longtable}{>{\tt}r>{\tt}rl}
    \toprule
    {\bf snmpMsgT} & {\bf snmpType} & {\bf Description}\\
    \midrule\endhead%
    0x0001 & 0xa0 & GetRequest\\
    0x0002 & 0xa1 & GetNextRequest\\
    0x0004 & 0xa2 & GetResponse\\
    0x0008 & 0xa3 & SetRequest\\
    \\
    0x0010 & 0xa4 & Trap (v1)\\
    0x0020 & 0xa5 & GetBulkRequest (v2c, v3)\\
    0x0040 & 0xa6 & InformRequest\\
    0x0080 & 0xa7 & Trap (v2c, v3)\\
    \\
    0x0100 & 0xa8 & Report\\
    \bottomrule
\end{longtable}

\subsection{Packet File Output}
In packet mode ({\tt --s} option), the snmpDecode plugin outputs the following columns:
\begin{longtable}{>{\tt}lll>{\tt\small}l}
    \toprule
    {\bf Column} & {\bf Type} & {\bf Description} & {\bf Flags}\\
    \midrule\endhead%
    \nameref{snmpVersion}          & U8 & Version           & \\
    snmpCommunity                  & S  & Community         & \\
    snmpUser                       & S  & Username (SNMPv3) & \\
    \hyperref[snmpTypes]{snmpType} & H8 & Message type      & \\
    \bottomrule
\end{longtable}

\subsection{Plugin Report Output}
The following information is reported:
\begin{itemize}
    \item Number of SNMP packets
    \item Number of SNMP GetRequest packets
    \item Number of SNMP GetNextRequest packets
    \item Number of SNMP GetResponse packets
    \item Number of SNMP SetRequest packets
    \item Number of SNMP Trap v1 packets
    \item Number of SNMP GetBulkRequest packets
    \item Number of SNMP InformRequest packets
    \item Number of SNMP Trap v2 packets
    \item Number of SNMP Report packets
    %\item Aggregated status flags ({\tt\nameref{snmpStat}})
\end{itemize}

\end{document}
