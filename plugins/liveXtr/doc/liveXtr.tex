\IfFileExists{t2doc.cls}{
    \documentclass[documentation]{subfiles}
}{
    \errmessage{Error: could not find 't2doc.cls'}
}

\begin{document}

\trantitle
    {liveXtr} % Plugin name
    {Live PCAP extractor} % Short description
    {Tranalyzer Development Team} % author(s)

\section{liveXtr}\label{s:liveXtr}

\subsection{Description}
This plugin extracts all flows which have the {\tt LIVEXTR} bit set in their
{\tt flow\_t.status}. This bit can be set by any other plugin wanting to extract traffic.
The traffic is stored in a round-robin buffer, this allows to extract packets processed before
the {\tt LIVEXTR} bit was set. This plugin can be used when Tranalyzer processes PCAP files and
also when it listens on an interface.

\subsection{Dependencies}

The {\tt liveXtr} plugin depends on the libpcap.


\subsection{Configuration Flags}
The following flags can be used to control the behavior of the plugin:
\begin{longtable}{>{\tt}lcl>{\tt\small}l}
    \toprule
    {\bf Name} & {\bf Default} & {\bf Description} & {\bf Flags}\\
    \midrule\endhead%
    LIVEXTR\_BUFSIZE & $2^{31}$ (2 GB)                 & Size of the round-robin buffer [byte]         & \\
    LIVEXTR\_MEMORY  & 1                               & 0: store round-robin buffer in a file,        & \\
                     &                                 & 1: store it in memory                         & \\
    LIVEXTR\_SPLIT   & 1                               & Split the output PCAP with t2 {\tt -W} option & \\
    LIVEXTR\_FILE    & {\tt\small "/tmp/livextr.data"} & Round-robin buffer file location              & LIVEXTR\_MEMORY=0\\
    \bottomrule
\end{longtable}


\subsection{Plugin Report Output}
The following information is reported:
\begin{itemize}
    \item Number of extracted packets
\end{itemize}

\subsection{Additional Output}
Non-standard output:
\begin{itemize}
    \item {\tt PREFIX\_livextr.pcap}: PCAP containing the extracted packets.
\end{itemize}

\end{document}
