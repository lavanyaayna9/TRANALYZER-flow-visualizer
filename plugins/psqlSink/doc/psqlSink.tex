\IfFileExists{t2doc.cls}{
    \documentclass[documentation]{subfiles}
}{
    \errmessage{Error: could not find 't2doc.cls'}
}

\begin{document}

\trantitle
    {psqlSink} % Plugin name
    {PostgreSQL} % Short description
    {Tranalyzer Development Team} % author(s)

\section{psqlSink}\label{s:psqlSink}

\subsection{Description}
The psqlSink plugin outputs flows to a PostgreSQL database.

\subsection{Dependencies}

\subsubsection{External Libraries}
This plugin depends on the {\bf libpq} library.
\begin{table}[!ht]
    \centering
    \begin{tabular}{>{\bf}r>{\tt}l>{\tt}l>{\tt}l}
        \toprule
        %                             &                      &                 \\
        %\midrule
        Ubuntu:                      & sudo apt-get install & libpq-dev       \\
        Arch:                        & sudo pacman -S       & postgresql-libs \\
        Gentoo:                      & sudo emerge          & postgresql      \\
        openSUSE:                    & sudo zypper install  & postgresql-devel\\
        Red Hat/Fedora\tablefootnote{If the {\tt dnf} command could not be found, try with {\tt yum} instead}:
                                     & sudo dnf install     & libpq-devel     \\
        macOS\tablefootnote{Brew is a packet manager for macOS that can be found here: \url{https://brew.sh}}:
                                     & brew install         & postgresql      \\
        \bottomrule
    \end{tabular}
\end{table}

\subsubsection{Core Configuration}
This plugin requires the following core configuration:
\begin{itemize}
    \item {\em \$T2HOME/tranalyzer2/src/tranalyzer.h}:
        \begin{itemize}
            \item {\tt BLOCK\_BUF=0}
        \end{itemize}
\end{itemize}

\subsection{Database Initialization}

The psqlSink plugin requires a PostgreSQL server running on {\tt PSQL\_HOST}, e.g., {\tt 127.0.0.1} on port {\tt PSQL\_PORT}, e.g., 5432.
In addition, a user with name {\tt PSQL\_USER}, e.g., {\tt postgres}, and password {\tt PSQL\_PASS}, e.g., {\tt postgres} {\bf MUST} exist and be allowed to create databases. This can be achieved with the following commands:\\

\begin{minipage}{.48\textwidth}
{\bf Linux:} {\tt \$ sudo -u postgres psql}\\
\end{minipage}
\hfill
\begin{minipage}{.48\textwidth}
{\bf macOS:} {\tt \$ psql postgres}\\
\end{minipage}

\noindent
{\tt postgres=\# \textcolor{darkblue}{CREATE ROLE} postgres \textcolor{darkblue}{WITH LOGIN PASSWORD} '\textcolor{cyan}{postgres}';}\\
{\tt CREATE ROLE}\\
{\tt postgres=\# \textcolor{darkblue}{ALTER ROLE} postgres \textcolor{darkblue}{CREATEDB};}\\
{\tt ALTER ROLE}

\subsection{Configuration Flags}
The following flags can be used to control the output of the plugin:
\begin{longtable}{>{\tt}lcl}
    \toprule
    {\bf Name} & {\bf Default} & {\bf Description}\\
    \midrule\endhead%
    PSQL\_OVERWRITE\_DB       & 2                        & 0: abort if DB already exists\\
                              &                          & 1: overwrite DB if it already exists\\
                              &                          & 2: reuse DB if it already exists\\
    PSQL\_OVERWRITE\_TABLE    & 2                        & 0: abort if table already exists\\
                              &                          & 1: overwrite table if it already exists\\
                              &                          & 2: append to table if it already exists\\
    PSQL\_TRANSACTION\_NFLOWS & 40000                    & 0: one transaction\\
                              &                          & > 0: one transaction every $n$ flows\\
    PSQL\_QRY\_LEN            & 32768                    & Max length for query\\
    PSQL\_HOST                & {\tt\small "127.0.0.1"}  & Address of the database\\
    PSQL\_PORT                & 5432                     & Port of the database\\
    PSQL\_USER                & {\tt\small "postgres"}   & Username to connect to DB\\
    PSQL\_PASS                & {\tt\small "postgres"}   & Password to connect to DB\\
    PSQL\_DBNAME              & {\tt\small "tranalyzer"} & Name of the database\\
    PSQL\_TABLE\_NAME         & {\tt\small "flow"}       & Name of the table\\
    \\
    \hyperref[psql:select]{PSQL\_SELECT}
                              & 0                        & Only insert specific fields into the DB\\
    \hyperref[psql:select]{PSQL\_SELECT\_FILE}
                              & {\small\tt "psql-columns.txt"}
                                                   & Filename of the field selector (one column name per line)\\
    \bottomrule
\end{longtable}

\subsubsection{Environment Variable Configuration Flags}
The following configuration flags can also be configured with environment variables ({\tt ENVCNTRL>0}):
\begin{itemize}
    \item {\tt PSQL\_HOST}
    \item {\tt PSQL\_PORT}
    \item {\tt PSQL\_USER}
    \item {\tt PSQL\_PASS}
    \item {\tt PSQL\_DBNAME}
    \item {\tt PSQL\_TABLE\_NAME}
    \item {\tt PSQL\_SELECT\_FILE} (require {\tt PSQL\_SELECT=1})
\end{itemize}

\subsection{Insertion of Selected Fields Only}\label{psql:select}

When {\tt PSQL\_SELECT=1}, the columns to insert into the DB can be customized with the help of {\tt PSQL\_SELECT\_FILE}.
The filename defaults to {\tt psql-columns.txt} in the user plugin folder, e.g., {\em \textasciitilde{}/.tranalyzer/plugins}.
The format of the file is simply one field name per line with lines starting with a {\tt `\#'} being ignored.
For example, to only insert source and destination addresses and ports, create the following file:

\begin{verbatim}
# Lines starting with a '#' are ignored and can be used to add comments
srcIP
srcPort
dstIP
dstPort
\end{verbatim}

\subsection{Post-Processing}
The following queries can be used to analyze bitfields in PostgreSQL:
\begin{itemize}
    \item Select all A flows:\\
        \begin{ttfamily}
            \textcolor{darkblue}{\bf SELECT} to\_hex(\textcolor{red}{"flowStat"}::\textcolor{darkblue}{bigint}), *\\
            \textcolor{darkblue}{\bf FROM} flow\\
            \textcolor{darkblue}{\bf WHERE} (\textcolor{red}{"flowStat"}::\textcolor{darkblue}{bigint} \& \textcolor{cyan}{1}) = \textcolor{cyan}{0}::\textcolor{darkblue}{bigint}
        \end{ttfamily}
    \item Select all IPv4 flows:\\
        \begin{ttfamily}
            \textcolor{darkblue}{\bf SELECT} *\\
            \textcolor{darkblue}{\bf FROM} flow\\
            \textcolor{darkblue}{\bf WHERE} (\textcolor{red}{"flowStat"}::\textcolor{darkblue}{bigint} \& x`\textcolor{red}{4000}'::\textcolor{darkblue}{bigint}) != \textcolor{cyan}{0}::\textcolor{darkblue}{bigint}
        \end{ttfamily}
    \item Select all IPv6 flows:\\
        \begin{ttfamily}
            \textcolor{darkblue}{\bf SELECT} to\_hex(\textcolor{red}{"flowStat"}::\textcolor{darkblue}{bigint}), *\\
            \textcolor{darkblue}{\bf FROM} flow\\
            \textcolor{darkblue}{\bf WHERE} (\textcolor{red}{"flowStat"}::\textcolor{darkblue}{bigint} \& x`\textcolor{red}{8000}'::\textcolor{darkblue}{bigint}) != \textcolor{cyan}{0}::\textcolor{darkblue}{bigint}
        \end{ttfamily}
\end{itemize}

\subsection{Example}

{\tt\color{blue} \# Run Tranalyzer}\\
{\tt \$ t2 -r file.pcap}\\

\noindent
{\tt\color{blue} \# Connect to the PostgreSQL database}\\
{\tt \$ psql -U postgres -d tranalyzer}\\

\noindent
{\tt\color{blue} \# Number of flows}\\
{\tt tranalyzer=\# \textcolor{darkblue}{SELECT COUNT}(*) \textcolor{darkblue}{FROM} flow;}\\

\noindent
{\tt\color{blue} \# 10 first srcIP/dstIP pairs}\\
{\tt tranalyzer=\# \textcolor{darkblue}{SELECT} \textcolor{red}{"srcIP"}, \textcolor{red}{"dstIP"} \textcolor{darkblue}{FROM} flow \textcolor{darkblue}{LIMIT} \textcolor{cyan}{10};}\\

\noindent
{\tt\color{blue} \# All flows from 1.2.3.4 to 1.2.3.5}\\
{\tt tranalyzer=\# \textcolor{darkblue}{SELECT} * \textcolor{darkblue}{FROM} flow \textcolor{darkblue}{WHERE} \textcolor{red}{"srcIP"} = '\textcolor{cyan}{1.2.3.4}' \textcolor{darkblue}{AND} \textcolor{red}{"dstIP"} = '\textcolor{cyan}{1.2.3.5}';}\\

\noindent
For examples of more complex queries, have a look in {\tt \$T2HOME/scripts/t2fm/psql/}.

\subsubsection{Clean up an existing database}

{\tt\color{blue} \# Connect to the PostgreSQL database}\\
{\tt \$ psql -U postgres}\\

\noindent
{\tt\color{blue} \# Drop the database}\\
{\tt tranalyzer=\# \textcolor{darkblue}{DROP DATABASE} tranalyzer;}\\

\end{document}
