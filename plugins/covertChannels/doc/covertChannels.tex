\IfFileExists{t2doc.cls}{
    \documentclass[documentation]{subfiles}
}{
    \errmessage{Error: could not find 't2doc.cls'}
}

\begin{document}

\trantitle
    {covertChannels}
    {Detects covert channels in IP traffic}
    {Tranalyzer Development Team} % author(s)

\section{covertChannels}
\label{s:covertChannels}

\subsection{Description}
The covertChannels plugin detects various types of covert channels (CCs) in IP traffic. Currently, it
detects most publicly available covert channel tools. In the future, the goal is to also
detect more discreet covert channels and custom implementations based on current research
(covert timing channels, SkyDe, \ldots). This plugin produces only output to the flow file.
Configuration is achieved by user defined compiler switches in {\tt src/covertChannels.h}.

\subsection{Required Files}

\subsubsection{cc\_dns\_whitelist.txt}

The file {\tt cc\_dns\_whitelist.txt} contains a domain names whitelist for the DNS CCs detection.
Domains in this file will never be flagged as a covert channel.

\begin{itemize}
    \itemsep0em
    \item One domain name per line.
    \item Lines starting with {\tt \%} are comments.
    \item Suffix match is used to compare domain names against the whitelist.
\end{itemize}

\subsubsection{cc\_ping\_whitelist.txt}

The file {\tt cc\_ping\_whitelist.txt} contains a whitelist of PING payloads.
When using the ICMP whitelist detection method, all payload patterns not in this file will be considered
as a covert channel.

\begin{itemize}
    \itemsep0em
    \item Lines starting with {\tt \%} are comments.
    \item One hex encoded pattern per line.
    \item The pattern starts at the 25th byte of the ICMP payload.
    \item Prefix match is used to compare the payload against the whitelist patterns.
\end{itemize}

For instance, to whitelist the PING packet shown in \reff{fig:cc_ping_whitelist}, the whitelist should
contain the following pattern: {\tt 101112131415161718191a1b1c1d1e1f202122232425262728292a2b2c2d2e2f3031323334353637}

\begin{figure}[hb]
  \centering
  \tranimg[width=0.8\textwidth]{wireshark-ping-pattern}
  \caption{Wireshark view of whitelisted PING pattern.}
  \label{fig:cc_ping_whitelist}
\end{figure}

\subsection{Configuration Flags}
The following flags can be used to control the output of the plugin:
\begin{longtable}{>{\tt}lcl}
    \toprule
    {\bf Name} & {\bf Default} & {\bf Description} \\
    \midrule\endhead%
    CC\_DETECT\_DNS        & 1 & detect CCs in DNS traffic\\
    CC\_DETECT\_ICMP\_ASYM & 1 & detect CCs in ICMP traffic (using flow asymmetry)\\
    CC\_DETECT\_ICMP\_WL   & 0 & detect CCs in ICMP traffic (using payload whitelist)\\
    CC\_DETECT\_ICMP\_NP   & 0 & detect CCs in ICMP traffic (bidirectional non-ping flow)\\
    CC\_DETECT\_HCOVERT    & 1 & detect CCs in HTTP GET requests (hcovert)\\
    CC\_DETECT\_DEVCC      & 1 & detect CCs in TCP timestamp field (devcc)\\
    CC\_DETECT\_IPID       & 1 & detect CCs in the IP Identification field (covert\_tcp)\\
    CC\_DETECT\_RTP\_TS    & 0 & detect CCs in the RTP timestamp field\\
    CC\_DETECT\_SKYDE      & 0 & detect CCs in Skype silent packets (SkyDe)\\
    CC\_DEBUG\_MESSAGES    & 0 & activate debug output\\
    \bottomrule
\end{longtable}

\subsection{Flow File Output}
The covertChannels plugin outputs the following column:
\begin{longtable}{>{\tt}lll>{\tt\small}l}
    \toprule
    {\bf Column} & {\bf Type} & {\bf Description} & {\bf Flags}\\
    \midrule\endhead%
    \nameref{covertChannels} & H16 & Detected covert channels bitfield & \\
    \bottomrule
\end{longtable}

\subsubsection{covertChannels}\label{covertChannels}
The covertChannels column is to be interpreted as follows:
\begin{longtable}{>{\tt}rl}
    \toprule
    {\bf covertChannels} & {\bf Description} \\
    \midrule\endhead%
    $2^0$ (=0x0001) & DNS CC (iodine, dnstunnel, nstx, \ldots)\\
    $2^1$ (=0x0002) & ICMP CC: asymmetric flow (hans, itun, loki, icmptx, \ldots)\\
    $2^2$ (=0x0004) & ICMP CC: non-whitelisted payload (hans, itun, loki, icmptx, \ldots)\\
    $2^3$ (=0x0008) & ICMP CC: bidirectional non-PING flow \\
    \\
    $2^4$ (=0x0010) & HTTP GET URL-encoded CC (hcovert)\\
    $2^5$ (=0x0020) & TCP timestamp CC (devcc)\\
    $2^6$ (=0x0040) & IP Identification CC (covert\_tcp)\\
    $2^7$ (=0x0080) & RTP timestamp CC\\
    \\
    $2^8$ (=0x0100) & Skype silent packets CC (SkyDe)\\
    \bottomrule
\end{longtable}

\subsection{Plugin Report Output}
The following information is reported:
\begin{itemize}
    \item Aggregated {\tt\nameref{covertChannels}}
    \item Number of covert channels packets
\end{itemize}

\subsection{TODO}

\begin{itemize}
    \item Smarter IPID covert channels detection (stegtunnel)
    \item SSH/Telnet based covert timing channels detection
\end{itemize}

\end{document}
