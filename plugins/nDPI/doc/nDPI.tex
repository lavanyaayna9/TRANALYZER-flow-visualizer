\IfFileExists{t2doc.cls}{
    \documentclass[documentation]{subfiles}
}{
    \errmessage{Error: could not find t2doc.cls}
}

% from https://tex.stackexchange.com/a/50263
\lstdefinelanguage{diff}{
    morecomment=[f][\color{blue}]{@@},     % group identifier
    morecomment=[f][\color{red}]-,         % deleted lines
    morecomment=[f][\color{green}]+,       % added lines
    morecomment=[f][\color{red}]{---}, % Diff header lines (must appear after +,-)
    morecomment=[f][\color{green}]{+++},
}

\begin{document}

\trantitle
    {nDPI}
    {Classification Based on Content Analysis}
    {Tranalyzer Development Team}

\section{nDPI}\label{s:nDPI}

\subsection{Description}
This plugin is a simple wrapper around the nDPI library: \url{https://github.com/ntop/nDPI}.
It classifies flows according to their protocol/application by analyzing the payload content
instead of using the destination port. This plugin produces output to the flow file and to
a protocol statistics file. Configuration is achieved by user defined compiler switches in
{\tt src/nDPI.h}.

\subsection{Dependencies}

\subsubsection{External Libraries}
This plugin depends on the {\bf libgcrypt} library.
\begin{table}[!ht]
    \centering
    \begin{tabular}{>{\bf}r>{\tt}l>{\tt}l}
        \toprule
        %                             &                      & \\
        %\midrule
        Ubuntu:                      & sudo apt-get install & libgcrypt20-dev\\
        Arch:                        & sudo pacman -S       & libgcrypt      \\
        Gentoo:                      & sudo emerge          & libgcrypt      \\
        openSUSE:                    & sudo zypper install  & libgcrypt-devel\\
        Red Hat/Fedora\tablefootnote{If the {\tt dnf} command could not be found, try with {\tt yum} instead}:
                                     & sudo dnf install     & libgcrypt-devel\\
        macOS\tablefootnote{Brew is a packet manager for macOS that can be found here: \url{https://brew.sh}}:
                                     & brew install         & libgcrypt      \\
        \bottomrule
    \end{tabular}
\end{table}

\subsection{Configuration Flags}
The following flags can be used to control the output of the plugin:
\begin{longtable}{>{\tt}lcl}
    \toprule
    {\bf Variable} & {\bf Default} & {\bf Description} \\
    \midrule\endhead%
    NDPI\_OUTPUT\_NUM    & 0 & Output a numerical classification\\
    NDPI\_OUTPUT\_STR    & 1 & Output a textual classification\\
    NDPI\_OUTPUT\_STATS  & 1 & Output nDPI protocol distribution in a separate file\\
    NDPI\_GUESS\_UNKNOWN & 1 & Try guessing unknown protocols\\
    \bottomrule
\end{longtable}

\subsection{Flow File Output}
The nDPI plugin outputs the following columns:
\begin{longtable}{>{\tt}lll>{\tt\small}l}
    \toprule
    {\bf Column} & {\bf Type} & {\bf Description} & {\bf Flags}\\
    \midrule\endhead%
    \nameref{nDPIMstrProto} & U16 & nDPI numerical master protocol     & NDPI\_OUTPUT\_NUM=1\\
    nDPISubProto            & U16 & nDPI numerical sub protocol        & NDPI\_OUTPUT\_NUM=1\\
    nDPIclass               & S   & nDPI based protocol classification & NDPI\_OUTPUT\_STR=1\\
    \bottomrule
\end{longtable}

\subsection{Packet File Output}
In packet mode ({\tt --s} option), the nDPI plugin outputs the following columns:
\begin{longtable}{>{\tt}lll>{\tt\small}l}
    \toprule
    {\bf Column} & {\bf Type} & {\bf Description} & {\bf Flags}\\
    \midrule\endhead%
    \nameref{nDPIMstrProto} & U16 & nDPI numerical master protocol     & NDPI\_OUTPUT\_NUM=1\\
    nDPISubProto            & U16 & nDPI numerical sub protocol        & NDPI\_OUTPUT\_NUM=1\\
    nDPIclass               & S   & nDPI based protocol classification & NDPI\_OUTPUT\_STR=1\\
    \bottomrule
\end{longtable}

\subsection{nDPIMstrProto}\label{nDPIMstrProto}
The {\tt nDPIMstrProto} column is to be interpreted as follows:
\traninput{proto}

\subsection{Plugin Report Output}
The following information is reported:
\begin{itemize}
    \item Number of flows classified
\end{itemize}

\subsection{Additional Output}
If {\tt NDPI\_OUTPUT\_STATS=1} then nDPI protocol distribution statistics are output in {\tt PREFIX\_nDPI.txt}.\\

\subsection{Post-Processing}
The {\tt\tranref{protStat}} script can be used to sort the {\tt PREFIX\_nDPI.txt} file for the most or least occurring protocols (in terms of number of packets or bytes).
It can output the top or bottom $N$ protocols or only those with at least a given percentage:
\begin{itemize}
    \item list all the options: {\tt protStat --{}--help}
    \item for better readability, use {\tt protStat} with {\tt tcol}: {\tt protStat ... | tcol}
    \item sorted list of protocols (by packets): {\tt protStat PREFIX\_nDPI.txt}
    \item sorted list of protocols (by bytes): {\tt protStat PREFIX\_nDPI.txt --b}
    \item top 10 protocols (by packets): {\tt protStat PREFIX\_nDPI.txt --n 10}
    \item bottom 5 protocols (by bytes): {\tt protStat PREFIX\_nDPI.txt --n --5 --b}
    \item protocols with packets percentage greater than 20\%: {\tt protStat PREFIX\_nDPI.txt --p 20}
    \item protocols with bytes percentage smaller than 5\%: {\tt protStat PREFIX\_nDPI.txt --b --p --5}
\end{itemize}

\subsection{How to Update nDPI to New Version}

\begin{itemize}
    \item download latest stable version (or git clone and checkout stable branch)
    \item delete {\tt src/nDPI} and replace it with this new version
    \item run the {\tt ./new\_ndpi\_prepatch.sh} script
    \item build the nDPI plugin: {\tt t2build -r nDPI}
    \item Replace the {\tt proto.tex} file using the {\tt prototex} utility and regenerate doc:
        \begin{center}
            {\tt make -C prototex \&\& ./prototex/prototex > doc/proto.tex}
        \end{center}
    \item Add the new files to SVN and delete removed files before commit.
\end{itemize}

\end{document}
