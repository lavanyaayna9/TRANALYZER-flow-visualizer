\IfFileExists{t2doc.cls}{
    \documentclass[documentation]{subfiles}
}{
    \errmessage{Error: could not find 't2doc.cls'}
}

\begin{document}

\trantitle
    {tp0f} % Plugin name
    {OS/Application Fingerprinting based on layer 3/4 (IP/TCP) analysis} % Short description
    {Tranalyzer Development Team} % author(s)

\section{tp0f}\label{s:tp0f}

\subsection{Description}
The tp0f plugin classifies IP addresses according to OS type and version.
It uses initial TTL and window size and can also use the rules from p0f.
%With additional HTTP and HTTPS rules programs such as browser versions can also be classified.
%At compilation a script {\tt tp0fL34conv} converts the supplied p0f file into a T2 readable file
%defined by {\tt TP0F\_L34FILE} in {\em tp0f.h}.
In order to label non-TCP flows, the plugin can store a hash of already classified IP addresses.

\subsubsection{Required Files}
If {\tt TP0FRULES=1}, then the file {\tt tp0fL34.txt} is required.

\subsection{Configuration Flags}
The following flags can be used to control the output of the plugin:

\begin{longtable}{>{\tt}lcl}
    \toprule
    {\bf Name}    & {\bf Default}             & {\bf Description}\\
    \midrule\endhead%
    TP0FRULES     & 1                         & 0: Standard OS guessing\\
                  &                           & 1: OS guessing and p0f L3/4 rules\\
    TP0FHSH       & 1                         & 0: No IP hash\\
                  &                           & 1: IP hash to recognize IP already classified\\
    TP0FRC        & 0                         & 0: Only human readable\\
                  &                           & 1: p0f rule and classifier numbers\\
    TP0F\_L34FILE & {\tt\small "tp0fL34.txt"} & File containing converted L3/4 rules\\
    \bottomrule
\end{longtable}

In {\em tp0flist.h}:

\begin{longtable}{>{\tt}lcl}
    \toprule
    {\bf Name} & {\bf Default} & {\bf Description}\\
    \midrule\endhead%
    TCPOPTMAX  & 40            & Maximal TCP option codes to store and process\\
    \bottomrule
\end{longtable}

\subsubsection{Environment Variable Configuration Flags}
The following configuration flags can also be configured with environment variables ({\tt ENVCNTRL>0}):
\begin{itemize}
    \item {\tt TP0F\_L34FILE}
\end{itemize}

\subsection{Flow File Output}
The p0f plugin outputs the following columns:
\begin{longtable}{>{\tt}lll>{\tt\small}l}
    \toprule
    {\bf Column} & {\bf Type} & {\bf Description} & {\bf Flags}\\
    \midrule\endhead%
    \nameref{tp0fStat} & H8  & Status                         & \\
    tp0fDis            & U8  & Initial TTL distance           & \\
    tp0fRN             & U16 & Rule number that triggered     & TP0FRC=1\\
    tp0fClass          & U8  & OS class of rule file          & TP0FRC=1\\
    tp0fProg           & U8  & Program category of rule file  & TP0FRC=1\\
    tp0fVer            & U8  & Version category of rule file  & TP0FRC=1\\
    tp0fClName         & SC  & OS class name                  & \\
    tp0fPrName         & SC  & OS/program name                & \\
    tp0fVerName        & SC  & OS/program version name        & \\
    \bottomrule
\end{longtable}

\subsubsection{tp0fStat}\label{tp0fStat}
The {\tt tp0fStat} column is to be interpreted as follows:
\begin{longtable}{>{\tt}rl}
    \toprule
    {\bf tp0fStat} & {\bf Description}\\
    \midrule\endhead%
    0x01 & SYN tp0f rule fired\\
    0x02 & SYN-ACK tp0f rule fired\\
    0x04 & ---\\
    0x08 & ---\\
    \\
    0x10 & ---\\
    0x20 & ---\\
    0x40 & IP already seen by tp0f\\
    0x80 & TCP option length or content corrupt \\
    \bottomrule
\end{longtable}

\subsection{Plugin Report Output}
The number of packets which fired a tp0f rule is reported.

%\subsection{Example Output}

%\subsection{Known Bugs and Limitations}

\subsection{TODO}
\begin{itemize}
    \item Integrate TLS rules
    \item Integrate HTTP rules
\end{itemize}

\subsection{References}
\begin{itemize}
    \item \url{http://www.netresec.com/?page=Blog&month=2011-11&post=Passive-OS-Fingerprinting}
    \item \url{http://lcamtuf.coredump.cx/p0f3/}
\end{itemize}

\end{document}
