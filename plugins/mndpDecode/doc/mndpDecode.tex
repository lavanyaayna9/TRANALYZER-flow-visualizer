\IfFileExists{t2doc.cls}{
    \documentclass[documentation]{subfiles}
}{
    \errmessage{Error: could not find 't2doc.cls'}
}

\begin{document}

% Declare author and title here, so the main document can reuse it
\trantitle
    {mndpDecode} % Plugin name
    {MikroTik Neighbor Discovery Protocol (MNDP)} % Short description
    {Tranalyzer Development Team} % author(s)

\section{mndpDecode}\label{s:mndpDecode}

\subsection{Description}
The mndpDecode plugin analyzes MNDP traffic.

\subsection{Configuration Flags}
The following flags can be used to control the output of the plugin:
\begin{longtable}{>{\tt}lcl>{\tt\small}l}
    \toprule
    {\bf Name}   & {\bf Default} & {\bf Description}                & {\bf Flags}\\
    \midrule\endhead%
    MNDP\_DEBUG  & 0             & Print debug messages             & \\
    MNDP\_LSTLEN & 5             & Max number of elements for lists & \\
    MNDP\_STRLEN & 32            & Max length for strings           & \\
    \bottomrule
\end{longtable}

\subsection{Flow File Output}
The mndpDecode plugin outputs the following columns:
\begin{longtable}{>{\tt}lll>{\tt\small}l}
    \toprule
    {\bf Column}       & {\bf Type} & {\bf Description}       & {\bf Flags}\\
    \midrule\endhead%
    \nameref{mndpStat} & H8         & Status                  & \\
    mndpMAC            & R(MAC)     & MAC-Address             & MNDP\_LSTLEN>0\\
    mndpIdentity       & R(STR)     & Identity                & MNDP\_LSTLEN>0\\
    mndpVersion        & R(STR)     & Version                 & MNDP\_LSTLEN>0\\
    mndpPlatform       & R(STR)     & Platform                & MNDP\_LSTLEN>0\\
    %mndpUptime         & R(U32)     & Uptime (seconds)        & MNDP\_LSTLEN>0\\
    mndpSoftwareID     & R(STR)     & Software-ID             & MNDP\_LSTLEN>0\\
    mndpBoard          & R(STR)     & Board                   & MNDP\_LSTLEN>0\\
    mndpUnpack         & R(U8)      & Packet compression type & MNDP\_LSTLEN>0\\
    mndpIface          & R(STR)     & Interface name          & MNDP\_LSTLEN>0\\
    mndpIPv4           & R(IP4)     & IPv4-Address            & MNDP\_LSTLEN>0\\
    mndpIPv6           & R(IP6)     & IPv6-Address            & MNDP\_LSTLEN>0\\
    \bottomrule
\end{longtable}

\subsubsection{mndpStat}\label{mndpStat}
The {\tt mndpStat} column is to be interpreted as follows:
\begin{longtable}{>{\tt}rl}
    \toprule
    {\bf mndpStat} & {\bf Description}\\
    \midrule\endhead%
    0x0\textcolor{magenta}{1} & Flow is MNDP\\
    0x0\textcolor{magenta}{2} & IPv4 address\\
    0x0\textcolor{magenta}{4} & IPv6 address\\
    0x0\textcolor{magenta}{8} & Unknown TLV type\\
    \\
    0x\textcolor{magenta}{1}0 & Invalid TLV length, e.g., length of MAC address > 6\\
    0x\textcolor{magenta}{2}0 & List was truncated... increase {\tt MNDP\_LSTLEN}\\\\
    0x\textcolor{magenta}{4}0 & String was truncated... increase {\tt MNDP\_STRLEN}\\
    0x\textcolor{magenta}{8}0 & Packet was snapped\\
    \bottomrule
\end{longtable}

\subsection{Packet File Output}
In packet mode ({\tt --s} option), the mndpDecode plugin outputs the following columns:
\begin{longtable}{>{\tt}lll>{\tt\small}l}
    \toprule
    {\bf Column}       & {\bf Type} & {\bf Description} & {\bf Flags}\\
    \midrule\endhead%
    \nameref{mndpStat} & H8  & Status\\
    mndpSeqNo          & U16 & Sequence Number\\
    mndpMAC            & STR & MAC-Address\\
    mndpIdentity       & STR & Identity\\
    mndpVersion        & STR & Version\\
    mndpPlatform       & STR & Platform\\
    mndpUptime         & U32 & Uptime (seconds)\\
    mndpSoftwareID     & STR & Software-ID\\
    mndpBoard          & STR & Board\\
    mndpUnpack         & U8  & Packet compression type\\
    mndpIface          & STR & Interface name\\
    mndpIPv4           & IP4 & IPv4-Address\\
    mndpIPv6           & IP6 & IPv6-Address\\
    \bottomrule
\end{longtable}

\subsection{Plugin Report Output}
The following information is reported:
\begin{itemize}
    \item Aggregated {\tt\nameref{mndpStat}}
    \item Number of MNDP packets
\end{itemize}

\end{document}
