\IfFileExists{t2doc.cls}{%
    \documentclass[documentation]{subfiles}
}{
    \errmessage{Error: could not find t2doc.cls}
}

\begin{document}

\trantitle
    {wavelet} % plugin name
    {Wavelet} % short description
    {Tranalyzer Development Team}

\section{wavelet}\label{s:wavelet}

\subsection{Description}
The wavelet plugin calculates the Daubechies wavelet transformation of the IP packet length or the inter-arrival time of packets (IAT).

\subsection{Configuration Flags}
The following flags, defined in {\em define\_global.h}, can be used to control the output of the plugin:
\begin{longtable}{>{\tt}lcl}
    \toprule
    {\bf Name} & {\bf Default} & {\bf Description} \\
    \midrule\endhead%
    WAVELET\_IAT      & 0               & Values to analyze:\\
                      &                 & \qquad 0: Packet length,\\
                      &                 & \qquad 1: Inter-arrival times (IAT)\\
    WAVELET\_SIG      & 0               & Print signal\\
    WAVELET\_PREC     & 0               & Precision:\\
                      &                 & \qquad 0: Float,\\
                      &                 & \qquad 1: Double\\
    WAVELET\_THRES    & 8               & Min. number of packets for analysis\\
    WAVELET\_MAX\_PKT & 40              & Max. number of selected packets\\
    WAVELET\_LEVEL    & 3               & Wavelet decomposition level\\
    WAVELET\_EXTMODE  & {\tt\small ZPD} & Extension Mode:\\
                      &                 & \qquad {\tt NON}: No extension,\\
                      &                 & \qquad {\tt SYM}: Symmetrization,\\
                      &                 & \qquad {\tt ZPD}: Zero-padding\\
    WAVELET\_TYPE     & {\tt\small DB3} & Mother Wavelet:\\
                      &                 & \qquad {\tt DB1}: Daubechies 1 wavelet\\
                      &                 & \qquad {\tt DB2}: Daubechies 2 wavelet\\
                      &                 & \qquad {\tt DB3}: Daubechies 3 wavelet\\
                      &                 & \qquad {\tt DB4}: Daubechies 4 wavelet\\
    \bottomrule
\end{longtable}

\subsection{Flow File Output}
The wavelet plugin outputs the following columns:
\begin{longtable}{>{\tt}lll>{\tt\small}l}
    \toprule
    {\bf Name} & {\bf Type} & {\bf Description} & {\bf Flags}\\
    \midrule\endhead%
    waveNumPnts    & U16       & Number of points                   & \\
    waveSig        & R(F/D)    & Packet length / IAT signal         & WAVELET\_PREC=0/1\\
    waveNumLvl     & U32       & Number of wavelet levels           & \\
    waveCoefDetail & R(R(F/D)) & Wavelet detail coefficients        & WAVELET\_PREC=0/1\\
    waveCoefApprox & R(R(F/D)) & Wavelet approximation coefficients & WAVELET\_PREC=0/1\\
    \bottomrule
\end{longtable}

\end{document}
