\IfFileExists{t2doc.cls}{
    \documentclass[documentation]{subfiles}
}{
    \errmessage{Error: could not find t2doc.cls}
}

\begin{document}

\trantitle
    {regex\_pcre}
    {Perl Compatible Regular Expressions (PCRE)}
    {Tranalyzer Development Team}

\section{regex\_pcre}\label{s:regex_pcre}

\subsection{Description}
The regex\_pcre plugin provides a full PCRE compatible regex engine.

\subsection{Dependencies}

\subsubsection{External Libraries}
This plugin depends on the {\bf pcre} library.
\begin{table}[!ht]
    \centering
    \begin{tabular}{>{\bf}r>{\tt}l>{\tt}l>{\tt}l}
        \toprule
        %                             &                      &                 \\
        %\midrule
        Ubuntu:                      & sudo apt-get install & libpcre3-dev\\
        Arch:                        & sudo pacman -S       & pcre pcre2\\
        %Gentoo:                      & sudo emerge          & \\
        openSUSE:                    & sudo zypper install  & pcre-devel\\
        Red Hat/Fedora\tablefootnote{If the {\tt dnf} command could not be found, try with {\tt yum} instead}:
                                     & sudo dnf install     & pcre-devel\\
        macOS\tablefootnote{Brew is a packet manager for macOS that can be found here: \url{https://brew.sh}}:
                                     & brew install         & pcre\\
        \bottomrule
    \end{tabular}
\end{table}

\subsubsection{Required Files}
The file {\tt regexfile.txt} is required (automatically generated from {\tt scripts/regfile.txt}).
Refer to \refs{regexfile.txt} for more details.

\subsection{Configuration Flags}

\subsubsection{regfile\_pcre.h}
The compiler constants in {\em regfile\_pcre.h} control the pre-processing and
compilation of the rule sets supplied in the regex file during the initialization phase of Tranalyzer.

\begin{longtable}{>{\tt}lcl>{\tt\small}l}
    \toprule
    {\bf Name}     & {\bf Default}            & {\bf Description}                                & {\bf Flags}\\
    \midrule\endhead%
    RULE\_OPTIMIZE & 1                        & 0: No opt rules allocated                        & \\
                   &                          & 1: Allocate opt rule structure and compile regex & \\
    REGEX\_MODE    & {\tt\small PCRE\_DOTALL} & Regex compile time options                       & \\
    PREIDMX        & 4                        & Max number of node predecessors                  & \\
    \bottomrule
\end{longtable}

\subsubsection{regex\_pcre.h}
The compiler constants in {\em regex\_pcre.h} control the execution and
the output the rule matches.

\begin{longtable}{>{\tt}lcl>{\tt\small}l}
    \toprule
    {\bf Variable}   & {\bf Default} & {\bf Description}                                        & {\bf Flags}\\
    \midrule\endhead%
    EXPERTMODE       &  0            & 0: Alarm with highest severity: class type and severity, & \\
                     &               & 1: full info                                             & \\
    PKTTIME          &  0            & 0: no time, 1: timestamp when rule matched               & \\
    AGGR             &  0            & 1: Aggregate alarms                                      & \\
    SALRMFLG         &  0            & 1: enable sending {\tt FL\_ALARM} for \tranrefpl{pcapd}  & \\
    MAXREGPOS        & 30            & Maximal \# of matches stored / flow                      & \\
    RGX\_POSIX\_FILE & {\tt\small "regexfile.txt"}
                                     & Name of regex file under {\em ./tranalyzer/plugins}      & \\
    OVECCOUNT        & 3             & Value \% 3                                               & \\
    \bottomrule
\end{longtable}

\subsubsection{Environment Variable Configuration Flags}
The following configuration flags can also be configured with environment variables ({\tt ENVCNTRL>0}):
\begin{itemize}
    \item {\tt RGX\_POSIX\_FILE}
\end{itemize}

\subsubsection{regexfile.txt}\label{regexfile.txt}
The {\em scripts/regexfile.txt} file has the following format:
\begin{center}
\begin{lstlisting}
#ID     PreID   Flags   ClassID Severity        Sel     Regexmode       FlwStat Proto   srcPort dstPort offset  Regex

# standalone rule: Alarm, start L7, Regexmode: default, select FlwStat: Req; Proto, dstPort
1       0       0x10    15      3       0x8000000d      0x0000000       0x00000000      6       0       80      0       (OPTIONS|GET|HEAD|POST|PUT|DELETE|TRACE|CONNECT)[^\r\n]*\/u7avi*\.bin

# standalone rule: Alarm, disabled, start L7, select Regexmode: (PCRE_CASELESS|PCRE_DOTALL), FlwStat: Teredo, IPv6, Vlan, Repl; Proto, srcPort
3       0       0x10    15      3       0x0800000e      0x0000005       0x00088101      6       80      0       0       \x31\xDB\x8D\x43\x0D\xCD\x80\x66.*\x31

# standalone rule: Alarm, start L7, Regexmode: default, FlwStat: IPv4, Rply
4       0       0x10    15      3       0x8000000c      0x0000000       0x00004001      6       80      0       20      \x38\x55\x42\x66\xe2\xb5\x34.*\xb5\x95\xbb

# standalone rule, Alarm, start L7, select Regexmode: (PCRE_CASELESS|PCRE_DOTALL)
100     0       0x10    1       0       0x88000000      0x0000005       0x00000000      6       0       80      0       ^http/1.0

# root rules to following tree, Reset if leaf fires
202     0       0x40    10      4       0x80000000      0x0000000       0x00000001      6       0       80      0       (GET|PUT).*update/u7avi1777u1705ff.bin
203     202,4   0x41    20      4       0x88000000      0x0000005       0x00000001      6       0       80      0       302 (?i)Found

# successors and predecessors, Reset if leaf fires
204     202,203 0x41    43      5       0x80000000      0x0000000       0x00000001      6       0       21      0       (?i)\.exe

# successors 206 & 205 to 204 AND ruleset, don't reset tree if 205 fires
205     204     0x16    40      4       0x80000002      0x0000000       0x00000000      6       0       20      0       ^get .*porno.*
206     204     0x56    35      6       0x8000000c      0x0000000       0x00000001      6       0       21      0       igfxzoom\.exe
\end{lstlisting}
\end{center}

Lines starting with a {\tt '\#'} denote a comment line and will be ignored.
All kind of rule trees can be formed using rules also acting on multiple packets using different {\tt ID}'s and {\tt Predecessor} as outlined in the example above.
Regex rules with the same {\tt ID} denote combined predecessors to other rules. Default is an OR operation unless {\tt ANDPin} bits are set. These bits denote the different inputs to a bitwise AND. The output is then provided
to the successor rule which compares with the {\tt ANDMask} bit field whether all necessary rules are matched.
Then an evaluation of the successor rule can take place. Thus, arbitrary rule trees can be constructed and results of
predecessors can be used for multiple successor rules. The variable {\tt Flags} controls the basic PCRE rule interpretation and the flow alarm production (see the table below), e.g. only if bit eight is set and alarm flow output is produced. {\tt ClassID} and {\tt Severity} denote information being printed in the flow file if the rule fires.

\begin{longtable}{>{\tt}rl}
    \toprule
    {\bf Flags} & {\bf Description}\\
    \midrule\endhead%
    $2^0$ (=0x01) & Predecessor OPS\\
    $2^1$ (=0x02) & Predecessor OPS\\
    $2^2$ (=0x04) & Leaf\\
    $2^3$ (=0x08) & --- \\
    \\
    $2^4$ (=0x10) & Print alarm to flow file\\
    $2^5$ (=0x20) & Rule active only in flow boundary\\
    $2^6$ (=0x40) & Reset {\tt REG\_F\_MTCH} tree if match\\
    $2^7$ (=0x80) & Internal: Regex match\\
    \bottomrule
\end{longtable}

\begin{longtable}{>{\tt}rll}
    \toprule
    {\bf Predecessor OPS} & {\bf OP} & {\bf Description}\\
    \midrule\endhead%
    0x00 & NONE & None, solitary rule\\
    0x01 & AND  & {\tt and(pred1, pred2, ...)}\\
    0x02 & OR   & {\tt or(pred1, pred2, ...)}\\
    0x03 & XOR  & {\tt xor(pred1, pred2, ...)}\\
    \bottomrule
\end{longtable}

The {\tt Sel} column controls the header selection of a rule in the lower nibble and the start of regex evaluation
in the higher nibble. The position of the bits in the control byte are outlined below:

\begin{longtable}{>{\tt}rl}
    \toprule
    {\bf Sel} & {\bf Description} \\
    \midrule\endhead%
    $2^{0}$  (=0x00000001) & Activate srcPort field\\
    $2^{1}$  (=0x00000002) & Activate dstPort field\\
    $2^{2}$  (=0x00000004) & Activate L4Proto field\\
    $2^{3}$  (=0x00000008) & Activate flowStat field\\
    \\
    $2^{27}$ (=0x08000000) & PCRE mode active; otherwise default\\
    \\
    $2^{28}$ (=0x10000000) & Header start: Layer 2\\
    $2^{29}$ (=0x20000000) & Header start: Layer 3\\
    $2^{30}$ (=0x40000000) & Header start: Layer 4\\
    $2^{31}$ (=0x80000000) & Header start: Layer 7\\
    \bottomrule
\end{longtable}

Bit 0 - 27 selects the first 32 bit of {\tt flowStat}, the protocol, source and destination port will be
evaluated per rule, all others will be ignored. The {\tt flowStat} field might contain other bits meaning more selection
options in future.
The {\tt offset} column depicts the start of the regex evaluation from the selected header start, default value 0.
The {\tt Regex} column accepts a full PCRE regex term. If the regex is not correct, the rule will be discarded
displaying an error message in the Tranalyzer report.

The {\tt regexmode} column denotes the mode of regex compilation and execution, listed below.
If {\bf 0x00000000} then the default defined by {\tt REGEX\_MODE} is used.

\begin{longtable}{>{\tt}r>{\tt}ll}
    \toprule
    {\bf regexmode} & {\bf Name} & {\bf Description} \\
    \midrule\endhead%
    $2^{0}$  (=0x00000001) & PCRE\_CASELESS            & Compile\\
    $2^{1}$  (=0x00000002) & PCRE\_MULTILINE           & Compile\\
    $2^{2}$  (=0x00000004) & PCRE\_DOTALL              & Compile\\
    $2^{3}$  (=0x00000008) & PCRE\_EXTENDED            & Compile\\
    \\
    $2^{4}$  (=0x00000010) & PCRE\_ANCHORED            & Compile, DFA exec\\
    $2^{5}$  (=0x00000020) & PCRE\_DOLLAR\_ENDONLY     & Compile\\
    $2^{6}$  (=0x00000040) & PCRE\_EXTRA               & Compile\\
    $2^{7}$  (=0x00000080) & PCRE\_NOTBOL              & Exec, DFA exec\\
    \\
    $2^{8}$  (=0x00000100) & PCRE\_NOTEOL              & Exec, DFA exec\\
    $2^{9}$  (=0x00000200) & PCRE\_UNGREEDY            & Compile\\
    $2^{10}$ (=0x00000400) & PCRE\_NOTEMPTY            & Exec, DFA exec\\
    $2^{11}$ (=0x00000800) & PCRE\_UTF8                & Compile\\
    \\
    $2^{12}$ (=0x00001000) & PCRE\_NO\_AUTO\_CAPTURE   & Compile\\
    $2^{13}$ (=0x00002000) & PCRE\_NO\_UTF8\_CHECK     & Compile, DFA exec\\
    $2^{14}$ (=0x00004000) & PCRE\_AUTO\_CALLOUT       & Compile\\
    $2^{15}$ (=0x00008000) & PCRE\_PARTIAL\_SOFT       & Exec, DFA exec\\
    \\
    $2^{16}$ (=0x00010000) & PCRE\_DFA\_SHORTEST       & DFA exec\\
    $2^{17}$ (=0x00020000) & PCRE\_DFA\_RESTART        & DFA exec\\
    $2^{18}$ (=0x00040000) & PCRE\_FIRSTLINE           & Compile\\
    $2^{19}$ (=0x00080000) & PCRE\_DUPNAMES            & Compile\\
    \\
    $2^{20}$ (=0x00100000) & PCRE\_NEWLINE\_CR         & Compile, DFA exec\\
    $2^{21}$ (=0x00200000) & PCRE\_NEWLINE\_LF         & Compile, DFA exec\\
    $2^{22}$ (=0x00400000) & PCRE\_NEWLINE\_ANY        & Compile, DFA exec\\
    $2^{23}$ (=0x00800000) & PCRE\_BSR\_ANYCRLF        & Compile, DFA exec\\
    \\
    $2^{24}$ (=0x01000000) & PCRE\_BSR\_UNICODE        & Compile, DFA exec\\
    $2^{25}$ (=0x02000000) & PCRE\_JAVASCRIPT\_COMPAT  & Compile\\
    $2^{26}$ (=0x04000000) & PCRE\_NO\_START\_OPTIMIZE & Compile, DFA exec\\
    $2^{27}$ (=0x08000000) & PCRE\_PARTIAL\_HARD       & Exec, DFA exec\\
    \\
    $2^{28}$ (=0x10000000) & PCRE\_NOTEMPTY\_ATSTART   & Exec, DFA exec\\
    $2^{29}$ (=0x20000000) & PCRE\_UCP                 & Compile\\
    \bottomrule
\end{longtable}

\subsection{Flow File Output}
The regex\_pcre plugin outputs the following columns:
\begin{longtable}{>{\tt}lll>{\tt\small}l}
    \toprule
    {\bf Column name} & {\bf Type} & {\bf Description} & {\bf Flags}\\
    \midrule\endhead%
    rgxCnt                  & U16                  & Number of regex alarms                        & \\
    rgxRID\_cType\_sev      & R(U16\_U8\_U8)       & Regex ID, class type and severity             & EXPERTMODE=0\\
    \\
    \multicolumn{4}{l}{If {\tt EXPERTMODE=1}, the following columns are displayed:}\\
    \\
    rgxRID\_cType\_sev\_    & R(U16\_U8\_U8\_      & Regex ID, class type, severity,               & PKTTIME=0\\
    \qquad pktN\_bPos       & \qquad U32\_U16)     & \qquad packet number and byte position        & \\
    rgxRID\_cType\_sev\_    & R(U16\_U8\_U8\_      & Regex ID, class type, severity,               & PKTTIME=1\\
    \qquad pktN\_bPos\_time & \qquad U32\_U16\_TS) & \qquad packet number, byte position and time) & \\
    \bottomrule
\end{longtable}

\subsection{Packet File Output}
In packet mode ({\tt --s} option), the regex\_pcre plugin outputs the following columns:
\begin{longtable}{>{\tt}lll>{\tt\small}l}
    \toprule
    {\bf Column} & {\bf Type} & {\bf Description} & {\bf Flags}\\
    \midrule\endhead%
    rgxCnt                  & U16                  & Number of regex alarms                        & \\
    rgxRID\_cType\_sev      & R(U16\_U8\_U8)       & Regex ID, class type and severity             & \\
    \bottomrule
\end{longtable}

\subsection{Plugin Report Output}
The following information is reported:
\begin{itemize}
    \item Number of alarms in number of flows with max severity
\end{itemize}

\end{document}
