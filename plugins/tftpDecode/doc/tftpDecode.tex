\IfFileExists{t2doc.cls}{
    \documentclass[documentation]{subfiles}
}{
    \errmessage{Error: could not find 't2doc.cls'}
}

\begin{document}

\trantitle
    {tftpDecode}
    {Trivial File Transfer Protocol (TFTP)}
    {Tranalyzer Development Team} % author(s)

\section{tftpDecode}\label{s:tftpDecode}

\subsection{Description}
The {\tt tftpDecode} plugin analyzes TFTP traffic.
User defined compiler switches are in {\em tftpDecode.h}.

\subsection{Configuration Flags}
The following flags can be used to control the output of the plugin:
\begin{longtable}{>{\tt}lcl>{\tt\small}l}
    \toprule
    {\bf Name} & {\bf Default} & {\bf Description} & {\bf Flags}\\
    \midrule\endhead%
    TFTP\_SAVE      &  0 & Save content to {\tt\small TFTP\_F\_PATH}       & \\
    TFTP\_RMDIR     &  1 & Empty {\tt\small TFTP\_F\_PATH} before starting & TFTP\_SAVE=1\\
    TFTP\_CMD\_AGGR &  1 & Aggregate TFTP commands/errors                  & \\
    TFTP\_BTFLD     &  1 & Bitfield coding of TFTP commands                & \\
    TFTP\_MXNMLN    & 15 & Maximal name length                             & \\
    TFTP\_MAXCNM    &  2 & Maximal length of command field                 & \\
    TFTP\_F\_PATH   & {\tt\small "/tmp/TFTPFILES/"}
                         & Path for extracted content                      & \\
    \bottomrule
\end{longtable}

\subsubsection{Environment Variable Configuration Flags}
The following configuration flags can also be configured with environment variables ({\tt ENVCNTRL>0}):
\begin{itemize}
    \item {\tt TFTP\_RMDIR}
    \item {\tt TFTP\_F\_PATH}
\end{itemize}

\subsection{Flow File Output}
The tftpDecode plugin outputs the following columns:
\begin{longtable}{>{\tt}lll>{\tt\small}l}
    \toprule
    {\bf Column} & {\bf Type} & {\bf Description} & {\bf Flags}\\
    \midrule\endhead%
    \nameref{tftpStat}                & H16  & Status               & \\
    tftpPFlow                         & U64  & Parent flow          & \\
    \hyperref[tftpOpCBF]{tftpOpCBF}   & H8   & Opcode bitfield      & TFTP\_BITFIELD=1\\
    \hyperref[tftpErrCBF]{tftpErrCBF} & H8   & Error code bitfield  & TFTP\_BITFIELD=1\\
    tftpNumOpcode                     & U8   & Number of opcodes    & \\
    \hyperref[tftpOpCBF]{tftpOpcode}  & RSC  & Opcodes              & TFTP\_MAXCNM>0\\
    tftpNumParam                      & U8   & Number of parameters & \\
    tftpParam                         & RS   & Parameters           & TFTP\_MAXCNM>0\\
    tftpNumError                      & U8   & Number of errors     & \\
    \hyperref[tftpErrCBF]{tftpErrC}   & RU16 & Error codes          & TFTP\_MAXCNM>0\\
    \bottomrule
\end{longtable}

\subsubsection{tftpStat}\label{tftpStat}
The {\tt tftpStat} column is to be interpreted as follows:
\begin{longtable}{>{\tt}rl>{\tt\small}l}
    \toprule
    {\bf tftpStat}     & {\bf Description}                                        & {\bf Flags}\\
    \midrule\endhead%
    $2^{0}$  (=0x0001) & TFTP flow found                                          & \\
    $2^{1}$  (=0x0002) & TFTP data read                                           & \\
    $2^{2}$  (=0x0004) & TFTP data write                                          & \\
    $2^{3}$  (=0x0008) & File open error                                          & TFTP\_SAVE=1\\
    \\
    $2^{4}$  (=0x0010) & Error in block send sequence                             & \\
    $2^{5}$  (=0x0020) & Error in block ack sequence                              & \\
    $2^{6}$  (=0x0040) & Error or TFTP protocol error or not TFTP                 & \\
    $2^{7}$  (=0x0080) & Array overflow\ldots increase {\tt TFTP\_MAXCNM}         & \\
    \\
    $2^{8}$  (=0x0100) & String truncated\ldots increase {\tt TFTP\_MXNMLN}       & \\
    $2^{9}$  (=0x0200) & ---                                                      & \\
    $2^{10}$ (=0x0400) & ---                                                      & \\
    $2^{11}$ (=0x0800) & Crafted packet or TFTP read/write parameter length error & \\
    \\
    $2^{12}$ (=0x1000) & TFTP active                                              & \\
    $2^{13}$ (=0x2000) & TFTP passive                                             & \\
    $2^{14}$ (=0x4000) & ---                                                      & \\
    $2^{15}$ (=0x8000) & ---                                                      & \\
    \bottomrule
\end{longtable}

\subsubsection{tftpOpcode and tftpOpCBF}\label{tftpOpCBF}
The {\tt tftpOpCBF} column is to be interpreted as follows:
\begin{longtable}{>{\tt}r>{\tt}cl}
    \toprule
    {\bf tftpOpCBF} & {\bf tftpOpcode} & {\bf Description} \\
    \midrule\endhead%
    $2^0$ (=0x01)   & RRQ              & Read request \\
    $2^1$ (=0x02)   & WRQ              & Write request \\
    $2^2$ (=0x04)   & DTA              & Read or write the next block of data \\
    $2^3$ (=0x08)   & ACK              & Acknowledgment \\
    $2^4$ (=0x10)   & ERR              & Error message \\
    $2^5$ (=0x20)   & OAK              & Option acknowledgment \\
    $2^6$ (=0x40)   & --{}--{}--       & --- \\
    $2^7$ (=0x80)   & --{}--{}--       & --- \\
    \bottomrule
\end{longtable}

\subsubsection{tftpErrC and tftpErrCBF}\label{tftpErrCBF}
The {\tt tftpErrCBF} column is to be interpreted as follows:
\begin{longtable}{c>{\tt}rl}
    \toprule
    {\bf tftpErrC} & {\bf tftpErrCBF} & {\bf Description} \\
    \midrule\endhead%
       & 0x00 & No Error \\
     0 & 0x01 & File not found \\
     1 & 0x02 & Access violation \\
     2 & 0x04 & Disk full or allocation exceeded \\
     3 & 0x08 & Illegal TFTP operation \\
     4 & 0x10 & Unknown transfer ID \\
     5 & 0x20 & File already exists \\
     6 & 0x40 & No such user \\
     7 & 0x80 & Terminate transfer due to option negotiation \\
    \bottomrule
\end{longtable}

\clearpage

\subsection{Packet File Output}
In packet mode ({\tt --s} option), the tftpDecode plugin outputs the following columns:
\begin{longtable}{>{\tt}lll>{\tt\small}l}
    \toprule
    {\bf Column} & {\bf Type} & {\bf Description} & {\bf Flags}\\
    \midrule\endhead%
    \hyperref[tftpOpCBF]{tftpOpcode} & SC & TFTP opcode & \\
    \bottomrule
\end{longtable}

\subsection{Plugin Report Output}
The following information is reported:
\begin{itemize}
    \item Aggregated {\tt\nameref{tftpStat}}
    \item Number of TFTP packets
    \item Number of files extracted ({\tt TFTP\_SAVE=1})
\end{itemize}

\end{document}
